%%%%%%%%%%%%%%%%%%%%%%%%%%%%%%%%%%%%%%%%%%%%%%%%%%
%%%%%%%%%%%%%%%%%%%%%%%%%%%%%%%%%%%%%%%%%%%%%%%%%%
%%%%%%%%%%%%%%%%%%%%%%%%%%%%%%%%%%%%%%%%%%%%%%%%%%

\documentclass{subnucbo}

\usepackage{graphicx}
% got figures? uncomment this

% lingua italiana
\usepackage[italian]{babel}

\usepackage[utf8x]{inputenc}


\title{Processi Deboli e Mixing dei Quark}
% Insert your title here



\author{Massimiliano Galli}
% Insert your name(s) here.  Quick note: Authors must be separated
% with \from{n} to mark their referring
% institute, e.g.:
% \author{A.U. Thor\from{1}, W.R. Iter\from{2} \atque A. Einstein\from{3}\thanks{Thanks here}}
% TeX automagically determines the institute you are referring to.





\instlist{\inst{} Dipartimento di Fisica Universit\`a di Bologna, Via Irnerio 46 - 40126 Bologna, Italy}
% Insert your institute(s) here. Quick note: as above, e.g.
%\instlist{\inst{1} Dipartimento di Fisica Universit\`a di Bologna, Via Irnerio 46 - 40126 Bologna, Italy
%\inst{2} Dipartimento di Fisica dell'Universit\`{a} and INFN, Sezione di Milano - Milano, Italy
%\inst{3} Societ\`a Italiana di Fisica}




\acyear{2017--2018}
% Insert the academic year




\begin{document}

\maketitle

\begin{abstract}
This sample paper is intended to briefly expose \LaTeX\ package \texttt{subnucbo}.
\end{abstract}

\section{Introduzione}
In questa sezione si introducono nozioni alle quali si farà riferimento in seguito.\\
Attraverso la teoria perturbativa è possibile ricavare la probabilità di transizione da uno stato iniziale \textit{i} ad uno stato finale \textit{f}:
\begin{equation}
        \Gamma = \frac{2 \pi} {\hbar} \cdot | M_{if}|^{2} \rho_{f}
        \label{eq:decay_rate}
\end{equation}
dove $\rho_{f}$ è la densità degli stati per unità di intervallo di energia, mentre $M_{if}$ è l'elemento di matrice per la probabilità di transizione ed è dato da:
\begin{equation}
        M_{if} = \int \phi^{*}_{i} V \phi_{f} d\tau
        \label{eq:matrix_element}
\end{equation}
con $V$ potenziale responsabile della transizione da uno stato all'altro.

\section{La teoria di Fermi del decadimento $\beta$}
Il decadimento beta dei nuclei rappresenta una trasmutazione di un elemento (\textit{Z}, \textit{N}), ove \textit{Z} è il numero di protoni ed \textit{N} quello di neutroni del nucleo verso un nucleo con $\textit{Z} + 1$ protoni (decadimento beta negativo) oppure con $\textit{Z} - 1$ protoni (decadimento beta positivo). Per descriverlo, Fermi sviluppò a partire dal 1934 una teoria basata sull'ipotesi di interazione puntiform,e tra quattro fermioni. Un valido prototipo per questo tipo di interazione è il decadimento del neutrone:
\begin{equation}
        n \rightarrow p e^{-} \overline{\nu_{e}}
        \label{eq:neutro_decay}
\end{equation}

\appendix

\section{}
Let us go then, you and I\ldots

\acknowledgments
This work was produced, supported and perpetrated by M. Bellacosa under
the auspices of the Italian Physical Society.

\begin{thebibliography}{0}
\bibitem{ref:apo} \BY{Boccaccio~G. \atque de~Cam\~oes~L.}
  \IN{Phys. Rev. A}{13}{1999}{12};
  \SAME{69}{999}{1666}.
\bibitem{ref:pul} \BY{Pulci~L.}
  preprint INFN 8181.
\bibitem{ref:bra} \BY{Bragg~B.}
  \TITLE{Tender comrade},
  in \TITLE{Workers Playtime},
                  edited by \NAME{Tizio A. \atque Caio B.}
                  (Unexeditor, Bologna) 1997, pp.~1-10.
\end{thebibliography}

\end{document}

%%
