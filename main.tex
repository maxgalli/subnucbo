%%%%%%%%%%%%%%%%%%%%%%%%%%%%%%%%%%%%%%%%%%%%%%%%%%
%%%%%%%%%%%%%%%%%%%%%%%%%%%%%%%%%%%%%%%%%%%%%%%%%%
%%%%%%%%%%%%%%%%%%%%%%%%%%%%%%%%%%%%%%%%%%%%%%%%%%
\documentclass{subnucbo}

\usepackage{graphicx}
% got figures? uncomment this

% lingua italiana
\usepackage[italian]{babel}
\usepackage[utf8x]{inputenc}

% feynman diagrams
\usepackage[compat=1.1.0]{tikz-feynman}

\title{Processi Deboli e Mixing dei Quark}
% Insert your title here

\author{Massimiliano Galli}
% Insert your name(s) here.  Quick note: Authors must be separated
% with \from{n} to mark their referring
% institute, e.g.:
% \author{A.U. Thor\from{1}, W.R. Iter\from{2} \atque A. Einstein\from{3}\thanks{Thanks here}}
% TeX automagically determines the institute you are referring to.

\instlist{\inst{} Dipartimento di Fisica Universit\`a di Bologna, Via Irnerio 46 - 40126 Bologna, Italy}
% Insert your institute(s) here. Quick note: as above, e.g.
%\instlist{\inst{1} Dipartimento di Fisica Universit\`a di Bologna, Via Irnerio 46 - 40126 Bologna, Italy
%\inst{2} Dipartimento di Fisica dell'Universit\`{a} and INFN, Sezione di Milano - Milano, Italy
%\inst{3} Societ\`a Italiana di Fisica}

\acyear{2017--2018}
% Insert the academic year


\begin{document}

\maketitle

\begin{abstract}
        This sample paper is intended to briefly expose \LaTeX\ package \texttt{subnucbo}.
\end{abstract}

\section{Introduzione}
In questa sezione si introducono nozioni alle quali si farà riferimento in seguito.\\
Attraverso la teoria perturbativa è possibile ricavare la probabilità di transizione da uno stato iniziale \textit{i} ad uno stato finale \textit{f}:
\begin{equation}
        \Gamma = \frac{2 \pi} {\hbar} \cdot | \mathcal{M}|^{2} \rho_{f}
        \label{eq:decay_rate}
\end{equation}
dove $\rho_{f}$ è la densità degli stati per unità di intervallo di energia (spazio delle fasi), mentre $\mathcal{M}$ è l'elemento di matrice per la probabilità di transizione e si calcola valutando i diagrammi di Feynman più rilevanti e applicando le regole opportune. La \ref{eq:decay_rate} è anche detta "regola d'oro di Fermi". \\
Inoltre è possibile dimostrare \cite{ref:griff} che per processi di decadimento è possibile scrivere la vita media della particella $\tau$ come:
\begin{equation}
        \tau = \frac{1}{\Gamma}
        \label{eq:tau}
\end{equation}
dove $\Gamma$ corrisponde in realtà alla sommatoria su tutti gli $n$ modi di decadimento della particella:
\begin{equation}
        \Gamma = \sum _ { i = 1 } ^ { n } \Gamma _ { i }
        \label{eq:gamma_sum}
\end{equation}
Si introduce infine la regola di Sargent~\cite{ref:BGS}: se $\tau$ è la vita media di una particella e $\Gamma_{i}/\Gamma$ è il suo \textit{branching ratio} in un particolare decadimento debole a tre corpi nello stato finale. allora la probabilità di transizione corrisponde a:
\begin{equation}
        W = \frac{(\Gamma_{i}/\Gamma)}{\tau} \simeq G_{F}^{2}E_{0}^{5} \simeq G_{F}^{2}\Delta m^{5}
        \label{eq:sargent_rule}
\end{equation}

\section{Verso l'universalità dell'interazione debole: la necessità del \textit{mixing} e l'intuizione di Cabibbo}
La teoria di Fermi, sviluppata a partire dal 1934, è considerata il prototipo dell'interazione debole: essa è puntiforme (in quanto coinvolge l'interazione di quattro fermioni in un punto) e presenta una costante di accoppiamento $G_{F}$, detta \textit{costante di Fermi}. Applicando la teoria a diversi decadimenti, come ad esempio nei casi di neutrone e muone, risulta evidente che $G_{F}$ calcolata in funzione dei risultati sperimentali dà risultati leggermente diversi. \\
Questo fatto, insieme ad alcune anomalie rilevate nei decadimenti deboli di particelle strane, portò alla necessità di aggiungere un nuovo tassello alla teoria.
\subsection{Decadimento del muone}
Il muone decade secondo:
\begin{equation}
        \mu^{-} \rightarrow e^{-} \overline{\nu_{e}} \nu_{\mu}
        \label{eq:muon_decay}
\end{equation}
Il diagramma di Feynman del processo è mostrato in Fig. \ref{fig:muon_decay}.
\begin{figure}[!h]
        \centering
        \feynmandiagram [layered layout, horizontal=a to b] {
                a [particle=\(\mu^{-}\)] -- [fermion] b -- [fermion] f1 [particle=\(\nu_{\mu}\)],
                b -- [boson, edge label'=\(W^{-}\)] c,
                c -- [anti fermion] f2 [particle=\(\overline \nu_{e}\)],
                c -- [fermion] f3 [particle=\(e^{-}\)],
        };
        \caption{Diagramma di Feynman per il decadimento del muone.}
        \label{fig:muon_decay}
\end{figure}
Per l'interazione debole, le regole di Feynman~\cite{ref:griff} prevedono il fattore V-A
\begin{equation}
        \mathcal { K } ^ { \mu } \rightarrow - \frac { i g _ { w } } { 2 \sqrt { 2 } } \gamma ^ { \mu } \left( 1 - \gamma ^ { 5 } \right)
\end{equation}
ad entrambi i vertici. A causa della grande massa del bosone $W^{-}$, il propagatore può essere approssimato da
\begin{equation}
        \mathcal { D } _ { \mu \nu } \rightarrow i \frac { g _ { \mu \nu } } { M _ { W } ^ { 2 } }
        \label{eq:propagator}
\end{equation}
Inserendo questi valori nell'espressione per l'ampiezza di decadimento si trova:
\begin{equation}
        \mathcal { M } = \frac { g _ { w } ^ { 2 } } { 8 M _ { W } ^ { 2 } } \left[ \overline { u } \left( \nu _ { \mu } \right) \gamma ^ { \mu } \left( 1 - \gamma ^ { 5 } \right) u ( \mu ) \right] \cdot \left[ \overline { u } \left( e ^ { - } \right) \gamma _ { \mu } \left( 1 - \gamma ^ { 5 } \right) v \left( \nu _ { e } \right) \right]
        \label{eq:amplitude_muon}
\end{equation}
dove $u$ e $v$ sono spinori di Dirac nello spazio dei momenti con $\overline { u } = u ^ { \dagger } \gamma ^ { 0 }$ e $\overline { v } = v ^ { \dagger } \gamma ^ { 0 }$. Sviluppando il calcolo e utilizzando la regola d'oro di Fermi si trova:
\begin{equation}
        \Gamma = \frac { m _ { \mu } ^ { 5 } c ^ { 4 } } { 192 \pi ^ { 3 } \hbar ^ { 7 } } G _ { F } ^ { 2 }
        \label{eq:muon_decay_rate}
\end{equation}
dove la costante di Fermi $G_{F}$ è definita da
\begin{equation}
        G _ { F } \equiv \frac { \sqrt { 2 } } { 8 } \left[ \frac { g _ { w } } { M _ { W } } \right] ^ { 2 } \cdot ( \hbar c ) ^ { 3 }
        \label{eq:fermi_constant}
\end{equation}
Ridefinendo poi
\begin{equation}
        G_{F} \leftrightarrow \frac{G_{F}}{\hbar^{3}c^{3}}
        \label{eq:fermi_constant_ridef}
\end{equation}
e sostituendo i valori sperimentali per la vita media del muone si ottiene:
\begin{equation}
        G^{\mu} _ { F } = 1.16637 \times 10 ^ { - 5 } \mathrm { GeV } ^ { - 2 }
        \label{eq:gf_muon_value}
\end{equation}

\subsection{Decadimento del neutrone}
Il neutrone decade secondo:
\begin{equation}
        n \rightarrow p e^{-} \overline{\nu_{e}}
        \label{eq:neutron_decay}
\end{equation}
Una prima approssimazione del decadimento beta del neutrone si ottiene assumendo neutrone e protone particelle puntiformi che si accoppiano direttamente con il bosone mediatore (Fig.~\ref{fig:neutron_decay_simple}).
\begin{figure}[!h]
        \centering
        \feynmandiagram [layered layout, horizontal=a to b] {
                a [particle=\(n\)] -- [fermion] b -- [fermion] f1 [particle=\(p\)],
                b -- [boson, edge label'=\(W^{-}\)] c,
                c -- [anti fermion] f2 [particle=\(\overline \nu_{e}\)],
                c -- [fermion] f3 [particle=\(e^{-}\)],
        };
        \caption{Diagramma di Feynman per il decadimento del neutrone, assumendo protone e neutrone non costituiti da quark.}
        \label{fig:neutron_decay_simple}
\end{figure}
L'elemento di matrice è determinato facilmente sostituendo $u(\nu_{\mu})$ e $u(\mu)$ in \ref{eq:amplitude_muon} con $u(n)$ e $u(p)$. Omettendo i dettagli del calcolo, che sono tuttavia descritti in \cite{ref:hayes}, si ricava:
\begin{equation}
        \Gamma = \frac { 2 \rho_{f} } { \pi ^ { 3 } \hbar ^ { 7 } } G _ { F } ^ { 2 } m _ { e } ^ { 5 } c ^ { 4 }
        \label{eq:neutron_decay_rate}
\end{equation}
Calcolando il valore di $\rho_{f}$ \cite{ref:BGSex} e sostituendo a $\Gamma$ l'inverso della vita media del neutrone $\tau_{n}=885.7 s$ si trova un valore di $G_{F}$ pari a:
\begin{equation}
        G^{n} _ { F } = 1.140 \times 10 ^ { - 5 } \mathrm { GeV } ^ { - 2 }
        \label{eq:gf_neutron_value}
\end{equation}
Questo valore discorda da quello ottenuto (Eq.~\ref{eq:gf_muon_value}) da un decadimento che coinvolge solo leptoni. Sembrò quindi inizialmente che interazioni coinvolgenti leptoni e quark fossero diverse, facendo così cadere l'ipotesi di universalità dell'interazione debole.

\subsection{Decadimenti deboli di particelle strane}
Poiché le interazioni deboli coinvolgono sia leptoni che adroni, occorre fare alcune classificazioni, basate sul fatto che nei processi dovuti alla WI siano coinvolti o meno dei leptoni. Nel caso di processi semi-leptonici o non-leptonici, si osservano decadimenti in cui non sono coinvolte particelle strane ($\Delta S=0$), oppure con decadimenti che violano la stranezza ($\Delta S=1$). Il decadimento debole delle particelle strane presenta alcune anomalie, indipendentemente che si considerino processi leptonici, semi-leptonici o non-leptonici.
\subsubsection{Decadimenti leptonici}
Si consideri il caso dei decadimenti puramente leptonici riportati in Tab.~\ref{tab:leptonic_decays}.
\begin{table}[!h]
        \begin{tabular}{llccc}
                \hline
                Decadimento & & $\Delta S$ & \tau\: (s)& $BR = \Gamma_{i}/\Gamma$    \\
                \hline
                $\pi^{-} \rightarrow \mu^{-} \overline{\nu_{\mu}}$ & $\overline{u}d \rightarrow W^{-} \rightarrow \mu^{-} \overline{\nu_{\mu}}$ & 0 & $2.6 \times 10^{-8}$ & $100\%$ \\
                $K^{-} \rightarrow \mu^{-} \overline{\nu_{\mu}}$ & $\overline{u}s \rightarrow W^{-} \rightarrow \mu^{-} \overline{\nu_{\mu}}$ & 1 & $1.27 \times 10^{-8}$ & $63.5\%$ \\
                \hline
        \end{tabular}
        \caption{Decadimenti leptonici di pione e kaone.}
        \label{tab:leptonic_decays}
\end{table}

\subsubsection{Decadimenti semi-leptonici}
In questo caso si ha che gli stati finali comprendono sia leptoni che adroni. Si considerino i decadimenti della $\Sigma^{-}$ mostrati in Tab.~\ref{tab:isemileptonic_decays}: la regola di Sargent (\ref{eq:sargent_rule}) fornisce una buona approssimazione per il calcolo della vita media. Tuttavia, anche in questo caso i risultati numerici sono corretti per i decadimenti con $\Delta S = 0$, ed errati di un fattore \sim 20 per i decadimenti con $\Delta S = 1$ . Di nuovo, si può immaginare che nel caso di decadimenti senza variazione di stranezza intervenga la costante $G_{d}$, mentre in quelli con $\Delta S = 1$ intervenga la costante $G_{s}$. Si può stimare il rapporto tra le due costanti usando i decadimenti della $\Sigma^{-}$:
\begin{equation}
        \frac{G_{s}^{2}}{G_{d}^{2}} = \frac{\Gamma(\Sigma^{-} \rightarrow n e^{-} \overline{\nu_{e}})/\Delta m^{5}_{\Delta S = 1}}{\Gamma(\Sigma^{-} \rightarrow \Lambda^{0} e^{-} \overline{\nu_{e}})/\Delta m^{5}_{\Delta S = 0}} = 0.057
        \label{eq:ratio_semileptonic}
\end{equation}

\begin{table}[!h]
        \begin{tabular}{llccc}
                \hline
                Decadimento & \Delta m\: (MeV) & $\Delta S$ & \tau\: (s)& $BR = \Gamma_{i}/\Gamma$    \\
                \hline
                $\Sigma^{-} \rightarrow \Lambda^{0} e^{-} \overline{\nu_{e}}$ & 81.7 & 0 & $1.48 \times 10^{-10}$ & $0.57 \times 10^{-4}$ \\
                $\Sigma^{-} \rightarrow n e^{-} \overline{\nu_{e}}$ & 257.8 & 1 & $1.48 \times 10^{-10}$ & $1.02 \times 10^{-3}$ \\
                \hline
        \end{tabular}
        \caption{Decadimenti semi-leptonici della $\Sigma^{-}$.}
        \label{tab:isemileptonic_decays}
\end{table}


\subsection{L'angolo di Cabibbo}
I fatti sperimentali sopracitati vennero interpretati da Nicola Cabibbo nel 1964. Egli mostrò che sia i leptoni che i quark sono autostati dell'interazione debole, con le seguenti assunzioni:
\begin{itemize}
        \item l'accoppiamento degli elettroni al campo debole è proporzionale ad una carica debole $g_{e\nu}$;
        \item l'accoppiamento dei muoni è proporzionale a $g_{\mu\nu}$, con quest'ultima identica a $g_{e\nu}$;
        \item l'accoppiamento dei quark $(u, d)$ genera le transizioni con $\Delta S = 0$ ed è proporzionale a $g_{ud}$;
        \item l'accoppiamento dei quark $(u, s)$ genera le transizioni con $\Delta S = 1$ ed è proporzionale a $g_{us}$;
\end{itemize}
In ogni vertice di un diagramma di Feynman occorre inserire la costante corrispondente, con il calcolo degli elementi di matrice che viene di conseguenza:
\begin{itemize}
        \item per i processi puramente leptonici si ha
                \begin{equation}
                        \left\langle f \left| H _ { W } \right| i \right\rangle \propto g _ { e v } ^ { 2 } = G _ { F }
                        \label{eq:genu}
                \end{equation}
        \item per i processi semi-leptonici con $\Delta S = 0$ si ha
                \begin{equation}
                        \left\langle f \left| H _ { W } \right| i \right) _ { \Delta S = 0 } \propto g _ { e v } g _ { u d } = G _ { d }
                        \label{eq:gd}
                \end{equation}
        \item per i processi semi-leptonici con $\Delta S = 1$ si ha
                \begin{equation}
                        \left\langle f \left| H _ { W } \right| i \right\rangle _ { \Delta S = 1 } \propto g _ { e v } g _ { u s } = G _ { s }
                        \label{eq:gs}
                \end{equation}
\end{itemize}
L'ipotesi di Cabibbo è che l'interazione debole dipenda da un solo parametro, la costante di Fermi $G_{F}$. Questa descrive l'accoppiamento del campo debole sia verso i leptoni che verso i quark tramite la relazione:
\begin{equation}
        G _ { F } = g _ { e v } ^ { 2 } = g _ { u d } ^ { 2 } + g _ { u s } ^ { 2 } \longrightarrow g _ { u d } = g _ { e v } \cos \theta _ { C } ; \quad g _ { u s } = g _ { e v } \sin \theta _ { C }
\end{equation}









\appendix

\section{}
Let us go then, you and I\ldots

\acknowledgments
This work was produced, supported and perpetrated by M. Bellacosa under
the auspices of the Italian Physical Society.

\begin{thebibliography}{0}
        \bibitem{ref:BGS}
                \BY{Braibant S., Giacomelli G., Spurio M.}
                \TITLE{Particelle e interazioni fondamentali},
                \PUBLISHER{Springer},
                \YEAR{2012}
        \bibitem{ref:griff}
                \BY{Griffiths D.}
                \TITLE{Introduction to elementary particles},
                \YEAR{1987}
        \bibitem{ref:hayes}
                \BY{Hayes C.B.}
                \TITLE{Neutron Beta-Decay},
                \YEAR{2012}
        \bibitem{ref:BGSex}
                \BY{Braibant S., Giacomelli G., Spurio M.}
                \TITLE{Particles and Fundamental Interactions: Supplements, Problems and Solutions},
                \PUBLISHER{Springer},
                \YEAR{2012}
\end{thebibliography}

\end{document}

%%
