%%%%%%%%%%%%%%%%%%%%%%%%%%%%%%%%%%%%%%%%%%%%%%%%%%
%%%%%%%%%%%%%%%%%%%%%%%%%%%%%%%%%%%%%%%%%%%%%%%%%%
%%%%%%%%%%%%%%%%%%%%%%%%%%%%%%%%%%%%%%%%%%%%%%%%%%
\documentclass{subnucbo}

\usepackage{graphicx}
% got figures? uncomment this

% lingua italiana
\usepackage[italian]{babel}
\usepackage[utf8x]{inputenc}

% feynman diagrams
\usepackage[compat=1.1.0]{tikz-feynman}

% math
\usepackage{amsmath}

\title{Processi Deboli e Mixing dei Quark}
% Insert your title here

\author{Massimiliano Galli}
% Insert your name(s) here.  Quick note: Authors must be separated
% with \from{n} to mark their referring
% institute, e.g.:
% \author{A.U. Thor\from{1}, W.R. Iter\from{2} \atque A. Einstein\from{3}\thanks{Thanks here}}
% TeX automagically determines the institute you are referring to.

\instlist{\inst{} Dipartimento di Fisica Universit\`a di Bologna, Via Irnerio 46 - 40126 Bologna, Italy}
% Insert your institute(s) here. Quick note: as above, e.g.
%\instlist{\inst{1} Dipartimento di Fisica Universit\`a di Bologna, Via Irnerio 46 - 40126 Bologna, Italy
%\inst{2} Dipartimento di Fisica dell'Universit\`{a} and INFN, Sezione di Milano - Milano, Italy
%\inst{3} Societ\`a Italiana di Fisica}

\acyear{2017--2018}
% Insert the academic year


\begin{document}

\maketitle

\begin{abstract}
        Uno degli aspetti caratteristici dell'interazione debole è il cosiddetto \textit{quark mixing}: questa teoria giustifica i risultati sperimentali postulando che i quark che compaiono nei processi deboli non siano gli autostati di \textit{flavor} che caratterizzano l'interazione forte, ma una loro combinazione lineare. \\
        Questo lavoro è diviso in due parti principali (che seguono una breve introduzione in \ref{sec:intro}): nella prima (\ref{sec:part_one}), dopo una descrizione dei primi modelli teorici ideati per l'interazione debole (\ref{subsec:fermi_theory}, \ref{subsec:v-a} e \ref{subsec:fermi_gt}), si analizzano le evidenze sperimentali che richiesero un'estensione della teoria (\ref{subsec:decays} e \ref{subsec:strange}) e si espone l'idea di Nicola Cabibbo di implementare un angolo di \textit{mixing} (\ref{subsec:cabibbo_angle}).
        Nella seconda parte (\ref{sec:part_two}) si descrive invece come la teoria del \textit{mixing}, opportunamente estesa anche alle correnti neutre (\ref{subsec:nc} e \ref{subsec:nc_charm}), portò alla predizione dei quark \textit{charm} (\ref{subsec:gim}), \textit{bottom} e \textit{top} (\ref{subsec:ckm}), con la definitiva condensazione della teoria nella matrice CKM.
\end{abstract}

\section{Introduzione}
\label{sec:intro}
\subsection{Caratteristiche generali dell'interazione debole}
L'interazione debole presenta una serie di proprietà che la rendono peculiare rispetto alle altre interazioni fondamentali:
\begin{itemize}
        \item è l'unica interazione capace di cambiare il \textit{flavor} dei quark;
        \item non conserva alcune quantità che sono invece conservate nelle altre interazioni, quali ad esempio parità, coniugazione di carica e stranezza;
        \item è la sola interazione che coinvolge i neutrini;
        \item è responsabile del decadimento beta dei nuclei.
\end{itemize}
Essa presenta inoltre dei bosoni mediatori ($W^{+}, W^{-}, Z^{0}$) molto massivi \cite{ref:PDG}:
\begin{equation}
        M _ { W } = 80.385 \pm 0.015 \text { GeV } / c ^ { 2 } , \quad M _ { Z } = 91.1876 \pm 0.0021 \text { GeV } / c ^ { 2 }
        \label{eq:bosons_masses}
\end{equation}
i quali fanno sì che l'interazione debole abbia un cortissimo raggio d'azione. I bosoni $W^{+}$ e $W^{-}$ sono responsabili dell'interazione debole a corrente carica (CC) mentre $Z^{0}$ dell'interazione debole a corrente neutra (NC). I vertici dell'interazione debole sono rappresentati in Fig.~\ref{fig:weak_vertex}. \\
La comprensione della struttura dell'interazione debole si è sviluppata ed evoluta nel corso di vari anni, complicata dalle numerose violazioni di leggi di conservazione che la caratterizzano. Il punto di partenza fu quello di interpretare in un modello fenomenologico il decadimento beta.
\begin{figure}[!h]
        \centering
        \subfloat{
                \feynmandiagram [layered layout, horizontal=a to b] {
                        a [particle=\(l^{-}\)] -- [fermion] b -- [fermion] c [particle=\(\nu_{l}\)],
                        b -- [boson, edge label'=\(W^{-}\)] d,
                };
        }
        \qquad
        \qquad
        \subfloat{
                \feynmandiagram [layered layout, horizontal=a to b] {
                        a [particle=\(l^{+}\)] -- [fermion] b -- [fermion] c [particle=\(\overline{\nu_{l}}\)],
                        b -- [boson, edge label'=\(W^{+}\)] d,
                };
        } \\
        \subfloat{
                \feynmandiagram [layered layout, horizontal=a to b] {
                        a [particle=\(q^{-1/3}\)] -- [fermion] b -- [fermion] c [particle=\(q^{+2/3}\)],
                        b -- [boson, edge label'=\(W^{-}\)] d,
                };
        }
        \qquad
        \qquad
        \subfloat{
                \feynmandiagram [layered layout, horizontal=a to b] {
                        a [particle=\(q^{+1/3}\)] -- [fermion] b -- [fermion] c [particle=\(q^{-2/3}\)],
                        b -- [boson, edge label'=\(W^{+}\)] d,
                };
        } \\
        \subfloat{
                \feynmandiagram [layered layout, horizontal=a to b] {
                        a [particle=\(l\)] -- [fermion] b -- [fermion] c [particle=\(l\)],
                        b -- [boson, edge label'=\(Z^{0}\)] d,
                };
        }
        \qquad
        \qquad
        \subfloat{
                \feynmandiagram [layered layout, horizontal=a to b] {
                        a [particle=\(q\)] -- [fermion] b -- [fermion] c [particle=\(q\)],
                        b -- [boson, edge label'=\(Z^{0}\)] d,
                };
        }
        \caption{Vertici dell'interazione debole. Le lettere \textit{l} e \textit{q} rappresentano rispettivamente leptoni e quark. Nei processi in cui compare una $Z^{0}$ non c'è differenza di carica tra le altre due particelle che costituiscono il vertice; al contrario, si può vedere che c'è variazione di carica quando compaiono $W^{+}$ e $W^{-}$. }
        \label{fig:weak_vertex}
\end{figure}

\subsection{Nozioni fondamentali}
In questa parte si introducono nozioni alle quali si farà riferimento in seguito.\\
Attraverso la teoria perturbativa è possibile ricavare la probabilità di transizione da uno stato iniziale \textit{i} ad uno stato finale \textit{f}:
\begin{equation}
        \Gamma = \frac{2 \pi} {\hbar} | \mathcal{M}|^{2} \frac{dN}{dE_{f}}
        \label{eq:decay_rate}
\end{equation}
dove $dN/dE_{f}$ è la densità degli stati per unità di intervallo di energia (spazio delle fasi), mentre $\mathcal{M}$ è l'elemento di matrice per la probabilità di transizione e si calcola valutando i diagrammi di Feynman più rilevanti e applicando le regole opportune. Nel caso in cui il processo descritto sia il decadimento di una particella, la grandezza (\ref{eq:decay_rate}) coincide con il tasso di decadimento. Il fattore spazio delle fasi contiene solo informazioni di natura cinematica; dipende da masse, energie e momenti dei partecipanti e riflette il fatto che un dato processo è tanto più probabile che avvenga quanto più "margine di manovra" possiede nello stato finale. La (\ref{eq:decay_rate}) è anche detta regola d'oro di Fermi. \\
Si può inoltre dimostrare \cite{ref:griff} che per processi di decadimento è possibile scrivere la vita media della particella $\tau$ come:
\begin{equation}
        \tau = \frac{1}{\Gamma}
        \label{eq:tau}
\end{equation}
dove $\Gamma$ corrisponde in realtà alla sommatoria su tutti gli $n$ modi di decadimento della particella:
\begin{equation}
        \Gamma = \sum _ { i = 1 } ^ { n } \Gamma _ { i }
        \label{eq:gamma_sum}
\end{equation}

\section{Verso l'universalità dell'interazione debole: la necessità del \textit{mixing} e l'intuizione di Cabibbo}
\label{sec:part_one}
\subsection{Teoria di Fermi del decadimento $\beta$}
\label{subsec:fermi_theory}
Il decadimento beta è la forma più comune di processo debole osservato nella materia ordinaria, ed ha costituito per molto tempo l'unico banco di prova utilizzabile per confrontare teoria ed esperimenti. Le transizioni beta che coinvolgono i nucleoni sono essenzialmente tre:
\begin{subequations}
        \begin{equation}
                n \rightarrow p  e ^ { - }  \overline { \nu } _ { e }
                \label{eq:beta_neg}
        \end{equation}
        \begin{equation}
                p \rightarrow n  e ^ { + }  \nu _ { e }
                \label{eq:beta_pos}
        \end{equation}
        \begin{equation}
                e ^ { - }  p \rightarrow n  \nu _ { e }
                \label{eq:el_capture}
        \end{equation}
\end{subequations}
che, in termini di transizioni nucleari, possono riscriversi:
\begin{subequations}
        \begin{equation}
                ( A , Z ) \rightarrow ( A , Z + 1 )  e ^ { - }  \overline { \nu } _ { e }
                \label{eq:beta_neg_trans}
        \end{equation}
        \begin{equation}
                ( A , Z ) \rightarrow ( A , Z - 1 )  e ^ { + }  \nu _ { e }
                \label{eq:beta_pos_trans}
        \end{equation}
        \begin{equation}
                e ^ { - }  ( A , Z ) \rightarrow ( A , Z - 1 )  \nu _ { e }
                \label{eq:el_capture_trans}
        \end{equation}
\end{subequations}
Il primo processo, detto decadimento beta negativo, determina l'instabilità del neutrone mentre il secondo è detto decadimento beta positivo e non avviene in maniera spontanea perché energicamente proibito; il terzo consiste nella cattura di un elettrone atomico da parte del nucleo. \\
Il primo modello che abbia descritto con successo diverse caratteristiche del decadimento beta fu sviluppato da Fermi nel 1934. Questa teoria è considerata il prototipo dell'interazione debole: è puntiforme (in quanto coinvolge l'interazione di quattro fermioni in un punto) e presenta una costante di accoppiamento $G_{F}$, detta costante di Fermi. La forza della teoria di Fermi sta nel fatto che la costante $G_{F}$ è universale, valida cioè per tutti i decadimenti di questo tipo. Un possibile vertice d'interazione postulato dalla teoria di Fermi è mostrato in Fig. \ref{fig:fermi_decay}.\\
\begin{figure}[!h]
        \centering
        \feynmandiagram [layered layout, horizontal= nuc1 to a ] {
                nuc1 [particle =\(^{A}_{Z}X\)]-- [fermion] a,
                a -- [anti fermion] neutrino [particle =\(\overline \nu_{e}\)],
                a -- [fermion] nuc2 [particle =\(^{A}_{Z+1}Y\)],
                a -- [fermion] el [particle=\(e^{-}\)],
        };
        \caption{Vertice a quattro fermioni della teoria di Fermi raffigurante un decadimento di tipo beta negativo.}
        \label{fig:fermi_decay}
\end{figure}
Prendendo come esempio il decadimento del neutrone, l'hamiltoniana predetta dalla teoria di Fermi si può scrivere nella seguente maniera:
\begin{equation}
        H _ { \mathrm { F } } = H _ { \mathrm { n } } ^ { 0 } + H _ { \mathrm { p } } ^ { 0 } + H _ { \mathrm { e } } ^ { 0 } + H _ { \nu } ^ { 0 } + \underbrace { \sum _ { i } C _ { i } \int \mathrm { d } ^ { 3 } x \left( \overline { \psi } _ { \mathrm { p } } \hat { O } _ { i } \psi _ { \mathrm { n } } \right) \left( \overline { \psi } _ { \mathrm { e } } \hat { O } _ { i } \psi _ { \nu } \right) } _ { \text { interazione } }
        \label{eq:fermi_hamiltonian}
\end{equation}
dove $\psi_{p}$, $\psi_{n}$, $\psi_{e}$ e $\psi_{\nu}$ sono le funzioni d'onda spinoriali delle quattro particelle, tali da soddisfare l'equazione di Dirac (Appendice \ref{app:dirac}), mentre i termini tra parentesi tonde sono quadrivettori densità di corrente che si possono scrivere in maniera generale nella seguente forma:
\begin{equation}
        J ^ { \mu } = \overline { \psi } \hat{O} _ { i } \psi
        \label{eq:current}
\end{equation}
Le quantità $\hat{O}_{i}$ sono operatori appropriati che caratterizzano il tipo di interazione, combinazione lineare delle matrici di Dirac $\gamma^{\mu}$.
Le (\ref{eq:current}) sono forme bilineari che si trasformano sotto trasformazioni di Lorentz in modo analogo ad una quantità scalare (S), pseudo-scalare (P), vettoriale (V), assiale (A) e tensoriale (T); a seconda dell'operatore scelto, l'indice in $\hat{O}_{i}$ può assumere i valori $i = S, P, V, T, A$. Le possibili correnti sono riportate in Tab~\ref{tab:bilinear}.
\begin{table}
        \centering
        \begin{tabular}{l  c  c}
                \hline
                & Corrente & Numero di Componenti \\
                \hline
                Scalare & $\overline { \psi } \psi$ & 1 \\
                Vettore & $\overline { \psi } \gamma ^ { \mu } \psi$ & 4 \\
                Tensore & $\overline { \psi } \sigma ^ { \mu \nu } \psi$ & 6 \\
                Vettore assiale & $\overline { \psi } \gamma ^ { \mu } \gamma ^ { 5 } \psi$ & 4 \\
                Pseudo-scalare & \overline { \psi } \gamma ^ { 5 } \psi & 1 \\
                \hline
        \end{tabular}
        \caption{Forme bilineari.}
        \label{tab:bilinear}
\end{table}
Fermi, che sviluppò la teoria in analogia a quella elettromagnetica, ipotizzò inizialmente che queste quantità fossero solo vettoriali, ma alcune osservazioni sperimentali degli anni successivi mostrarono che non era così.
\subsection{Teoria V-A delle interazioni deboli}
\label{subsec:v-a}
La teoria $V-A$ dell'interazione debole venne sviluppata a partire dal 1957 da Feynmann e Gell-Mann come estensione della teoria di Fermi: ciò che rese necessario un aggiornamento della teoria fu la scoperta, avvenuta alcuni anni prima, della violazione della parità da parte dell'interazione debole. Così come la teoria di Fermi, essa si basa sui seguenti fatti:
\begin{itemize}
        \item per particelle con spin 1/2, le funzioni d'onda appropriate sono spinori a quattro componenti che soddisfano l'equazione di Dirac;
        \item l'ampiezza del processo è proporzionale al quadrivettore densità di corrente.
\end{itemize}
Come si può intuire dal nome, la teoria $V-A$ asserisce che le correnti che compaiono nelle interazioni deboli sono combinazioni di soli termini vettoriali e assiali. Questa scelta è dovuta ai seguenti motivi:
\begin{itemize}
        \item correnti vettoriali e assiali, prese singolarmente, conservano la parità, mentre una loro combinazione la viola;
        \item l'integrazione nella teoria di fenomeni osservati sperimentalmente, quali l'elicità di neutrini ed elettroni, consente di escludere quantità scalari, pseudo-scalari e tensoriali.
\end{itemize}
Si approfondiscono nel seguito i fatti appena citati.
\subsubsection{Violazione della parità da parte di corrente $V-A$}
Tenendo presente che l'elemento di matrice per l'interazione debole si può scrivere in generale nella seguente maniera:
\begin{equation}
        \mathcal{M} \propto \eta_{\mu\nu}J^{\mu}J^{\nu}
        \label{eq:weak_matrix_element}
\end{equation}
e ricordando (Appendice \ref{app:dirac}) che gli spinori $\psi$ e $\overline{\psi} = \psi ^ { + } \gamma ^ { 0 }$ si trasformano sotto l'operazione di parità come
\begin{equation}
        \psi \stackrel { P } { \rightarrow } \gamma ^ { 0 } \psi \quad,\quad \overline { \psi } \stackrel { P } { \rightarrow } \left( \gamma ^ { 0 } \psi \right) ^ { + } \gamma ^ { 0 } = \psi ^ { + } \gamma ^ { 0 } \gamma ^ { 0 } = \overline { \psi } \gamma ^ { 0 }
\end{equation}
è possibile dimostrare che correnti puramente vettoriali o assiali conservano la parità, mentre una loro combinazione la viola.
\paragraph{Corrente vettoriale} La corrente vettoriale $V ^ { \mu } = \overline { \psi } \gamma ^ { \mu } \psi$ si trasforma sotto parità come:
\begin{equation}
        \overline { \psi } \gamma ^ { \mu } \psi \stackrel { P } { \rightarrow } \left( \overline { \psi } \gamma ^ { 0 } \right) \gamma ^ { \mu } \left( \gamma ^ { 0 } \psi \right)
\end{equation}
Per le coordinate tempo e spazio si ha separatamente:
\begin{equation}
        \begin{aligned}
                \overline { \psi } \gamma ^ { 0 } \psi \stackrel { P } { \rightarrow } \left( \overline { \psi } \gamma ^ { 0 } \right) \gamma ^ { 0 } \left( \gamma ^ { 0 } \psi \right) & = \overline { \psi } \gamma ^ { 0 } \psi \\ \overline { \psi } \gamma ^ { k } \psi \stackrel { P } { \rightarrow } \left( \overline { \psi } \gamma ^ { 0 } \right) \gamma ^ { k } \left( \gamma ^ { 0 } \psi \right) & = - \overline { \psi } \gamma ^ { k } \psi
        \end{aligned}
\end{equation}
con $k=1,2,3$, dalle quali si vede che le componenti spaziali cambiano segno mentre quella temporale no. L'elemento di matrice (\ref{eq:weak_matrix_element}) si trasforma quindi sotto parità come:
\begin{equation}
        \mathcal{M} \propto V _ { 1 } ^ { 0 } V _ { 2 } ^ { 0 } - \vec { V } _ { 1 } \vec { V } _ { 2 } \stackrel { P } { \rightarrow } V _ { 1 } ^ { 0 } V _ { 2 } ^ { 0 } - \left( - \vec { V } _ { 1 } \right) \left( - \vec { V } _ { 2 } \right) = V _ { 1 } ^ { 0 } V _ { 2 } ^ { 0 } - \vec { V } _ { 1 } \vec { V } _ { 2 }
\end{equation}
conservando la parità.
\paragraph{Corrente assiale} La corrente assiale $A ^ { \mu } = \overline { \psi } \gamma ^ { \mu } \gamma ^ { 5 } \psi$ si trasforma sotto parità come:
\begin{equation}
        \overline { \psi } \gamma ^ { \mu } \gamma ^ { 5 } \psi \stackrel { P } { \rightarrow } \left( \overline { \psi } \gamma ^ { 0 } \right) \gamma ^ { \mu } \gamma ^ { 5 } \left( \gamma ^ { 0 } \psi \right)
\end{equation}
Per le coordinate tempo e spazio si ha separatamente:
\begin{equation}
        \begin{array} { l } { \overline { \psi } \gamma ^ { 0 } \gamma ^ { 5 } \psi \stackrel { P } { \rightarrow } \left( \overline { \psi } \gamma ^ { 0 } \right) \gamma ^ { 0 } \gamma ^ { 5 } \left( \gamma ^ { 0 } \psi \right) = - \overline { \psi } \gamma ^ { 0 } \gamma ^ { 5 } \psi } \\ { \overline { \psi } \gamma ^ { k } \gamma ^ { 5 } \psi \stackrel { P } { \rightarrow } \left( \overline { \psi } \gamma ^ { 0 } \right) \gamma ^ { k } \gamma ^ { 5 } \left( \gamma ^ { 0 } \psi \right) = \overline { \psi } \gamma ^ { k } \gamma ^ { 5 } \psi } \end{array}
\end{equation}
con $k=1,2,3$, dalle quali si vede che la componente temporale cambia segno mentre quelle spaziali no. L'elemento di matrice (\ref{eq:weak_matrix_element}) si trasforma quindi sotto parità come:
\begin{equation}
        \begin{aligned}
                \mathcal{M} & = A _ { 1 } ^ { 0 } A _ { 2 } ^ { 0 } - \vec { A } _ { 1 } \vec { A } _ { 2 } \\ \stackrel { P } { \rightarrow } \mathcal{M} & = \left( - A _ { 1 } ^ { 0 } \right) \left( - A _ { 2 } ^ { 0 } \right) - \vec { A } _ { 1 } \vec { A } _ { 2 } = A _ { 1 } ^ { 0 } A _ { 2 } ^ { 0 } - \vec { A } _ { 1 } \vec { A } _ { 2 }
        \end{aligned}
\end{equation}
conservando anche in questo caso la parità.
\paragraph{Corrente vettoriale - assiale} L'elemento di matrice di una teoria $V-A$ è:
\begin{equation}
        \begin{aligned}
                \mathcal{M} & \propto \eta _ { \mu \nu } \left( V _ { 1 } ^ { \mu } - A _ { 1 } ^ { \mu } \right) \left( V _ { 2 } ^ { \mu } - A _ { 2 } ^ { \mu } \right) \\ & = \left( V _ { 1 } ^ { 0 } - A _ { 1 } ^ { 0 } \right) \left( V _ { 2 } ^ { 0 } - A _ { 2 } ^ { 0 } \right) - \left( \vec { V } _ { 1 } - \vec { A } _ { 1 } \right) \left( \vec { V } _ { 2 } - \vec { A } _ { 2 } \right)
        \end{aligned}
\end{equation}
che sotto parità si trasforma come:
\begin{equation}
        \begin{aligned}
                \mathcal{M} & = \left( V _ { 1 } ^ { 0 } - A _ { 1 } ^ { 0 } \right) \left( V _ { 2 } ^ { 0 } - A _ { 2 } ^ { 0 } \right) - \left( \vec { V } _ { 1 } - \vec { A } _ { 1 } \right) \left( \vec { V } _ { 2 } - \vec { A } _ { 2 } \right) \\ \stackrel { P } { \longrightarrow } \mathcal{M} & = \left( V _ { 1 } ^ { 0 } + A _ { 1 } ^ { 0 } \right) \left( V _ { 2 } ^ { 0 } + A _ { 2 } ^ { 0 } \right) - \left( - \vec { V } _ { 1 } - \vec { A } _ { 1 } \right) \left( - \vec { V } _ { 2 } - \vec { A } _ { 2 } \right) \\ & = \left( V _ { 1 } ^ { 0 } + A _ { 1 } ^ { 0 } \right) \left( V _ { 2 } ^ { 0 } + A _ { 2 } ^ { 0 } \right) - \left( \vec { V } _ { 1 } + \vec { A } _ { 1 } \right) \left( \vec { V } _ { 2 } + \vec { A } _ { 2 } \right)
        \end{aligned}
\end{equation}
\subsubsection{Esclusione di S, T, P}
Si vuole ora mostrare come l'inclusione di osservazioni sperimentali quali le elicità dei leptoni porti ad escludere l'utilizzo di quantità scalari, tensoriali e pseudoscalari. Si utilizza come esempio una corrente che trasforma elettrone in neutrino, come quella che compare nel decadimento $\beta$ del neutrone:
\begin{equation}
        \overline { \psi } _ { \mathrm { e } } \hat { O } _ { i } \psi _ { \nu }
        \label{eq:e_nu_current}
\end{equation}
Facendo riferimento ai risultati esposti in Appendice \ref{app:dirac} si ha:
\begin{equation}
        \overline { \psi} _ {e} \hat { O } _ { i } \psi _ { \nu } \rightarrow \overline { \left( \hat { P } _ { L }  \psi_ {e} \right) } \hat { O } _ { i } \left( \hat { P } _ { L } \psi _ { \nu } \right)
\end{equation}
Dato che $\gamma ^ { 5 } ^ { \dagger } = \gamma ^ { 5 }$ e $\gamma ^ { 5 } \gamma ^ { 0 } = - \gamma ^ { 0 } \gamma ^{ 5 }$, si trova:
\begin{equation}
        \begin{aligned}
                \overline {\left( \hat { P } _ { L } \psi \right)} & = \left( \hat { P } _ { L }  \psi \right) ^ { \dagger } \gamma ^ { 0 } = \psi ^ { \dagger } \hat { P } _ { L } ^ { \dagger } \gamma ^ { 0 } \\ & = \psi ^ { \dagger } \left( \frac { 1 - \gamma ^ { 5 } } { 2 } \right) ^ { \dagger } \gamma ^ { 0 } = \psi ^ { \dagger } \frac { 1 - \gamma ^ { 5 } } { 2 } \gamma ^ { 0 } \\ & = \psi ^ { \dagger } \gamma ^ { 0 } \frac { 1 + \gamma ^ { 5 } } { 2 } = \overline { \psi } \hat { P } _ { R }
        \end{aligned}
\end{equation}
La forma modificata della corrente sarà dunque:
\begin{equation}
        \overline { \psi } _ {e} \hat { O } _ { i } ^ { \prime } \psi _ { \nu }
\end{equation}
con
\begin{equation}
        \hat { O } _ { i } ^ { \prime } = \hat { P } _ { R } \hat { O } _ { i } \hat { P } _ { L }
\end{equation}
Calcolando $\hat { O } _ { i } ^ { \prime }$ per i cinque operatori di Tab. \ref{tab:bilinear} si trova:
\begin{subequations}
        \begin{equation}
                \hat { P } _ { R } 1  \hat { P } _ { L } = 0
                \label{subeq:psp}
        \end{equation}
        \begin{equation}
                \hat { P } _ { R } \gamma ^ { \mu } \hat { P } _ { L } = \gamma ^ { \mu } \left( \hat { P } _ { L } \right) ^ { 2 } = \gamma ^ { \mu } \hat { P } _ { L }
                \label{subeq:pvp}
        \end{equation}
        \begin{equation}
                \begin{aligned}
                        \hat { P } _ { R } \sigma ^ { \mu \nu } \hat { P } _ { L }  & = \frac { \mathrm { i } } { 2 } \hat { P } _ { R } \left( \gamma ^ { \mu } \gamma ^ { \nu } - \gamma ^ { \nu } \gamma ^ { \mu } \right) \hat { P } _ { L } \\ & = \frac { \mathrm { i } } { 2 } \left( \gamma ^ { \mu } \hat { P } _ { L } \hat { P } _ { R }  \gamma ^ { \nu } - \gamma ^ { \nu } \hat { P } _ { L } \hat { P } _ { R } \gamma ^ { \mu } \right) \\ & = 0
                \end{aligned}
                \label{subeq:ptp}
        \end{equation}
        \begin{equation}
                \begin{aligned}
                        \hat { P } _ { R } \gamma ^ { \mu } \gamma ^ { 5 } \hat { P } _ { L }  & = \gamma ^ { \mu } \hat { P } _ { L }  \gamma ^ { 5 } \hat { P } _ { L } = - \gamma ^ { \mu } \left( \hat { P } _ { L } \right) ^ { 2 } \\ & = - \gamma ^ { \mu } \hat { P } _ { L }
                \end{aligned}
                \label{subeq:pap}
        \end{equation}
        \begin{equation}
                \hat { P } _ { R } \gamma ^ { 5 } \hat { P } _ { L } = - \hat { P } _ { R } \hat { P } _ { L } = 0
                \label{subeq:ppp}
        \end{equation}
\end{subequations}
con l'evidente risultato che solo quantità vettoriali e assiali sono rilevanti nella descrizione dell'interazione debole. Proseguendo i calcoli per (\ref{subeq:pvp}) e (\ref{subeq:pap}) si vede che esse forniscono lo stesso risultato a meno di un segno:
\begin{subequations}
        \begin{equation}
                \gamma ^ { \mu } \hat { P } _ { L }  = \frac { 1 } { 2 } \gamma ^ { \mu } \left( 1 - \gamma ^ { 5 } \right)
                \label{subeq:pvp_exp}
        \end{equation}
        \begin{equation}
                - \gamma ^ { \mu } \hat { P } _ { L } = - \frac { 1 } { 2 } \gamma ^ { \mu } \left( 1 - \gamma ^ { 5 } \right)
                \label{subeq:pap_exp}
        \end{equation}
\end{subequations}
Trascurando il fattore $1/2$ si può quindi scrivere l'operatore $V-A$ come:
\begin{equation}
        \hat { O } _ { i } ^ { \prime } = \gamma ^ { \mu } \left( 1 - \gamma ^ { 5 } \right) = \gamma ^ { \mu } - \gamma ^ { \mu } \gamma ^ { 5 }
        \label{eq:va_operator}
\end{equation}

\subsection{Tipi di transizione nel decadimento $\beta$ e forma della corrente adronica}
\label{subsec:fermi_gt}
L'operatore (\ref{eq:va_operator}) introdotto precedentemente, è in realtà in certi casi un'approssimazione; la sua forma più generale è infatti:
\begin{equation}
        \left( 1 - \gamma ^ { 5 } \right) \rightarrow \left( c _ { V } - c _ { A } \gamma ^ { 5 } \right)
\end{equation}
dove $c_{V}$ e $c_{A}$ sono costanti introdotte per tenere in considerazione l'effetto che l'attività interna degli adroni può avere sulla conservazione delle correnti vettoriale e assiale. Esse vanno determinate sperimentalmente: mentre nel caso di una corrente leptonica si ha $c_{V}=-c_{A}=1$ ed entrambe le componenti si conservano, per le correnti in cui sono coinvolti adroni la situazione richiede uno studio più approfondito. Per il decadimento beta, in particolare, i contributi vettoriale e assiale sono dati da tipi diversi di transizione.
\paragraph{Transizioni di Fermi e Gamow-Teller} Dato che nel decadimento beta nucleare si hanno neutrone e protone che si muovono non relativisticamente è facile convincersi (si veda \cite{ref:greiner}) che in questo limite le quantità introdotte in Tab. \ref{tab:bilinear} si comportano nella seguente maniera:
\begin{equation}
        S , V \rightarrow u_{A}^{p} ^ { \dagger } u_{A}^ { n  } \quad , \quad T , A \rightarrow u_{A} ^ { p } ^ { \dagger } \sigma^{\mu\nu} u_{A} ^ {n } \quad , \quad P \rightarrow 0
\end{equation}
dove $u_{A}^{p}$ e $u_{A}^{n}$ sono gli spinori a due componenti che dominano in limite non-relativistico. I casi rilevanti prendono il nome di transizioni di Fermi:
\begin{equation}
        S , V \rightarrow u_{A} ^ {p } ^ { \dagger } u_{A} ^ { n  }
        \label{eq:sv_app}
\end{equation}
e di Gamow-Teller:
\begin{equation}
        T , A \rightarrow u _{A}^ { p } ^ { \dagger } \sigma^{\mu\nu} u _{A} ^{ n  }
        \label{eq:ta_app}
\end{equation}
La differenza principale tra le due sta nella variazione di momento angolare totale tra nucleo iniziale e nucleo finale. Tale variazione è connessa con lo spin dei due leptoni $e ^ { - } , \overline { \nu } _ { e }$ (oppure $e^{+}, \nu_{e}$): entrambi hanno spin 1/2, quindi la variazione dello spin nucleare può essere nulla (spin di neutrino ed elettrone antiparalleli), oppure $\pm1$ (spin paralleli). Assumendo che l'elettrone e il neutrino siano emessi in uno stato di momento angolare $l=0$, la variazione dello spin del nucleo è pari alla somma degli spin dei due leptoni. Per l'orientazione degli spin di elettrone e neutrino si ha quindi che nelle transizioni di Fermi ($0 \rightarrow 0$) gli spin sono antiparalleli (stato di singoletto), mentre in quelle di Gamow-Teller ($0 \rightarrow 1$) gli spin sono paralleli (stato di tripletto); esiste inoltre il caso delle transizioni cosiddette miste, in cui si ha $\frac{1}{2} \rightarrow \frac{1}{2}$ e gli spin possono essere antiparalleli (lo spin del nucleo non cambia) o paralleli (lo spin del nucleo cambia direzione).\\
\begin{table}
        \centering
        \begin{tabular}{c  c  c}
                \hline
                decadimento & transizione $J^{P}$ & $G _ { F } ^ { 2 } | \mathcal { M } | ^ { 2 }$ ($MeV^{2}fm^{6}$) \\
                \hline
                $^ { 14 } \mathrm { O } \rightarrow ^ { 14 } \mathrm { N } ^ { * } + e ^ { + } + \nu _ { e }$ & $0 ^ { + } \rightarrow 0 ^ { + }$ & $1.52 \times 10 ^ { - 8 }$ \\
                $^{6}H e \rightarrow^ { 6 } L i + e ^ { - } + \overline { \nu }$ & $0^{+} \rightarrow 1^{+}$ & $7.45 \times 10 ^ { - 8 }$ \\
                $n \rightarrow p e ^ { - } \overline { \nu }$ & $\frac { 1 } { 2 }^{+} \rightarrow \frac { 1 } { 2 } ^ { + }$ & $4.25 \times 10 ^ { - 8 }$ \\
                \hline
        \end{tabular}
        \caption{Decadimenti $\beta$ suddivisi nelle diverse transizioni}
        \label{tab:beta_decays}
\end{table}
Si tiene conto di quanto sopra esposto nella costruzione delle correnti e, conseguentemente, nel calcolo di $|\mathcal{M}|^{2}$: come si vede in (\ref{eq:sv_app}) e (\ref{eq:ta_app}), le transizioni di Fermi sono di tipo vettoriale, mentre quelle di Gamow-Teller sono di tipo assiale; per le prime si scrive $| \mathcal { M } | ^ { 2 } \equiv \left| \mathcal { M } _ { F } \right| ^ { 2 } \propto c _ { V } ^ { 2 }$ mentre per le altre si ha $| \mathcal { M } | ^ { 2 } \equiv \left| \mathcal { M } _ { G T } \right| ^ { 2 } \propto c _ { A } ^ { 2 }$. Analizzando Tab. \ref{tab:beta_decays} si può vedere la dipendenza del prodotto $G_{F}^{2}|\mathcal{M}|^{2}$ dalla variazione dello spin nella transizione del nucleo.
Sfruttando i primi due processi riportati in Tab. \ref{tab:beta_decays} è possibile stimare \cite{ref:BGSex} il rapporto tra le costanti $c_{V}$ e $c_{A}$:
\begin{equation}
        \lambda = c _ { A } / c _ { V } = - 1.2695 \pm 0.0029
        \label{eq:ca_cv}
\end{equation}
\subsection{Esempi di decadimento: muone, neutrone e pione}
\label{subsec:decays}
Si applica ora la teoria $V-A$ presentata nella parte precedente a tre importanti esempi di decadimento: muone, neutrone e pione carico.
\paragraph{Decadimento del muone} Il muone decade secondo:
\begin{equation}
        \mu^{-} \rightarrow e^{-} \overline{\nu_{e}} \nu_{\mu}
        \label{eq:muon_decay}
\end{equation}
Il diagramma di Feynman del processo è mostrato in Fig. \ref{fig:muon_decay}.
\begin{figure}[!h]
        \centering
        \feynmandiagram [layered layout, horizontal=a to b] {
                a [particle=\(\mu^{-}\)] -- [fermion] b -- [fermion] f1 [particle=\(\nu_{\mu}\)],
                b -- [boson, edge label'=\(W^{-}\)] c,
                c -- [anti fermion] f2 [particle=\(\overline \nu_{e}\)],
                c -- [fermion] f3 [particle=\(e^{-}\)],
        };
        \caption{Diagramma di Feynman per il decadimento del muone.}
        \label{fig:muon_decay}
\end{figure}
Per l'interazione debole, le regole di Feynman (Appendice \ref{app:feynman_rules}) prevedono il fattore V-A
\begin{equation}
        \mathcal { K } ^ { \mu } \rightarrow - \frac { i g _ { w } } { 2 \sqrt { 2 } } \gamma ^ { \mu } \left( 1 - \gamma ^ { 5 } \right)
\end{equation}
dove $g_{w}$ è la costante di accoppiamento debole, ad entrambi i vertici. A causa della grande massa del bosone $W^{-}$, il propagatore può essere approssimato da
\begin{equation}
         \mathcal { D } _ { \mu \nu } \rightarrow i \frac { g _ { \mu \nu } } { M _ { W } ^ { 2 } }
        \label{eq:propagator}
\end{equation}
Proseguendo nell'applicazione delle regole di Feynman si ottiene l'uguaglianza:
\begin{equation}
        - i \mathcal { M } = \left[ \overline { u } \left( \nu _ { \mu } \right) \mathcal { K } ^ { \mu } u \left( \mu \right) \right] \cdot \mathcal { D } _ { \mu \nu } \cdot \left[ \overline { u } \left( e^{-} \right) \mathcal { K } ^ { \nu } v \left( \nu _ { e } \right) \right]
\end{equation}
dove $u$ e $v$ sono spinori di Dirac nello spazio dei momenti con $\overline { u } = u ^ { \dagger } \gamma ^ { 0 }$ e $\overline { v } = v ^ { \dagger } \gamma ^ { 0 }$. Inserendo la forma esplicita di fattori di vertice e propagatore si ha:
\begin{equation}
        \mathcal { M } = \frac { g _ { w } ^ { 2 } } { 8 M _ { W } ^ { 2 } } \left[ \overline { u } \left( \nu _ { \mu } \right) \gamma ^ { \mu } \left( 1 - \gamma ^ { 5 } \right) u ( \mu ) \right] \cdot \left[ \overline { u } \left( e ^ { - } \right) \gamma _ { \mu } \left( 1 - \gamma ^ { 5 } \right) v \left( \nu _ { e } \right) \right]
        \label{eq:amplitude_muon}
\end{equation}
Il calcolo di $| \mathcal { M } | ^ { 2 }$ è illustrato dettagliatamente in \cite{ref:hayes}. Qui si riporta il risultato finale:
\begin{equation}
         | \mathcal { M } | ^ { 2 }  = 2 \left[ \frac { g _ { w } } { M _ { W } } \right] ^ { 4 } \left( p _ { \mu } \cdot p _ { \nu _ { \mu } } \right) \left( p _ { \nu _ { e } } \cdot p _ { e } \right)
\end{equation}
Sostituendo questo valore nella regola d'oro di Fermi (\ref{eq:decay_rate}) si trova:
\begin{equation}
        \Gamma = \frac { m _ { \mu } ^ { 5 } c ^ { 4 } } { 192 \pi ^ { 3 } \hbar ^ { 7 } } G _ { F } ^ { 2 }
        \label{eq:muon_decay_rate}
\end{equation}
dove la costante di Fermi $G_{F}$ è definita da
\begin{equation}
        G _ { F } \equiv \frac { \sqrt { 2 } } { 8 } \left[ \frac { g _ { w } } { M _ { W } } \right] ^ { 2 } \cdot ( \hbar c ) ^ { 3 }
        \label{eq:fermi_constant}
\end{equation}
Ridefinendo poi
\begin{equation}
        G_{F} \leftrightarrow \frac{G_{F}}{\hbar^{3}c^{3}}
        \label{eq:fermi_constant_ridef}
\end{equation}
e sostituendo i valori sperimentali per la vita media del muone si ottiene:
\begin{equation}
        G^{\mu} _ { F } = 1.16637 \times 10 ^ { - 5 } \mathrm { GeV } ^ { - 2 }
        \label{eq:gf_muon_value}
\end{equation}
Questo valore è lo stesso riportato in \cite{ref:PDG}.
\paragraph{Decadimento del neutrone}
Il decadimento beta del neutrone è riportato in Eq. (\ref{eq:beta_neg}) mentre il diagramma di Feynman, realizzato considerando protone e neutrone in termini di quark costituenti, è rappresentato in Fig. \ref{fig:neutron_decay_quarks}.
\begin{figure}[!h]
        \centering
        \begin{tikzpicture}
                \begin{feynman}
                        \vertex (a) {\(d\)};
                        \vertex [right=of a] (b);
                        \vertex [above right=of b] (f1) {\(u\)};
                        \vertex [above= 0.4 cm of a](d) {\(d\)};
                        \vertex [right=of d] (e);
                        \vertex [above right=of e] (g1) {\(d\)};
                        \vertex [above= 0.4 cm of d](m) {\(u\)};
                        \vertex [right=of m] (n);
                        \vertex [above right=of n] (h1) {\(u\)};
                        \vertex [below right=of b] (c);
                        \vertex [above right=of c] (f2) {\(\overline \nu_{e}\)};
                        \vertex [below right=of c] (f3) {\(e^{-}\)};
                        \diagram* {
                                (a) -- [fermion] (b) -- [fermion] (f1),
                                (d) -- [fermion] (e) -- [fermion] (g1),
                                (m) -- [fermion] (n) -- [fermion] (h1),
                                (b) -- [boson, edge label'=\(W^{-}\)] (c),
                                (c) -- [anti fermion] (f2),
                                (c) -- [fermion] (f3),
                        };
                \end{feynman}
        \end{tikzpicture}
        \caption{Diagramma di Feynman per il decadimento del neutrone, assumendo protone e neutrone costituiti da quark.}
        \label{fig:neutron_decay_quarks}
\end{figure}
Il fattore al vertice quark-quark si può scrivere:
\begin{equation}
        \mathcal { K } ^ { \mu } \rightarrow - \frac { i g _ { w } } { 2 \sqrt { 2 } } \gamma ^ { \mu } \left( c _ { V } - c _ { A } \gamma ^ { 5 } \right)
\end{equation}
dove le costanti $c_{V}$ e $c_{A}$ sono le stesse introdotte in \ref{subsec:fermi_gt}. Se la parte vettoriale è conservata, allora $c_{V} = 1$; altrimenti il rapporto tra le due costanti assume il valore (\ref{eq:ca_cv}). Si può quindi scrivere:
\begin{equation}
        \mathcal { M } = \frac { g _ { w } ^ { 2 } } { 8 M _ { W } ^ { 2 } } \cdot \left[ \overline { u } ( p ) \gamma ^ { \mu } \left( c _ { V } - c _ { A } \gamma ^ { 5 } \right) u ( n ) \right] \cdot \left[ \overline { u } ( e ) \gamma _ { \mu } \left( 1 - \gamma ^ { 5 } \right) v \left( \nu _ { e } \right) \right]
        \label{eq:neutron_complex_matrix_element}
\end{equation}
Si noti che in questa espressione si possono chiaramente distinguere la corrente leptonica
\begin{equation}
        \overline { u } ( e ) \gamma _ { \mu } \left( 1 - \gamma ^ { 5 } \right) v \left( \nu _ { e } \right)
        \label{eq:leptonic_current}
\end{equation}
e quella barionica
\begin{equation}
        \overline { u } ( p ) \gamma ^ { \mu } \left( c _ { V } - c _ { A } \gamma ^ { 5 } \right) u ( n )
        \label{eq:barionic_current}
\end{equation}
Questo risultato è perfettamente in linea col fatto che l'elemento di matrice è proporzionale ad un prodotto di correnti, come riportato in Eq. (\ref{eq:weak_matrix_element}).
Il calcolo di (\ref{eq:neutron_complex_matrix_element}), i cui dettagli sono riportati in \cite{ref:hayes}, consente di ottenere un tasso di decadimento pari a:
\begin{equation}
        \Gamma = \frac { \rho_{f} m _ { e } ^ { 5 } c ^ { 4 } } { 64 \pi ^ { 3 } \hbar } \left[ \frac { g _ { w } } { M _ { W } } \right] ^ { 4 } c _ { V } ^ { 2 } \left( 1 + 3 \lambda ^ { 2 } \right)
\end{equation}
che mostra come l'effetto dell'aggiunta dei fattori $c_{V}$ e $c_{A}$ sia quello di introdurre un fattore
\begin{equation}
        \frac { 1 } { 4 } c _ { V } ^ { 2 } \left( 1 + 3 \lambda ^ { 2 } \right)
\end{equation}
nel valore del tasso di decadimento. Di nuovo, è possibile riscrivere questo valore in termini di $G_{F}$:
\begin{equation}
        \Gamma = \frac { \rho_{f} m _ { e } ^ { 5 } } { 2 \pi ^ { 3 } \hbar ^ { 7 } } G _ { F } ^ { 2 } c_{V}^{2} \left( 1 + 3 \lambda ^ { 2 } \right)
\end{equation}
Calcolando il valore dello spazio delle fasi $\rho_{f}$ e sostituendo a $\Gamma$ l'inverso della vita media del neutrone $\tau_{n}=885.7\; \mathrm{s}$ , si trova un valore di $G_{F}$ pari a:
\begin{equation}
        G^{n} _ { F } = 1.140 \times 10 ^ { - 5 } \mathrm { GeV } ^ { - 2 }
        \label{eq:gf_neutron_value}
\end{equation}
Questo valore discorda da quello ottenuto in Eq.~(\ref{eq:gf_muon_value}) da un decadimento che coinvolge solo leptoni (in particolare, è leggermente più piccolo). Sembrò quindi inizialmente che interazioni coinvolgenti leptoni e quark fossero diverse, facendo così cadere l'ipotesi di universalità dell'interazione debole.

\paragraph{Decadimento del pione carico}
\begin{figure}[!h]
        \centering
        \begin{tikzpicture}
                \begin{feynman}
                        \vertex (a);
                        \vertex [above left=of a](i2) {\(\overline u\)};
                        \vertex [below left=of a] (i1) {\(d\)};
                        \vertex [right= 3.0 cm of a] (b);
                        \vertex [above right=of b] (f1) {\(l^{-}\)};
                        \vertex [below right=of b] (f2) {\(\overline \nu_{l}\)};
                        \diagram* {
                                (i1) -- [fermion] (a) -- [fermion] (i2),
                                (a) -- [boson, edge label=\(W^{-}\)] (b),
                                (f1) -- [fermion] (b) -- [fermion] (f2),
                        };
                        \draw [decoration={brace}, decorate] (i1.south west) -- (i2.north west)
                        node [pos=0.5, left] {\(\pi^{-}\)};
                \end{feynman}
        \end{tikzpicture}
        \caption{Diagramma di Feynman per il decadimento del pione negativo, costituito dai quark $\overline{u}$ e $d$.}
        \label{fig:pion_decay}
\end{figure}
Un altro decadimento cruciale per provare la bontà della teoria è quello del pione carico, il cui confronto con le osservazioni sperimentali costituisce una conferma della teoria $V-A$. In Fig. \ref{fig:pion_decay} è riportato il decadimento del mesone $\pi^{-}$, dove i leptoni indicati con $l$ possono essere $\mu^{-}$ o $e^{-}$, e $\overline{\nu}_{l}$ i rispettivi antineutrini. Le possibilità saranno dunque:
\begin{subequations}
        \begin{equation}
                \pi^{-} \rightarrow e^{-} \overline{\nu}_{e}
                \label{eq:pi_e}
        \end{equation}
        \begin{equation}
                \pi^{-} \rightarrow \mu^{-} \overline{\nu}_{\mu}
                \label{eq:pi_mu}
        \end{equation}
\end{subequations}
L'elemento di matrice si può scrivere in maniera generale come:
\begin{equation}
        \mathcal { M } = \frac { g _ { w } ^ { 2 } } { 8 \left( M _ { W C } c \right) ^ { 2 } } \left[ \overline { u } ( \nu_{l} ) \gamma^{ \mu } \left( 1 - \gamma ^ { 5 } \right) v ( l ) \right] F ^ { \mu }
        \label{eq:matrix_element_pion_decay}
\end{equation}
dove $F^{\mu}$ è dato da una una quantità scalare che moltiplica il momento del pione $p^{\mu}$:
\begin{equation}
        F ^ { \mu } = f _ { \pi } p ^ { \mu }
\end{equation}
Il calcolo di $|\mathcal{M}|^{2}$, per il quale si rimanda a \cite{ref:griff}, porta:
\begin{equation}
         | \mathcal { M } | ^ { 2 }  = \left( \frac { g _ { w } } { 2 M _ { W } } \right) ^ { 4 } f _ { \pi } ^ { 2 } m _ { l } ^ { 2 } \left( m _ { \pi } ^ { 2 } - m _ { l } ^ { 2 } \right)
\end{equation}
che, sostituito in (\ref{eq:decay_rate}), fa sì che il tasso di decadimento sia:
\begin{equation}
        \Gamma = \frac { f _ { \pi } ^ { 2 } } { \pi \hbar m _ { \pi } ^ { 3 } } \left( \frac { g _ { w } } { 4 M _ { w } } \right) ^ { 4 } m _ { l } ^ { 2 } \left( m _ { \pi } ^ { 2 } - m _ { l } ^ { 2 } \right) ^ { 2 }
        \label{eq:decay_rate_pion}
\end{equation}
Calcolando questa grandezza per entrambi i leptoni, si ha il seguente \textit{branching ratio}:
\begin{equation}
        R = \frac { \Gamma _ { \pi \rightarrow e } } { \Gamma _ { \pi \rightarrow \mu } } = \frac { m _ { e } ^ { 2 } \left( m_{\pi}^{2} - m_{e}^2  \right) ^ { 2 } } { m _ { \mu } ^ { 2 } \left( m_{\pi}^2 - m_{\mu}^2 \right) ^ { 2 } } \simeq  1.27 \cdot 10 ^ { - 4 }
        \label{eq:br_ratio}
\end{equation}
in ottimo accordo con i dati sperimentali.
\subsection{Decadimenti deboli di particelle strane}
\label{subsec:strange}
Un altro fatto che richiese un ampliamento della teoria formulata da Fermi fu l'osservazione di anomalie nei decadimenti che coinvolgono particelle strane, in particolare nella differenza di comportamento tra processi con e senza variazione di stranezza ($\Delta S = 1$ e $\Delta S = 0$ rispettivamente).
Si consideri il caso dei decadimenti riportati in Tab.~\ref{tab:leptonic_decays}, i cui diagrammi di Feynmann sono riportati rispettivamente in Fig. \ref{fig:pion_decay} e \ref{fig:kaon_decay}.
\begin{table}[!h]
        \begin{tabular}{llccc}
                \hline
                Decadimento & & $\Delta S$ & \tau\: (s)& $BR = \Gamma_{i}/\Gamma$    \\
                \hline
                $\pi^{-} \rightarrow \mu^{-} \overline{\nu_{\mu}}$ & $\overline{u}d \rightarrow W^{-} \rightarrow \mu^{-} \overline{\nu_{\mu}}$ & 0 & $2.6 \times 10^{-8}$ & $100\%$ \\
                $K^{-} \rightarrow \mu^{-} \overline{\nu_{\mu}}$ & $\overline{u}s \rightarrow W^{-} \rightarrow \mu^{-} \overline{\nu_{\mu}}$ & 1 & $1.27 \times 10^{-8}$ & $63.5\%$ \\
                \hline
        \end{tabular}
        \caption{Decadimenti leptonici di pione e kaone.}
        \label{tab:leptonic_decays}
\end{table}
\begin{figure}[!h]
        \centering
        \begin{tikzpicture}
                \begin{feynman}
                        \vertex (a);
                        \vertex [above left=of a](i2) {\(\overline u\)};
                        \vertex [below left=of a] (i1) {\(s\)};
                        \vertex [right= 3.0 cm of a] (b);
                        \vertex [above right=of b] (f1) {\(\mu^{-}\)};
                        \vertex [below right=of b] (f2) {\(\overline \nu_{\mu}\)};
                        \diagram* {
                                (i1) -- [fermion] (a) -- [fermion] (i2),
                                (a) -- [boson, edge label=\(W^{-}\)] (b),
                                (f1) -- [fermion] (b) -- [fermion] (f2),
                        };
                        \draw [decoration={brace}, decorate] (i1.south west) -- (i2.north west)
                        node [pos=0.5, left] {\(K^{-}\)};
                \end{feynman}
        \end{tikzpicture}
        \caption{Diagramma di Feynman per il decadimento del kaone negativo.}
        \label{fig:kaon_decay}
\end{figure}
Utilizzando la regola d'oro di Fermi e la costante $G_{F}$ per il calcolo della vita media del kaone, si ottiene una vita media circa venti volte più piccola di quella misurata sperimentalmente. Per spiegare questa anomalia si può immaginare che, nel caso dei quark, la costante di accoppiamento dipenda dal sapore: nel caso del $\pi$ il decadimento coinvolge il quark $d$, e si può immaginare di utilizzare una costante di accoppiamento $G _ { d } \simeq G _ { F }$, in quanto la vita media calcolata con $G_{F}$ fornisce risultati adeguati; nel caso del $K$ il decadimento coinvolge il quark $s$, e si può immaginare di utilizzare una costante di accoppiamento $G _ { s } < G _ { d }$ in maniera da ottenere il risultato della vita media corretto. Così facendo, si ottiene:
\begin{equation}
        \frac { G _ { s } ^ { 2 } } { G _ { d } ^ { 2 } } \simeq 0.05
        \label{eq:frac_lepton_decay}
\end{equation}

La stessa anomalia si può notare considerando i decadimenti in Tab.~\ref{tab:isemileptonic_decays}, i cui diagrammi di Feynmann sono riportati in Fig. \ref{fig:sigma_decay}. Utilizzando la regola di Sargent
\begin{equation}
        \frac { \left( \Gamma _ { i } / \Gamma \right) } { \tau } \simeq G _ { F } ^ { 2 } E _ { 0 } ^ { 5 } \simeq G _ { F } ^ { 2 } \Delta m ^ { 5 }
        \label{eq:sargent}
\end{equation}
per il calcolo della vita media, si può vedere come anche in questo caso i risultati numerici siano corretti per i decadimenti con $\Delta S = 0$, ed errati di un fattore \sim 20 per i decadimenti con $\Delta S = 1$ . Di nuovo, si può immaginare che nel caso di decadimenti senza variazione di stranezza intervenga la costante $G_{d}$, mentre in quelli con $\Delta S = 1$ intervenga la costante $G_{s}$. Si può stimare il rapporto tra le due costanti usando i decadimenti della $\Sigma^{-}$:
\begin{equation}
        \frac{G_{s}^{2}}{G_{d}^{2}} = \frac{\Gamma(\Sigma^{-} \rightarrow n e^{-} \overline{\nu_{e}})/\Delta m^{5}_{\Delta S = 1}}{\Gamma(\Sigma^{-} \rightarrow \Lambda^{0} e^{-} \overline{\nu_{e}})/\Delta m^{5}_{\Delta S = 0}} = 0.057
        \label{eq:ratio_semileptonic}
\end{equation}
\begin{table}[!h]
        \begin{tabular}{llccc}
                \hline
                Decadimento & \Delta m\: (MeV) & $\Delta S$ & \tau\: (s)& $BR = \Gamma_{i}/\Gamma$    \\
                \hline
                $\Sigma^{-} \rightarrow \Lambda^{0} e^{-} \overline{\nu_{e}}$ & 81.7 & 0 & $1.48 \times 10^{-10}$ & $0.57 \times 10^{-4}$ \\
                $\Sigma^{-} \rightarrow n e^{-} \overline{\nu_{e}}$ & 257.8 & 1 & $1.48 \times 10^{-10}$ & $1.02 \times 10^{-3}$ \\
                \hline
        \end{tabular}
        \caption{Decadimenti semi-leptonici della $\Sigma^{-}$.}
        \label{tab:isemileptonic_decays}
\end{table}
Si conferma quindi che il rapporto tra $G _ { s } ^ { 2 }$ e $G _ { d } ^ { 2 }$ vale come indicato in (\ref{eq:frac_lepton_decay}), indipendentemente dal tipo di transizioni tra adroni: ciò deve quindi riflettere una proprietà dei quark costituenti.
\begin{figure}[!h]
        \centering
        \subfloat{
        \begin{tikzpicture}
                \begin{feynman}
                        \vertex (a1) {\(d\)};
                        \vertex[right=4.0 cm of a1] (a2) {\(d\)};
                        \vertex[below=2em of a1] (b1) {\(d\)};
                        \vertex[below=2em of a2] (b2) {\(d\)};
                        \vertex[below=2em of b1] (c1) {\(s\)};
                        \vertex[right=1.5 cm of c1] (w1);
                        \vertex[below=2em of b2] (c2) {\(u\)};
                        \vertex[below=1.0 cm of c2] (d1) {\(e^{-}\)};
                        \vertex[below=2em of d1] (d2) {\(\overline \nu_{e}\)};
                        \vertex at ($(d1)!0.5!(d2) - (1cm, 0)$) (w2);
                        \diagram* {
                                (a1) -- [fermion] (a2),
                                (b1) -- [fermion] (b2),
                                (c1) -- [fermion] (c2),
                                (d2) -- [fermion] (w2) -- [fermion] (d1),
                                (w1) -- [boson, edge label=\(W^{-}\)] (w2),
                        };
                        \draw [decoration={brace}, decorate] (c1.south west) -- (a1.north west)
                                node [pos=0.5, left] {\(\Sigma^{-}\)};
                        \draw [decoration={brace}, decorate] (a2.north east) -- (c2.south east)
                                node [pos=0.5, right] {\(n\)};
                \end{feynman}
        \end{tikzpicture}
        }
        \qquad
        \qquad
        \subfloat{
        \begin{tikzpicture}
                \begin{feynman}
                        \vertex (a1) {\(s\)};
                        \vertex[right=4.0 cm of a1] (a2) {\(s\)};
                        \vertex[below=2em of a1] (b1) {\(d\)};
                        \vertex[below=2em of a2] (b2) {\(d\)};
                        \vertex[below=2em of b1] (c1) {\(d\)};
                        \vertex[right=1.5 cm of c1] (w1);
                        \vertex[below=2em of b2] (c2) {\(u\)};
                        \vertex[below=1.0 cm of c2] (d1) {\(e^{-}\)};
                        \vertex[below=2em of d1] (d2) {\(\overline \nu_{e}\)};
                        \vertex at ($(d1)!0.5!(d2) - (1cm, 0)$) (w2);
                        \diagram* {
                                (a1) -- [fermion] (a2),
                                (b1) -- [fermion] (b2),
                                (c1) -- [fermion] (c2),
                                (d2) -- [fermion] (w2) -- [fermion] (d1),
                                (w1) -- [boson, edge label=\(W^{-}\)] (w2),
                        };
                        \draw [decoration={brace}, decorate] (c1.south west) -- (a1.north west)
                                node [pos=0.5, left] {\(\Sigma^{-}\)};
                        \draw [decoration={brace}, decorate] (a2.north east) -- (c2.south east)
                                node [pos=0.5, right] {\(\Lambda^{0}\)};
                \end{feynman}
        \end{tikzpicture}
        } \\
        \caption{Decadimenti di $\Sigma^{-}$ con $\Delta S = 1$ (sinistra) e $\Delta S = 0$ (destra).}
        \label{fig:sigma_decay}
\end{figure}

\subsection{L'angolo di Cabibbo}
\label{subsec:cabibbo_angle}
I fatti sperimentali sopracitati vennero interpretati da Nicola Cabibbo nel 1964. Egli mostrò che sia i leptoni che i quark sono autostati dell'interazione debole, con le seguenti assunzioni:
\begin{itemize}
        \item l'accoppiamento degli elettroni al campo debole è proporzionale ad una carica debole $g_{e\nu}$;
        \item l'accoppiamento dei muoni è proporzionale a $g_{\mu\nu}$, con quest'ultima identica a $g_{e\nu}$;
        \item l'accoppiamento dei quark $(u, d)$ genera le transizioni con $\Delta S = 0$ ed è proporzionale a $g_{ud}$;
        \item l'accoppiamento dei quark $(u, s)$ genera le transizioni con $\Delta S = 1$ ed è proporzionale a $g_{us}$;
\end{itemize}
In ogni vertice di un diagramma di Feynman occorre inserire la costante corrispondente, con il calcolo degli elementi di matrice che viene di conseguenza:
\begin{itemize}
        \item per i processi puramente leptonici si ha
                \begin{equation}
                        \left\langle f \left| H _ { W } \right| i \right\rangle \propto g _ { e v } ^ { 2 } = G _ { F }
                        \label{eq:genu}
                \end{equation}
        \item per i processi semi-leptonici con $\Delta S = 0$ si ha
                \begin{equation}
                \left\langle f \left| H _ { W } \right| i \right\rangle _ { \Delta S = 0 } \propto g _ { e v } g _ { u d } = G _ { d }
                \label{eq:gd}
        \end{equation}
\item per i processi semi-leptonici con $\Delta S = 1$ si ha
        \begin{equation}
                \left\langle f \left| H _ { W } \right| i \right\rangle _ { \Delta S = 1 } \propto g _ { e v } g _ { u s } = G _ { s }
                \label{eq:gs}
        \end{equation}
\end{itemize}
L'ipotesi di Cabibbo è che l'interazione debole dipenda da un solo parametro, la costante di Fermi $G_{F}$ (come già immaginato da Fermi precedentemente). Questa descrive l'accoppiamento del campo debole sia verso i leptoni che verso i quark tramite la relazione:
\begin{equation}
        G _ { F } = g _ { e v } ^ { 2 } = g _ { u d } ^ { 2 } + g _ { u s } ^ { 2 } \longrightarrow g _ { u d } = g _ { e v } \cos \theta _ { C } ; \quad g _ { u s } = g _ { e v } \sin \theta _ { C }
\end{equation}
In questo modello, quanto sopra esposto corrisponde al fatto che i quark che partecipano all'interazione debole non sono gli autostati di sapore che caratterizzano l'interazione forte, ma una loro combinazione lineare, che può considerarsi ruotata di un angolo $\theta_{c}$ rispetto ai quark ordinari. Detti i nuovi autostati dell'interazione debole ($u_{c}, d_{c}, s_{c}$) si hanno così i seguenti doppietti deboli:
\begin{equation}
        \left( \begin{array} { c } { v _ { e } } \\ { e ^ { - } } \end{array} \right) , \left( \begin{array} { c } { v _ { \mu } } \\ { \mu ^ { - } } \end{array} \right) , \left( \begin{array} { c } { u } \\ { d _ { c } } \end{array} \right) = \left( \begin{array} { c } { u } \\ { d \cos \theta _ { c } + s \sin \theta _ { c } } \end{array} \right)
        \label{eq:weak_doublets}
\end{equation}
dove $\theta_{c}$ è l'angolo di Cabibbo e si trova sperimentalmente essere $\theta _ { c } = 0.235\: \mathrm { rad }$. La scelta di $u$ come autostato non mescolato è semplicemente una convenzione. Per ogni doppietto di leptoni l'accoppiamento debole è specificato da $G_{F}$. L'apparente differenza tra i valori della costante di accoppiamento è dovuta al processo di miscelamento dei quark. Una rappresentazione grafica dell'autostato debole $d_{c}$ che costituisce un doppietto con $u$ è visibile in Fig. \ref{fig:dmixing}.\\
\begin{figure}[t]
        \centering
        \begin{tikzpicture}[>=latex, font=\scriptsize]
                \draw[->] (0,0) -- (0,4.5);
                \draw[->] (0,0) -- (4.5,0);
                \draw[->] (0,0) -- (4.5,0.8);
                \draw (0,0) ++(0:30mm) arc (0:9.9:30mm);
                \node at (4.9,0) {$d$};
                \node at (0,4.9) {$s$};
                \node at (4.9,0.8) {$d_{c}$};
                \node at (3.5,0.3) {$\theta_{c}$};
        \end{tikzpicture}
        \caption{Autostato dell'interazione debole $d_{c}$.}
        \label{fig:dmixing}
\end{figure}

Si possono in questo modo interpretare i fatti sperimentali esposti in precedenza: dato che $\sin\theta_{c}=0.235$ e $\cos\theta_{c}=0.972$, transizioni con $\Delta S = 0$ hanno una costante di accoppiamento effettiva maggiore di transizioni con $\Delta S = 1$. Si spiega inoltre perché $G_{F}^{n}$ (Eq.~(\ref{eq:gf_neutron_value})) è più piccola di $G_{F}^{\mu}$ (Eq.~(\ref{eq:gf_muon_value})): in realtà, quello che si misura nel decadimento del neutrone è $G_{F}\cos \theta_{c}$. \\
Il valore dell'angolo di Cabibbo può essere ricavato confrontando decadimenti semi-leptonici analoghi con $\Delta S = 1$ e $\Delta S = 0$. Utilizzando, ad esempio, i decadimenti visti in Tab. \ref{tab:leptonic_decays} si ottiene il seguente rapporto tra le frazioni di decadimento:
\begin{equation}
        \frac { \Gamma \left( K ^ { - } \rightarrow \mu ^ { - } \overline\nu _ { \mu } \right) } { \Gamma \left( \pi ^ { - } \rightarrow \mu ^ { - } \overline \nu _ { \mu } \right) } = \frac { m _ { K } ^ { 2 } \left[ 1 - \left( m _ { \mu } ^ { 2 } / m _ { K } ^ { 2 } \right) \right] ^ { 2 } } { m _ { \pi } ^ { 2 } \left[ 1 - \left( m _ { \mu } ^ { 2 } / m _ { \pi } ^ { 2 } \right) \right] ^ { 2 } } \tan ^ { 2 } \theta _ { c }
\end{equation}
Inserendo il valore misurato delle masse delle particelle coinvolte si determina $\theta_{c} = (0.235 \pm 0.006)$.

\section{L'estensione della teoria: correnti neutre, meccanismo GIM e matrice CKM}
\label{sec:part_two}
\subsection{Interazione debole a corrente neutra}
\label{subsec:nc}
Come anticipato nell'introduzione, i processi deboli a corrente carica (CC) non sono gli unici esistenti. Ci sono infatti reazioni, come ad esempio
\begin{equation}
        \begin{array} { c } { \nu _ { \mu } e ^ { - } \rightarrow \nu _ { \mu } e ^ { - } } \\ { \overline { \nu } _ { \mu } e ^ { - } \rightarrow \overline { \nu } _ { \mu } e ^ { - } } \end{array}
        \label{eq:neutral_current}
\end{equation}
che possono avvenire solo attraverso lo scambio di un bosone $Z^{0}$. Questi processi vengono chiamati interazioni deboli a corrente neutra (NC). \\
Come il fotone, anche lo $Z^{0}$ può essere scambiato sia nel canale $t$ che nel canale $s$ (Fig. \ref{fig:neutral_currents}): nel primo caso si ha lo scambio del bosone tra un neutrino e un fermione, con questi due che rimangono gli stessi durante tutto il processo, mentre nel secondo caso si ha un processo di annichilazione $f \overline { f } \rightarrow Z ^ { 0 }$ seguito dalla creazione di una coppia $f ^ { \prime } \overline { f ^ { \prime } }$. \\
L'interazione a corrente neutra fu scoperta nel 1977 utilizzando una camera a bolle a liquido pesante (Gargamelle, CERN) esposta ad un fascio di neutrini muonici ad alta energia; furono osservate le reazioni sugli elettroni atomici $\nu _ { \mu } e ^ { - } \rightarrow \nu _ { \mu } e ^ { - }$ e su nucleoni $\nu _ { \mu } N \rightarrow \nu _ { \mu } + \text {adroni}$, in cui sono visibili nello stato finale le sole tracce delle particelle cariche. In entrambi i casi, visto che mancava la segnatura caratteristica della presenza di un muone, si concluse che le reazioni possono procedere solo via NC. \\
Storicamente, le NC sono state introdotte per rimuovere divergenze: nel caso, per esempio, della reazione $\nu \overline { \nu } \rightarrow W ^ { + } W ^ { - }$, il diagramma all'ordine più basso contiene lo scambio di un elettrone e dà luogo ad una divergenza, cancellata dal diagramma contenente lo scambio di una $Z^{0}$. \\
Si noti che le interazioni a corrente neutra non erano previste dalla teoria di Fermi e rappresentano uno dei motivi che ne richiesero l'estensione.
\begin{figure}[!h]
        \centering
        \subfloat{
                \feynmandiagram [vertical=a to b] {
                        i1 [particle=\(e^{-}\)] -- [fermion] a -- [fermion] i2 [particle=\(e^{-}\)],
                        a -- [boson, edge label=\(Z^{0}\)] b,
                        f1 [particle=\(\nu_{e}\)] -- [fermion] b -- [fermion] f2 [particle=\(\nu_{e}\)],
                };
        }
        \qquad
        \qquad
        \subfloat{
                \feynmandiagram [horizontal=a to b] {
                        i1 [particle=\(e^{-}\)] -- [fermion] a -- [fermion] i2 [particle=\(e^{+}\)],
                        a -- [boson, edge label=\(Z^{0}\)] b,
                        f1 [particle=\(\overline{q}\)] -- [fermion] b -- [fermion] f2 [particle=\(q\)],
                };
        }
        \caption{Esempi di processi a corrente neutra (NC): a sinistra il bosone è scambiato nel canale $t$, mentre a destra nel canale $s$.}
        \label{fig:neutral_currents}
\end{figure}
\subsection{Quark \textit{charm} e meccanismo GIM}
\label{subsec:gim}
Nel 1955 Gell-Mann e Pais osservarono che è possibile formare due combinazioni lineari dei mesoni $K$ neutri che sono autostati della simmetria CP, e quindi dell'interazione debole, e che questi corrispondono alle particelle che decadono nei due diversi stati di CP. Scegliendo le seguenti combinazioni lineari:
\begin{equation}
        | K _ { 1 } ^ { 0 } \rangle = 1 / \sqrt { 2 } ( | K ^ { 0 } \rangle + | \overline { K } ^ { 0 } \rangle ) \quad ; \quad | K _ { 2 } ^ { 0 } \rangle = 1 / \sqrt { 2 } ( | K ^ { 0 } \rangle - | \overline { K } ^ { 0 } \rangle )
\end{equation}
si può vedere come questi siano due stati distinti, combinazioni degli autostati delle interazioni forti, che hanno masse diverse e decadono in modi diversi:
\begin{equation}
        \begin{array} { c c } { | K _ { 1 } ^ { 0 } \rangle } & { \frac { 1 } { \sqrt { 2 } } ( | d \overline { s } \rangle + | s \overline { d } \rangle ) \quad C P = + 1 \longrightarrow \pi \pi } \\ { | K _ { 2 } ^ { 0 } \rangle } & { \frac { 1 } { \sqrt { 2 } } ( | d \overline { s } \rangle - | s \overline { d } \rangle ) \quad C P = - 1 \longrightarrow \pi \pi \pi } \end{array}
\end{equation}
Il valore della differenza di massa tra $K^{0}_{1}$ e $K^{0}_{2}$ si può calcolare nel modello a quark ed è in particolare proporzionale all'elemento di matrice che descrive la probabilità di transizione $K ^ { 0 } \leftrightarrow \overline { K } ^ { 0 }$, che ha $\Delta S = 2$. Il calcolo di questa quantità, considerando il contributo dei soli quark $u, d, s$ fornisce tuttavia un valore molto più grande del risultato sperimentale. Per questo motivo si ipotizzò l'esistenza di qualche fenomeno che impedisce le transizioni in cui cambia il sapore dei quark ma non la carica elettrica. \\
Questo fatto fu messo in luce nel 1970 da Glashow, Iliopulos e Maiani, i quali proposero l'esistenza di un quarto quark (denominato \textit{c, charm}) con le seguenti proprietà:
\begin{enumerate}
        \item carica elettrica +2/3, isospin $I=0$, stranezza $S=0$ e numero barionico $B=1/3$;
        \item un nuovo numero quantico $C$ che, analogamente alla stranezza, si conserva nell'interazione forte ma non in quella debole;
        \item è autostato dell'interazione debole e forma un secondo doppietto di quark con il secondo stato ruotato della teoria di Cabibbo $s_{ c } = s \operatorname { c o s } \theta _ { c } - d \operatorname { s i n } \theta _ { c }$.
\end{enumerate}
Con questa nuova ipotesi, si ha un nuovo doppietto per l'interazione debole:
\begin{equation}
        \left( \begin{array} { l } { c } \\ { s _ { c } } \end{array} \right) = \left( \begin{array} { c } { c } \\ { s \cos \theta _ { c } - d \sin \theta _ { c } } \end{array} \right)
        \label{eq:c_sc_doublet}
\end{equation}
Utilizzando le stessi notazioni usate in (\ref{eq:genu}), (\ref{eq:gd}) e (\ref{eq:gs}) è possibile descrivere i processi semi-leptonici in cui interviene il quark $c$ usando l'ipotesi del meccanismo GIM:
\begin{equation}
        \langle ( \nu _ { e } e ^ { + } ) s | H _ { W } | c \rangle \propto g _ { e \nu } g _ { c s } = G _ { F } \operatorname { c o s } \theta _ { c }
        \label{eq:gcs}
\end{equation}
\begin{equation}
        \left\langle \left( \nu _ { e } e ^ { + } \right) d \left| H _ { W } \right| c \right\rangle \propto g _ { e \nu } g _ { c d } = G _ { F } \sin \theta _ { c }
        \label{eq:gcd}
\end{equation}
da cui segue che nei decadimenti di mesoni charmati con $\Delta C = 1$ in mesoni non charmati, le transizioni $c \rightarrow s$ (che hanno accoppiamento $\cos^{2}\theta_{c}$) dominano sulle transizioni $c \rightarrow d$ (accoppiamento $\sin^{2}\theta_{c}$). \\
Riassumendo, gli stati ruotati autostati dell'interazione debole che compaiono nelle due famiglie di quark
\begin{equation}
        \left( \begin{array} { l } { u } \\ { d_{c} } \end{array} \right)  \quad \left( \begin{array} { l } { c } \\ { s_{c} } \end{array} \right)
\end{equation}
si possono scrivere in forma matriciale
\begin{equation}
        \left( \begin{array} { l } { d_{c} } \\ { s_{c} \end{array} \right) = \left( \begin{array} { c c } { \cos \theta _ {c} } & { \sin \theta _ { c } } \\ { - \sin  \theta _ { c } } & { \cos \theta _ { c } } \end{array} \right) \left( \begin{array} { l } { d } \\ { s } \end{array} \right)
\end{equation}
\begin{figure}[t]
        \centering
        \begin{tikzpicture}[>=latex, font=\scriptsize]
                \draw[->] (0,0) -- (0,4.5);
                \draw[->] (0,0) -- (4.5,0);
                \draw[->] (0,0) -- (-0.8,4.5);
                \draw[->] (0,0) -- (4.5,0.8);
                \draw (0,0) ++(0:30mm) arc (0:9.9:30mm);
                \draw (0,30mm) arc (90:99.9:30mm);
                \node at (4.9,0) {$d$};
                \node at (0,4.9) {$s$};
                \node at (4.9,0.8) {$d_{c}$};
                \node at (-0.8,4.9) {$s_{c}$};
                \node at (3.5,0.3) {$\theta_{c}$};
                \node at (-0.3,3.5) {$\theta_{c}$};
        \end{tikzpicture}
        \caption{Autostati dell'interazione debole $d_{c}$ e $s_{c}$.}
        \label{fig:sdmixing}
\end{figure}
Una rappresentazione grafica dei due autostati dell'interazione debole $d_{c}$ ed $s_{c}$ è riportata in Fig. \ref{fig:sdmixing}. Grazie a questa scrittura, il termine dell'hamiltoniana $H_{W}$ che dipende dal tipo di quark coinvolto può essere formalmente scritto nel modo seguente (considerando come esempio transizioni da un quark iniziale $d_{c}$ o $s_{c}$ ad uno stato finale $\overline{u}$ o $\overline{c}$):
\begin{equation}
        ( \overline { u } , \overline { c } ) \left( \begin{array} { c } { d _ { c } } \\ { s _ { c } } \end{array} \right) = ( \overline { u } , \overline { c } ) \left( \begin{array} { c } { \operatorname { c o s } \theta _ { c } \operatorname { s i n } \theta _ { c } } \\ { - \operatorname { s i n } \theta _ { c } \operatorname { c o s } \theta _ { c } } \end{array} \right) \left( \begin{array} { l } { d } \\ { s } \end{array} \right) = ( \overline { u } , \overline { c } ) V _ { c } \left( \begin{array} { l } { d } \\ { s } \end{array} \right)
\end{equation}

\begin{figure}[!h]
        \centering
        \feynmandiagram [layered layout, horizontal=a to b] {
                a [particle=\(c\)] -- [fermion] b [label=0:\(\;\;g_{cd} (g_{cs})\)]-- [fermion] f1 [particle=\(d (s)\)],
                b -- [boson, edge label'=\(W^{+}\)] c,
                c [label=0:\(\;\;g_{e\nu}\)] -- [anti fermion] f2 [particle=\(\nu_{e}\)],
                c -- [fermion] f3 [particle=\(e^{+}\)],
        };
        \caption{Interazione debole con un quark $c$ che si trasforma in $d$ o $s$.}
        \label{fig:gim_decay}
\end{figure}
\subsection{Indizi sul quarto quark dalle correnti neutre}
\label{subsec:nc_charm}
Un quarto quark era necessario anche per spiegare alcune anomalie connesse con le correnti neutre: i processi a corrente neutra sperimentalmente osservati sono caratterizzati dalla regola di selezione $\Delta S = 0$, mentre quelli con $\Delta S = 1$ non sono osservati. Più in generale, si dice che sono sfavorite le correnti neutre con violazione di \textit{flavour} (FCNC). Tuttavia, il meccanismo che ruota lo stato dei quark ($d, s$), postulando l'esistenza dei soli tre quark $u, d, s$, produce un termine con $\Delta S = 1$ quando considera NC:
\begin{equation}
        \begin{array} { c } J _ { \mu } = \overline { u }  \gamma _ { \mu } \left( 1 + \gamma _ { 5 } \right) d_{c} \approx { \left( \overline { u } , \overline { d } _ { c } \right) \left( \begin{array} { c } { u } \\ { d _ { c } } \end{array} \right) = \left( \overline { u } , \overline { d } \cos \theta _ { c } + \overline { s } \sin \theta _ { c } \right) \left( \begin{array} { c } { u } \\ { d \cos \theta _ { c } + s \sin \theta _ { c } } \end{array} \right) = } \\ { = \underbrace { u \overline { u } + \left( d \overline { d } \cos ^ { 2 } \theta _ { c } + s \overline { s } \sin ^ { 2 } \theta _ { c } \right) } _ { \Delta S = 0 } + \underbrace { ( s \overline { d } + \overline { s } d ) \sin \theta _ { c } \cos \theta _ { c } } _ { \Delta S = 1 } } \end{array}
        \label{eq:gim_before}
\end{equation}
\begin{figure}[!h]
        \centering
        \begin{tikzpicture}
                \begin{feynman}
                        \vertex (a);
                        \vertex [above left=of a](i2) {\(\overline u\)};
                        \vertex [below left=of a] (i1) {\(u\)};
                        \vertex [right= 2.0 cm of a] (b);
                        \vertex [right= 6.0 cm of a](c);
                        \vertex [above left=of c](j2) {\(\overline { d } _{ c }\)};
                        \vertex [below left=of c] (j1) {\(d_{c}\)};
                        \vertex [right= 2.0 cm of c] (d);
                        \diagram* {
                                (i1) -- [fermion] (a) -- [fermion] (i2),
                                (a) -- [boson, edge label=\(Z^{0}\)] (b),
                                (j1) -- [fermion] (c) -- [fermion] (j2),
                                (c) -- [boson, edge label=\(Z^{0}\)] (d),
                        };
                \end{feynman}
        \end{tikzpicture}
        \caption{Diagrammi di Feynman degli elementi che costituiscono la corrente calcolata in Eq. (\ref{eq:gim_before}).}
        \label{fig:gim_before}
\end{figure}
Ciascun termine $u \overline { u } , d \overline { d } , s \overline { s }$ rappresenta la densità di probabilità per la transizione di uno stesso sapore di quark da uno stato iniziale a quello finale, con emissione di una $Z^{0}$. Sono inoltre previste le transizioni mai osservate $s\overline{d}$ e $d\overline{s}$. I diagrammi di Feynman delle correnti che compaiono in (\ref{eq:gim_before}) sono rappresentati in Fig. \ref{fig:gim_before}.
Con il nuovo doppietto (\ref{eq:c_sc_doublet}), la corrente neutra può riscriversi nella forma:
\begin{equation}
        \begin{array} { l } J _ { \mu } = \overline { u }  \gamma _ { \mu } \left( 1 + \gamma _ { 5 } \right) d _ { c } + \overline { c } \gamma _ { \mu } \left( 1 + \gamma _ { 5 } \right) s _ { c } \approx { \left( \overline { u } , \overline { d } _ { c } \right) \left( \begin{array} { c } { u } \\ { d _ { c } } \end{array} \right) + \left( \overline { c } , \overline { s } _ { c } \right) \left( \begin{array} { c } { c } \\ { s _ { c } } \end{array} \right) = } \\ { = \underbrace { u \overline { u } + c \overline { c } + ( d \overline { d } + s \overline { s } ) \cos ^ { 2 } \theta _ { c } + ( s \overline { s } + d \overline { d } ) \sin ^ { 2 } \theta _ { c } } _ { \Delta S = 0 } + \underbrace { ( s \overline { d } + \overline { s } d - \overline { s } d - s \overline { d } ) \sin \theta _ { c } \cos \theta _ { c } } _ { \Delta S = 1 } } \end{array}
        \label{eq:gim_after}
\end{equation}
e i termini che prevedono transizioni con $\Delta S = 1$ sono automaticamente cancellati.
I diagrammi di Feynman delle correnti che compaiono in (\ref{eq:gim_after}) sono rappresentati in Fig. \ref{fig:gim_after}.\\
\begin{figure}[!h]
        \centering
        \begin{tikzpicture}
                \begin{feynman}
                        \vertex (a);
                        \vertex [above left=of a](i2) {\(\overline u\)};
                        \vertex [below left=of a] (i1) {\(u\)};
                        \vertex [right= 2.0 cm of a] (b);
                        \vertex [right= 6.0 cm of a](c);
                        \vertex [above left=of c](j2) {\(\overline { d } _{ c }\)};
                        \vertex [below left=of c] (j1) {\(d_{c}\)};
                        \vertex [right= 2.0 cm of c] (d);
                        \vertex [below= 4.0 cm of a](e);
                        \vertex [above left=of e](k2) {\(\overline c\)};
                        \vertex [below left=of e] (k1) {\(c\)};
                        \vertex [right= 2.0 cm of e] (f);
                        \vertex [right= 6.0 cm of e](g);
                        \vertex [above left=of g](l2) {\(\overline { s } _{ c }\)};
                        \vertex [below left=of g] (l1) {\(s_{c}\)};
                        \vertex [right= 2.0 cm of g] (h);
                        \diagram* {
                                (i1) -- [fermion] (a) -- [fermion] (i2),
                                (a) -- [boson, edge label=\(Z^{0}\)] (b),
                                (j1) -- [fermion] (c) -- [fermion] (j2),
                                (c) -- [boson, edge label=\(Z^{0}\)] (d),
                                (k1) -- [fermion] (e) -- [fermion] (k2),
                                (e) -- [boson, edge label=\(Z^{0}\)] (f),
                                (l1) -- [fermion] (g) -- [fermion] (l2),
                                (g) -- [boson, edge label=\(Z^{0}\)] (h),
                        };
                \end{feynman}
        \end{tikzpicture}
        \caption{Diagrammi di Feynman degli elementi che costituiscono la corrente calcolata in Eq. (\ref{eq:gim_after}).}
        \label{fig:gim_after}
\end{figure}
Una conferma di quanto appena scritto si può trovare nel basso valore misurato per il \textit{branching ratio} del decadimento $K ^ { 0 } \rightarrow \mu ^ { + }  \mu ^ { - }$, che risulta essere \cite{ref:PDG}:
\begin{equation}
        B _ { \mu ^ { + } \mu ^ { - } } = \Gamma \left( K ^ { 0 } \rightarrow \mu ^ { + } \mu ^ { - } \right) / \Gamma \left( K ^{ 0 } \rightarrow a l l \right) = ( 6.84 \pm 0.11 ) \times 10 ^ { - 9 }
        \label{eq:k0_br}
\end{equation}
Il diagramma di Feynman del processo è visibile in Fig. \ref{fig:k0_z_decay}.
\begin{figure}[!h]
        \centering
        \begin{tikzpicture}
                \begin{feynman}
                        \vertex (a);
                        \vertex [below left=of a](i2) {\(\overline s\)};
                        \vertex [above left=of a] (i1) {\(d\)};
                        \vertex [right= 3.0 cm of a] (b);
                        \vertex [below right=of b] (f1) {\(\mu^{+}\)};
                        \vertex [above right=of b] (f2) {\(\mu^{-}\)};
                        \diagram* {
                                (i1) -- [fermion] (a) -- [fermion] (i2),
                                (a) -- [boson, edge label=\(Z^{0}\)] (b),
                                (f1) -- [fermion] (b) -- [fermion] (f2),
                        };
                        \draw [decoration={brace}, decorate] (i2.south west) -- (i1.north west)
                        node [pos=0.5, left] {\(K^{0}\)};
                \end{feynman}
        \end{tikzpicture}
        \caption{Diagramma di Feynman per il decadimento del kaone neutro mediato da uno $Z^{0}$.}
        \label{fig:k0_z_decay}
\end{figure}
Il decadimento potrebbe tuttavia procedere anche secondo il diagramma mostrato a sinistra in Fig. \ref{fig:k0_boxes}, il quale porterebbe ad una stima di (\ref{eq:k0_br}) molto maggiore del risultato osservato; adottando il meccanismo GIM si ha che anche il secondo diagramma di Fig. \ref{fig:k0_boxes} contribuisce al calcolo, fornendo un contributo che si cancella quasi esattamente con l'altro e giustificando così il valore molto piccolo di (\ref{eq:k0_br}).

\begin{figure}[!h]
        \centering
        \subfloat{
                \feynmandiagram [layered layout, horizontal=a to b] {
                % Draw the top and bottom lines
                i1 [particle=\(d\)]
                        -- [fermion] a
                        -- [photon, edge label=\(W^{-}\)] b
                        -- [fermion] f1 [particle=\(\mu^{-}\)],
                i2 [particle=\(\overline s\)]
                        -- [anti fermion] c
                        -- [photon, edge label'=\(W^{+}\)] d
                        -- [anti fermion] f2 [particle=\(\mu^{+}\)],
                % Draw the two internal fermion lines
                { [same layer] a -- [fermion, edge label'=\(u\)] c },
                { [same layer] b -- [anti fermion, edge label=\(\nu_{\mu}\)] d},
                };
        }
        \qquad
        \qquad
        \subfloat{
                \feynmandiagram [layered layout, horizontal=a to b] {
                % Draw the top and bottom lines
                i1 [particle=\(d\)]
                        -- [fermion] a
                        -- [photon, edge label=\(W^{-}\)] b
                        -- [fermion] f1 [particle=\(\mu^{-}\)],
                i2 [particle=\(\overline s\)]
                        -- [anti fermion] c
                        -- [photon, edge label'=\(W^{+}\)] d
                        -- [anti fermion] f2 [particle=\(\mu^{+}\)],
                % Draw the two internal fermion lines
                { [same layer] a -- [fermion, edge label'=\(c\)] c },
                { [same layer] b -- [anti fermion, edge label=\(\nu_{\mu}\)] d},
                };

        } \\
        \caption{Diagrammi di Feynman al secondo ordine del decadimento $K ^ { 0 } \rightarrow \mu ^ { + }  \mu ^ { - }$; i due contributi si annullano in manera quasi esatta, essendo il primo proporzionale a $\sin\theta_{c}\cos\theta_{c}$ e il secondo a $-\sin\theta_{c}\cos\theta_{c}$.}
        \label{fig:k0_boxes}
\end{figure}

\subsection{I sei quark e la matrice CKM}
\label{subsec:ckm}
Nel 1973 Kobayashi e Maskawa estesero l'idea di Cabibbo a tre generazioni di quark. Si hanno così i seguenti autostati dell'interazione debole:
\begin{equation}
        \left( \begin{array} { l } { u } \\ { d_{c}  } \end{array} \right)  \quad \left( \begin{array} { l } { c } \\ { s_{c} \end{array} \right) , \quad \left( \begin{array} { l } { t } \\ { b_{c} } \end{array} \right)
\end{equation}
che sono legati agli stati fisici dalla matrice di Cabibbo-Kobayashi-Maskawa:
\begin{equation}
        \left( \begin{array} { l } { d _ { c } } \\ { s _ { c } } \\ { b _ { c } } \end{array} \right) = V _ { C K M } \left( \begin{array} { l } { d } \\ { s } \\ { b } \end{array} \right)
\end{equation}
dove
\begin{equation}
        V_{CKM} = \left( \begin{array} { c } { V _ { u d } V _ { u s } V _ { u b } } \\ { V _ { c d } V _ { c s } V _ { c b } } \\ { V _ { t d } V _ { t s } V _ { t b } } \end{array} \right)
        \label{eq:ckm}
\end{equation}
$V_{CKM}$ è una matrice unitaria, tale cioè che $V ^ { \dagger } V = 1 = V V ^ { \dagger }$, in cui ogni elemento specifica l'accoppiamento tra i due quark scritti a pedice. La parametrizzazione proposta originariamente è la seguente ($c_{i}=\cos\theta_{i}, s_{i}=\sin\theta_{i}$):
\begin{equation}
V _ { C K M } = \left( \begin{array} { c c c } { c _ { 1 } } & { c _ { 3 } s _ { 1 } } & { s _ { 1 } s _ { 3 } } \\ { - c _ { 2 } s _ { 1 } } & { c _ { 1 } c _ { 2 } c _ { 3 } - s _ { 2 } s _ { 3 } e ^ { i \delta } } & { c _ { 1 } c _ { 2 } s _ { 3 } + c _ { 3 } s _ { 2 } e ^ { i \delta } } \\ { s _ { 1 } s _ { 2 } } & { - c _ { 1 } c _ { 3 } s _ { 2 } - c _ { 2 } s _ { 3 } e ^ { i \delta }} & {- c _ { 1 } s _ { 2 } s _ { 3 } + c _ { 2 } c _ { 3 } e ^ { i \delta } } \end{array} \right)
        \label{eq:ckm_param}
\end{equation}
dove $\theta _ { 1 } , \theta _ { 2 } , \theta _ { 3 }$ sono tre angoli di mixing e $\delta$ è un angolo di fase che, se diverso da zero, porta alla violazione di $CP$ nell'interazione debole. Un'altra parametrizzazione molto usata si trova rinominando gli assi cartesiani con il nome dei quark \textit{down-type}, come mostrato in Fig. \ref{fig:mixing3D}, e compiendo una serie di rotazioni attorno ad essi. Operando delle rotazioni nel seguente ordine: la prima di un angolo $\theta_{12}$ attorno all'asse \textit{b}, la seconda di un angolo $\theta_{13}$ attorno all'asse \textit{s} e la terza di un angolo $\theta_{23}$ attorno all'asse delle \textit{d}, si ottiene:
\begin{equation}
        V_{reale}=
        \begin{pmatrix}1 & 0 & 0\\0 & c_{23} & s_{23}\\0 & -s_{23} & c_{23}\end{pmatrix}
        \begin{pmatrix}c_{13} & 0 & s_{13}\\0 & 1 & 0\\-s_{13} & 0 & c_{13}\end{pmatrix}
        \begin{pmatrix}c_{12} & -s_{12} & 0\\s_{12} & c_{12} & 0\\0 & 0 & 1\end{pmatrix}
        \label{eq:CKMsencos}
\end{equation}
L'attuale miglior stima del valore degli angoli è:
\begin{equation}
        \theta_{12}=(13.04\pm0.08)^{\circ}\,\,\,\,\,\,\,\,\,\,\,\theta_{23}=(2.38\pm0.06)^{\circ}\,\,\,\,\,\,\,\,\,\,\,\theta_{13}=(0.201\pm0.011)^{\circ}
\end{equation}
I valori degli angoli di rotazione sono molto piccoli, ma in Fig.~\ref{fig:mixing3D} sono stati esagerati per renderli visibili.
Ora risulta necessario introdurre la fase:
\begin{equation}
        \begin{aligned}
                V= &
                \begin{pmatrix}1 & 0 & 0\\0 & c_{23} & s_{23}\\0 & -s_{23} & c_{23}\end{pmatrix}
                \begin{pmatrix}c_{13} & 0 & s_{13}e^{-i\delta_{13}}\\0 & 1 & s0\\-s_{13}e^{+i\delta_{13}} & 0 & c_{13}\end{pmatrix}
                \begin{pmatrix}c_{12} & -s_{12} & 0\\s_{12} & c_{12} & 0\\0 & 0 & 1\end{pmatrix}\\
                \\
                = &
                \begin{pmatrix}
                c_{12}c_{13} & s_{12}c_{13} & s_{13}e^{-i\delta_{13}}\\
                -s_{12}c_{23}-c_{12}s_{23}s_{13}e^{+i\delta_{13}} & c_{12}c_{23}-s_{12}s_{23}s_{13}e^{+i\delta_{13}} & s_{23}c_{13}\\
                s_{12}s_{23}-c_{12}c_{23}s_{13}e^{+i\delta_{13}} & -c_{12}s_{23}-s_{12}c_{23}s_{13}e^{+i\delta_{13}} & c_{23}c_{13}
                \end{pmatrix}
        \end{aligned}
\end{equation}
L'introduzione della fase in questo modo è possibile solamente se la matrice CKM è unitaria. Il suo valore è stimato essere
\begin{equation}
        \delta_{13}=(1.20\pm0.08)^{\circ}
\end{equation}

\begin{figure}[t]
        \centering
        \begin{tikzpicture}[>=latex, font=\scriptsize]
                \draw[->] (0,0,0) -- (4.5,0,0);
                \draw[->] (0,0,0) -- (0,4.5,0);
                \draw[->] (0,0,0) -- (0,0,5.8);
                \draw[->,line width=1.3pt] (0,0,0) -- (4.4,0.8,-0.5);
                \draw[->,line width=1.3pt] (0,0,0) -- (-0.8,4.3,0.5);
                \draw[->,line width=1.3pt] (0,0,0) -- (1.2,0.5,7.3);
                \draw[dashed] (0,0,0) -- (4.8,0.65,0.5);
                \draw[dashed] (0,0,0) -- (-0.1,4.5,0.5);
                \draw[dashed] (0,0,0) -- (1,0.5,7.3);
                %sintassi: (0,0)center; (startangle=0:radius=3.5) arc (startangle=0:endangle=7,radius=3.5)
                \draw [-latex, black, thick, line width=0.5] (0,0) ++ (0.1:3.5) arc (0.1:5.5:3.5);
                %\fill [top color=white, bottom color=yellow] (0,0) -- (0:3.5) arc (0.2:5.4:3.5);
                \draw [-latex, black, thick,line width=0.5] (0,0) ++ (5.5:3.8) arc (5.5:12:3.8);
                %\fill [top color=white, bottom color=lime] (0,0) -- (5.5:3.8) arc (5.5:11.8:3.8);
                \draw [-latex, black, thick,line width=0.5] (0,0) ++ (90.1:40mm) arc (90.1:94:40mm);
                %\fill [top color=white, bottom color=lightgray] (0,0) -- (90.3:40mm) arc (90.3:93.8:40mm);
                \draw [-latex, black, thick,line width=0.5] (0,0) ++ (94:35mm) arc (94:103.3:35mm);
                %\fill [top color=white, bottom color=lime] (0,0) -- (94.2:35mm) arc (94.2:103.1:35mm);
                \draw [-latex, black, thick,line width=0.5] (0,0) ++ (225:25mm) arc (225:232:25mm);
                %\fill [top color=yellow, bottom color=white] (0,0) -- (225.2:25mm) arc (225.2:231.8:25mm);
                \draw [-latex, black, thick,line width=0.5] (0,0) ++ (232:25.5mm) arc (232:234.5:25.5mm);
                %\fill [top color=lightgray, bottom color=white] (0,0) -- (232.2:25.5mm) arc (232.2:234.3:25.5mm);
                \node at (4.7,0,0) {s};
                \node at (0,4.8,0) {b};
                \node at (0,0,6.3) {d};
                \node at (4.6,0.8,-0.5) {$s_{c}$};
                \node at (-0.8,4.5,0.5) {$b_{c}$};
                \node at (1.2,0.3,7.5) {$d_{c}$};
                \node at (4.1,0.23) {$\theta_{12}$};
                \node at (4.1,1.3) {$\theta_{23}$};
                \node at (-0.3,4.6,0) {$\theta_{13}$};
                \node at (-0.63,3.8,0) {$\theta_{23}$};
                \node at (-0.5,0,4.4) {$\theta_{12}$};
                \node at (1.1,0,5.5) {$\theta_{13}$};
                \end{tikzpicture}
        \caption{Rotazione dei quark.}
        \label{fig:mixing3D}
\end{figure}

Gli elementi della matrice CKM, determinati sperimentalmente, sono i seguenti:
\begin{equation}
        V _ { C K M } = \left( \begin{array} { c c c } { 0.97428 \pm 0.00015 } & { 0.2253 \pm 0.0007 } & { 0.00347 \pm 0.00016 } \\ { 0.2252 \pm 0.0007 } & { 0.97333 \pm 0.00015 } & { 0.041 \pm 0.001 } \\ { 0.0086 \pm 0.0003 } & { 0.040 \pm 0.001 } & { 0.99915 \pm 0.00005 } \end{array} \right)
        \label{eq:ckm_values}
\end{equation}
Il fatto che gli elementi non diagonali siano molto piccoli mentre quelli sulla diagonale siano vicini all'unità ha un significato fisico importante, cioè che il modello predice una specifica sequenza di decadimenti:
\begin{equation}
        t \rightarrow b \rightarrow c \rightarrow s \rightarrow u
\end{equation}
Con lo schema di mixing, quark e leptoni hanno lo stesso accoppiamento e si può così parlare di universalità quark-leptoni.

\appendix
\section{Equazione e formalismo di Dirac}
\label{app:dirac}
\subsection{Equazione e matrici $\gamma$}
L'equazione di Dirac (1928) descrive il comportamento quanto-meccanico e relativistico di particelle puntiformi massive con spin semintero:
\begin{equation}
        \left( i \gamma ^ { \mu } \partial _ { \mu } - m \right) \psi = 0
        \label{eq:dirac_equation}
\end{equation}
dove $\psi$ è un vettore colonna a quattro componenti (spinore)
\begin{equation}
        \psi = \left( \begin{array} { c } { \psi _ { 1 } } \\ { \psi _ { 2 } } \\ { \psi _ { 3 } } \\ { \psi _ { 4 } } \end{array} \right)
        \label{eq:spinor}
\end{equation}
e $\gamma^{\mu} = \left\{ \gamma ^ { 0 } , \gamma ^ { 1 } , \gamma ^ { 2 } , \gamma ^ { 3 } \right\}$ è un set di quattro matrici $4\times 4$ che, seguendo la rappresentazione di Dirac, si scrivono:
\begin{equation}
        \gamma ^ { 0 } = \left( \begin{array} { c c } { 1 } & { 0 } \\ { 0 } & { - 1 } \end{array} \right) \quad \gamma ^ { i } = \left( \begin{array} { c c } { 0 } & { \sigma _ { i } } \\ { - \sigma _ { i } } & { 0 } \end{array} \right)
        \label{eq:gamma_components}
\end{equation}
con le $\sigma_{i}$ matrici di Pauli:
\begin{equation}
        \sigma_{1} = \sigma _ { x } = \left( \begin{array} { c c } { 0 } & { 1 } \\ { 1 } & { 0 } \end{array} \right) \quad \sigma_{2} = \sigma _ { y } = \left( \begin{array} { c c } { 0 } & { - i } \\ { i } & { 0 } \end{array} \right) \quad \sigma_{3} = \sigma _ { z } = \left( \begin{array} { c c } { 1 } & { 0 } \\ { 0 } & { - 1 } \end{array} \right)
        \label{eq:pauli_matrices}
\end{equation}
Scritte per esteso, le matrici $\gamma$ saranno:
\begin{equation}
        \begin{array} {c} \gamma ^ { 0 } = \left( \begin{array} { c c c c } { 1 } & { 0 } & { 0 } & { 0 } \\ { 0 } & { 1 } & { 0 } & { 0 } \\ { 0 } & { 0 } & { - 1 } & { 0 } \\ { 0 } & { 0 } & { 0 } & { - 1 } \end{array} \right)  \quad \gamma ^ { 1 } = \left( \begin{array} { c c c c } { 0 } & { 0 } & { 0 } & { 1 } \\ { 0 } & { 0 } & { 1 } & { 0 } \\ { 0 } & { - 1 } & { 0 } & { 0 } \\ { - 1 } & { 0 } & { 0 } & { 0 } \end{array} \right) \\ \gamma ^ { 2 } = \left( \begin{array} { c c c c } { 0 } & { 0 } & { 0 } & { - i } \\ { 0 } & { 0 } & { i } & { 0 } \\ { 0 } & { i } & { 0 } & { 0 } \\ { - i } & { 0 } & { 0 } & { 0 } \end{array} \right)  \quad \gamma ^ { 3 } = \left( \begin{array} { c c c c } { 0 } & { 0 } & { 1 } & { 0 } \\ { 0 } & { 0 } & { 0 } & { - 1 } \\ { - 1 } & { 0 } & { 0 } & { 0 } \\ { 0 } & { 1 } & { 0 } & { 0 } \end{array} \right) \end{array}
        \label{eq:gamma_extended}
\end{equation}
con
\begin{equation}
        \begin{array} { c } { \left( \gamma ^ { 0 } \right) ^ { 2 } = 1 , \quad \left( \gamma ^ { 1 } \right) ^ { 2 } = \left( \gamma ^ { 2 } \right) ^ { 2 } = \left( \gamma ^ { 3 } \right) ^ { 2 } = - 1 } \\ { \gamma ^ { \mu } \gamma ^ { \nu } + \gamma ^ { \nu } \gamma ^ { \mu } = 0 , \quad \text { per } \mu \neq \nu } \end{array}
        \label{eq:gamma_properties}
\end{equation}
Si evidenziano inoltre i seguenti fatti e proprietà:
\begin{itemize}
        \item in molti casi è utile definire la matrice:
        \begin{equation}
                \gamma ^ { 5 } : = i \gamma ^ { 0 } \gamma ^ { 1 } \gamma ^ { 2 } \gamma ^ { 3 } = \left( \begin{array} { l l l l } { 0 } & { 0 } & { 1 } & { 0 } \\ { 0 } & { 0 } & { 0 } & { 1 } \\ { 1 } & { 0 } & { 0 } & { 0 } \\ { 0 } & { 1 } & { 0 } & { 0 } \end{array} \right)
                \label{eq:gamma5}
        \end{equation}
                con
        \begin{equation}
                \begin{array} { c } { \left( \gamma ^ { 5 } \right) ^ { 2 } = 1 , \quad { \gamma ^ { 5 } \gamma ^ { \mu } + \gamma ^ { \mu } \gamma ^ { 5 } = 0 \end{array}
                \label{eq:gamma5_properties}
        \end{equation}
        \item definendo lo spinore aggiunto:
        \begin{equation}
                \overline { \psi } \equiv \psi ^ { \dagger } \gamma ^ { 0 } = \psi ^ { T ^ { * } } \gamma ^ { 0 }
                \label{eq:spinor_adj}
        \end{equation}
        è possibile costruire le grandezze relativisticamente invarianti elencate in Tab.~\ref{tab:bilinear};
        \item esiste una funzione $\psi^{\prime}$ rappresentabile nella forma:
        \begin{equation}
                \psi ^ { \prime } = S \psi \quad ; \quad \operatorname { con } S ^ { - 1 } \gamma ^ { \mu } S = \sum _ { \nu } a _ { \mu \nu } \gamma ^ { \nu }
                \label{eq:s_operator}
        \end{equation}
        per la quale si ha Lorentz-invarianza dell'equazione di Dirac; questa funzione viene utilizzata per dimostrare \cite{ref:BGSex} la Lorentz-invarianza delle grandezze elencate in Tab. \ref{tab:bilinear};
        \item $\gamma^{0}$ rappresenta l'operatore Parità, per il quale si ha:
        \begin{equation}
                \gamma ^ { 0 } \psi ( - \mathbf { r } , t ) \equiv \psi ( \mathbf { r } , t ) \quad ; \quad \gamma ^ { 0 } \psi ( \mathbf { r } , t ) \equiv \psi ( - \mathbf { r } , t )
        \end{equation}
        \item dato $\mathcal{K}$ operatore di coniugazione complessa, si ha che $\gamma ^ { 1 } \gamma ^ { 3 } \mathcal { K }$ rappresenta l'operatore di inversione temporale, per il quale si ha:
        \begin{equation}
                \gamma ^ { 1 } \gamma ^ { 3 } \mathcal { K } \psi ( \mathbf { r } , t ) = \psi ( \mathbf { r } , - t )
        \end{equation}
        \item $i \gamma ^ { 2 } \mathcal { K }$ rappresenta l'operatore di coniugazione di carica e parità CP, per il quale si ha:
        \begin{equation}
                \left( i \gamma ^ { 2 } \mathcal { K } \right) \psi _ { + } (\mathbf{p}) = \psi _ { - } ( - \mathbf { p } )
        \end{equation}
\end{itemize}

\subsection{Proprietà delle soluzioni}
L'equazione di Dirac è un insieme di quattro equazioni ognuna delle quali deve avere una soluzione.
Tali soluzioni si scrivono nella seguente maniera (utilizzando le unità naturali):
\begin{equation}
        \psi = u e ^ { i ( \mathbf { p } \cdot \mathbf { r } - E t ) } = u e ^ { - i p _ { \mu } x ^ { \mu } }
        \label{eq:dirac_sol}
\end{equation}
dove $u$ è uno spinore a quattro componenti che dipende dal momento. Sostituendo queste soluzioni nella \ref{eq:dirac_equation} si ottiene:
\begin{equation}
        \left( \gamma ^ { \mu } p _ { \mu } - m \right) u = 0
        \label{eq:dirac_eq_momentum}
\end{equation}
detta anche equazione di Dirac nello spazio dei momenti. Sviluppando i conti e utilizzando la rappresentazione
\begin{equation}
        \psi = \left( \begin{array} { l } { u_{A} } \\ { u_{B} } \end{array} \right) e ^ { i ( \mathbf { p } \cdot \mathbf { r } - E t ) }
        \label{eq:dirac_sol_spinors}
\end{equation}
dove $u_{A}$ e $u_{B}$ sono spinori a due componenti, si ottiene:
\begin{equation}
        u _ { A } = \frac { \boldsymbol { \sigma } \cdot \mathbf { p } } { E - m } u _ { B } \quad u _ { B } = \frac { \boldsymbol { \sigma } \cdot \mathbf { p } } { E + m } u _ { A }
\end{equation}
Sostituendo la prima nella seconda (o viceversa), si può vedere come questo sia un sistema omogeneo che ammette soluzione solo se
\begin{equation}
        E = \pm \sqrt { p ^ { 2 } + m ^ { 2 } }
        \label{eq:energy}
\end{equation}
Scegliendo per $u_{A}$ e $u_{B}$ gli autostati dell'operatore $\sigma_{z}$, $u _ { A } = \left( \begin{array} { c } { 1 } \\ { 0 } \end{array} \right) , u _ { A } = \left( \begin{array} { l } { 0 } \\ { 1 } \end{array} \right), u _ { B } = \left( \begin{array} { c } { 1 } \\ { 0 } \end{array} \right) , u _ { B } = \left( \begin{array} { l } { 0 } \\ { 1 } \end{array} \right)$, si ottengono:
\begin{equation}
        \begin{align}
                u ^ { 1 } = \left( \begin{array} { c } { 1 } \\ { 0 } \\ { p _ { z } / ( E + m ) } \\ { \left( p _ { x } + i p _ { y } \right) / ( E + m ) } \end{array} \right)\quad \quad \quad u ^ { 2 } = \left( \begin{array} { c } { 0 } \\ { 1 } \\ { \left( p _ { x } - i p _ { y } \right) / ( E + m ) } \\ { - p _ { z } / ( E + m ) } \end{array} \right) \\ u ^ { 3 } = \left( \begin{array} { c } { - p _ { z } / ( - E + m ) } \\ { \left( - p _ { x } - i p _ { y } \right) / ( - E + m ) } \\ { 1 } \\ { 0 } \end{array} \right) \quad u ^ { 4 } = \left( \begin{array} { c } { \left( - p _ { x } + i p _ { y } \right) / ( - E + m ) } \\ { p _ { z } / ( - E + m ) } \\ { 0 } \\ { 1 } \end{array} \right)
        \end{align}
        \label{eq:u1u2u3u4}
\end{equation}
Le soluzioni $u^{1}$ e $u^{2}$ descrivono particelle di energia (\ref{eq:energy}) positiva e momento $\mathbf{p}$. Per le due rimanenti, che descrivono particelle ad energia negativa, si applica:
\begin{equation}
        \begin{align}
                v ^ { 2 } \left( p ^ { \mu } \right) \equiv u ^ { 3 } \left( - p ^ { \mu } \right) = \left( \begin{array} { c } { p _ { z } / ( E + m ) } \\ { \left( p _ { x } + i p _ { y } \right) / ( E + m ) } \\ { 1 } \\ { 0 } \end{array} \right) \\ v ^ { 1 } \left( p ^ { \mu } \right) \equiv u ^ { 4 } \left( - p ^ { \mu } \right) = \left( \begin{array} { c } { \left( p _ { x } - i p _ { y } \right) / ( E + m ) } \\ { - p _ { z } / ( E + m ) } \\ { 0 } \\ { 1 } \end{array} \right)
        \end{align}
        \label{eq:v1v2}
\end{equation}
così da introdurre l'interpretazione di antiparticelle ad energia positiva. Utilizzando quanto appena riportato, le soluzioni della (\ref{eq:dirac_equation}) si scrivono:
\begin{equation}
        \begin{align}
                \psi = u ^ { 1 } \left( p ^ { \mu } \right) e ^ { - i p_{\mu} x^{\mu} } \quad \psi = u ^ { 2 } \left( p ^ { \mu } \right) e ^ { - i p_{\mu} x^{\mu} } \\ \psi = v ^ { 1 } \left( p ^ { \mu } \right) e ^ {  i p_{\mu}  x^{\mu} } \quad \psi = v ^ { 2 } \left( p ^ { \mu } \right) e ^ {  i p_{\mu} x^{\mu} }
        \end{align}
        \label{eq:u1u2v1v2}
\end{equation}

\subsection{Elicità e chiralità}
Ricordando che
\begin{equation}
        \hat{\mathbf { S } } = \frac { \hbar } { 2 } \hat{\mathbf { \Sigma } } \quad , \quad \hat{\mathbf { \Sigma } } = \left( \begin{array} { l l } { \boldsymbol{\sigma }} & { 0 } \\ { 0 } & { \boldsymbol{\sigma} } \end{array} \right)
\end{equation}
si può facilmente verificare che $u^{1}, u^{2}, v^{1}, v^{2}$ sono autostati di $\hat { S _ { z } }$ solo se il momento è lungo l'asse $z$. \\
In maniera più generale, è conveniente definire l'operatore elicità, che estrae la componente dello spin lungo la direzione del moto di una particella:
\begin{equation}
        \hat { \Lambda }  = \frac { \hat { \mathbf { \Sigma } } \cdot \mathbf{p} } { |\mathbf{p}| }
        \label{eq:helicity}
\end{equation}
Per fermioni di spin semintero, gli autovalori di $\hat{\Lambda}$ possono essere $\lambda=1$ (particella destrorsa o ad elicità positiva) e $\lambda=-1$ (particella sinistrorsa o ad elicità negativa). Per poter inserire nelle espressioni delle hamiltoniane funzioni d'onda che abbiano uno specifico stato di elicità, si introducono gli operatori di proiezione:
\begin{equation}
        \hat { P } _ { + } = \frac { 1 + \hat { \Lambda } } { 2 } \quad , \quad \hat { P }_{-} = \frac { 1 - \hat { \Lambda } } { 2 }
        \label{eq:projection_operators}
\end{equation}
che estraggono le componenti di elicità rispettivamente positiva e negativa da uno spinore arbirario. Scrivendo:
\begin{equation}
        \mathbb{1} = \frac { 1 - \hat{\Lambda } } { 2 } + \frac { 1 + \hat{\Lambda } } { 2 } = \hat{P}_{-} + \hat{P} _ { + }
\end{equation}
si conclude infatti che uno stato qualsiasi $u$ si può sempre scomporre:
\begin{equation}
        \mathbb { 1 } u = \left( \hat{P} _ { + } + \hat{P} _ { - } \right) u = u _ { + } + u _ { - }
\end{equation}
con
\begin{equation}
        \hat { P } _ { + } u = u_{+}\quad ,\quad \hat { P } _ { - } u = u_{-}
\end{equation}
Se invece si considera lo spinore di un antifermione si avranno:
\begin{equation}
        \hat { P } _ { + } v = v_{-} \quad ,\quad \hat { P } _ { - } v = v _ { + }
\end{equation}
Le componenti $u_{+}$, $u_{-}$, $v_{+}$ e $v_{-}$ sono dette a elicità definita.
L'elicità non è tuttavia una quantità Lorentz-invariante: eseguendo una trasformazione di Lorentz, è possibile porsi in un sistema di riferimento in cui l'impulso della particella inverte la sua direzione, e quindi in cui l'elicità cambia segno. Di conseguenza, le componenti a elicità definita di un dato stato sono diverse in diversi riferimenti.
Risulta quindi conveniente introdurre una quantità che sia Lorentz-invariante: la chiralità. Essa può essere definita attraverso i seguenti operatori di proiezione:
\begin{equation}
        \hat{P}_ { L } = \frac { 1 } { 2 } \left( 1 - \gamma ^ { 5 } \right) \quad , \quad \hat{P} _ { R } = \frac { 1 } { 2 } \left( 1 + \gamma ^ { 5 } \right)
        \label{eq:plpr}
\end{equation}
Anche in questo caso, uno stato si può scomporre in stati a chiralità definita:
\begin{equation}
        \mathbb { 1 } u = \left( \hat{P} _ { R } + \hat{P} _ { L } \right) u = u _ { R } + u _ { L }
\end{equation}
con
\begin{equation}
        u _ { L } = \hat{P} _ { L } u \quad u _ { R } = \hat{P} _ { R } u
\end{equation}
e, per gli antifermioni:
\begin{equation}
        v _ { R } = \hat{P} _ { L } v \quad v _ { L } = \hat{P} _ { R } v
\end{equation}
Si notino le seguenti importanti caratteristiche che dipendono dal fatto che una particella sia massiva o meno.
\paragraph{Particelle massive} In questo caso, uno stato a chiralità definita risulta essere composto da una sovrapposizione di stati di elicità positiva e negativa, con pesi diversi:
\begin{equation}
        p_{+} = \left| \left\langle u _ { + } | u \right\rangle \right| ^ { 2 } \quad , \quad p_{-} = \left| \left\langle u _ { - } | u \right\rangle \right| ^ { 2 }
\end{equation}
grazie all'introduzione di questi pesi è possibile calcolare il valor medio dell'elicità per stati a chiralità definita, che risulta coincidere con la polarizzazione longitudinale:
\begin{equation}
        \langle \hat{\Lambda} \rangle = \frac { p_{+} - p_{-}} { p_{+} + p_{-} }
        \label{eq:mean_value_helicity}
\end{equation}
Questo valore si può calcolare e risulta
\begin{equation}
        \langle \hat{\Lambda} \rangle = - \frac { p } { | E | } = - \frac { v } { c } = - \beta
\end{equation}
per stati chirali \textit{left}, con segno opposto per stati chirali \textit{right}. Quindi, dato un fermione massivo in uno stato chirale \textit{left} o \textit{right}, c'è in generale una probabilità non nulla di trovarlo con elicità opposta alla chiralità, probabilità che decresce all'aumentare della velocità.
\paragraph{Particelle massless} Nel caso di particelle con massa nulla, gli operatori di elicità e chiralità coincidono.
Si ha infatti:
\begin{equation}
        \hat{P} _ { +,- } = \frac { 1 \pm \hat{\Lambda}} { 2 } \rightarrow \frac { 1 \pm \gamma ^ { 5 } } { 2 } = \hat{P} _ { R , L }
\end{equation}
con l'uguaglianza che diventa esatta nel caso in cui la massa sia nulla. Si può quindi dire che particelle \textit{massless} hanno sempre valore definito di elicità.

\section{Regole di Feynman}
\label{app:feynman_rules}
Le regole di Feynman consistono in un insieme di passaggi che consentono di ricavare l'ampiezza $\mathcal{M}$ di un dato processo a partire dal diagramma di Feynman. La forma più elementare è quella in cui tutte le particelle hanno spin zero; essa si scrive come segue:
\begin{enumerate}
        \item si etichettano i quadri-momenti entranti e uscenti $p_{1}, p_{2}, \ldots, p_{n}$, definendo una direzione positiva assegnata arbitrariamente, e i quadri-momenti interni $q_{1}, q_{2}, \ldots, q_{m}$;
        \item per ogni vertice si scrive un fattore
                \begin{equation}
                        -ig
                \end{equation}
                dove g è la costante di accoppiamento;
        \item per ogni linea interna si scrive un fattore
                \begin{equation}
                        \frac { i } { q _ { j } ^ { 2 } - m _ { j } ^ { 2 } c ^ { 2 } }
                \end{equation}
                dove $q_{j}$ è il quadri-momento della linea ($q _ { j } ^ { 2 } \equiv q _ { j } ^ { \mu } q _ { j  \mu } $) e $m_{j}$ è la massa della particella che la linea descrive (si noti che $q _ { j } ^ { 2 } \neq m _ { j } ^ { 2 } c ^ { 2 }$, perché la particella è virtuale);
        \item per ogni vertice si scrive un fattore
                \begin{equation}
                        ( 2 \pi ) ^ { 4 } \delta ^ { 4 } \left( p _ { 1 } + p _ { 2 } + \ldots - p _ { n } \right)
                \end{equation}
                dove si hanno segni positivi per le particelle entranti e negativi per quelle uscenti; questo fattore impone la conservazione di energia e momento ad ogni vertice, dal momento che la delta è nulla a meno che la somma dei momenti entranti uguagli quella dei momenti uscenti;
        \item per ogni linea interna si scrive un fattore \begin{equation}
                        \frac { 1 } { ( 2 \pi ) ^ { 4 } } d ^ { 4 } q _ { j }
                \end{equation}
                e si integra su tutti i $q_{j}$;
        \item si semplificano i fattori e si uguaglia quanto rimasto a $-i\mathcal{M}$.
\end{enumerate}
Nel caso dell'interazione debole, in cui compaiono fermioni di spin 1/2 e bosoni di spin unitario, si hanno i seguenti passaggi:
\begin{enumerate}
        \item si etichettano i quadri-momenti entranti e uscenti $p_{1}, p_{2}, \ldots, p_{n}$, con i corrispondenti spin $s_{1}, s_{2}, \ldots, s_{n}$ e i quadri-momenti interni $q_{1}, q_{2}, \ldots, q_{m}$; per le linee esterne le direzioni delle frecce vengono assegnate distinguendo fermioni da antifermioni, mentre per quelle interne si assegnano preservando la "direzione del flusso" del diagramma;
        \item fermioni e antifermioni esterni contribuiscono con gli spinori introdotti in Eq. (\ref{eq:u1u2u3u4}) e (\ref{eq:v1v2});
        \item per ogni vertice si scrive un fattore
                \begin{equation}
                        \frac { - i g _ { w } } { 2 \sqrt { 2 } } \gamma ^ { \mu } \left( 1 - \gamma ^ { 5 } \right)
                \end{equation}
        \item per ogni linea interna si scrivono i seguenti fattori:
                \begin{equation}
                        \begin{array} { l l } { \text { fermioni e antifermioni: } } & { \frac { i \left( \gamma ^ { \mu } q _ { \mu } + m c \right) } { q ^ { 2 } - m ^ { 2 } c ^ { 2 } } } \\ { \text { bosoni: } } & { \frac{- i \left( g _ { \mu \nu } - q _ { \mu } q _ { v } / M ^ { 2 } c ^ { 2 } \right)}{ q ^ { 2 } - M ^ { 2 } c ^ { 2 }  }   }\\ \end{array}
                \end{equation}
        \item i punti successivi sono i 4, 5 e 6 del modello precedente.
\end{enumerate}


\begin{thebibliography}{0}
        \bibitem{ref:PDG}
                \BY{M. Tanabashi et al. (Particle Data Group)}
                \TITLE{Review of Particle Physics},
                \JOURNAL{Phys. Rev. D},
                \YEAR{2018}
        \bibitem{ref:griff}
                \BY{Griffiths D.}
                \TITLE{Introduction to elementary particles},
                \YEAR{1987}
        \bibitem{ref:BGS}
                \BY{Braibant S., Giacomelli G., Spurio M.}
                \TITLE{Particelle e interazioni fondamentali},
                \PUBLISHER{Springer},
                \YEAR{2012}
        \bibitem{ref:hayes}
                \BY{Hayes C.B.}
                \TITLE{Neutron Beta-Decay},
                \YEAR{2012}
        \bibitem{ref:BGSex}
                \BY{Braibant S., Giacomelli G., Spurio M.}
                \TITLE{Particles and Fundamental Interactions: Supplements, Problems and Solutions},
                \PUBLISHER{Springer},
                \YEAR{2012}
        \bibitem{ref:greiner}
                \BY{Greiner W., Muller B.}
                \TITLE{Gauge theory of weak interactions},
                \PUBLISHER{Springer},
                \YEAR{2000}
\end{thebibliography}

\end{document}

%%
