%%%%%%%%%%%%%%%%%%%%%%%%%%%%%%%%%%%%%%%%%%%%%%%%%%
%%%%%%%%%%%%%%%%%%%%%%%%%%%%%%%%%%%%%%%%%%%%%%%%%%
%%%%%%%%%%%%%%%%%%%%%%%%%%%%%%%%%%%%%%%%%%%%%%%%%%

\documentclass{subnucbo}

\usepackage{graphicx}
% got figures? uncomment this


\title{Sample paper}
% Insert your title here



\author{M. Rossi}
% Insert your name(s) here.  Quick note: Authors must be separated
% with \from{n} to mark their referring
% institute, e.g.:
% \author{A.U. Thor\from{1}, W.R. Iter\from{2} \atque A. Einstein\from{3}\thanks{Thanks here}}
% TeX automagically determines the institute you are referring to.





\instlist{\inst{} Dipartimento di Fisica Universit\`a di Bologna, Via Irnerio 46 - 40126 Bologna, Italy}
% Insert your institute(s) here. Quick note: as above, e.g.
%\instlist{\inst{1} Dipartimento di Fisica Universit\`a di Bologna, Via Irnerio 46 - 40126 Bologna, Italy
%\inst{2} Dipartimento di Fisica dell'Universit\`{a} and INFN, Sezione di Milano - Milano, Italy
%\inst{3} Societ\`a Italiana di Fisica}




\acyear{2010--2011}
% Insert the academic year




\begin{document}

\maketitle

\begin{abstract}
This sample paper is intended to briefly expose \LaTeX\ package \texttt{subnucbo}.
\end{abstract}

\section{Description}
This is a very short sample paper distributed with the class
\texttt{subnucbo}.
It is just a collection of examples about the syntax of commands
which behave in a different way from the standard \LaTeX\
and/or new commands not defined in \LaTeX.


You can also use this file as a template for your own paper:
copy it to another filename and then modify as needed.

\section{Examples}


\subsection{Mathematics}
Here is a lettered array~(\ref{e.all}), with eqs.~(\ref{e.house})
and~(\ref{e.phi}):
\begin{eqnletter}
 \label{e.all}
 \drm x_\sy{F} & = & 1.2\cdot10^3\un{cm}, \qquad
                     \tx{where\ } \sy{F} = \tx{Fermi}    \label{e.house}\\
 \phi_i        & = & i\pi                                \label{e.phi}
\end{eqnletter}

\subsection{Tables}
Tables~\ref{tab:pricesI}
inserted at this point.

\begin{table}
  \caption{Prices of important items.}
  \label{tab:pricesI}
  \begin{tabular}{rcl}
    \hline
      Ice-cream      & 2.50  & euro    \\
      More ice-cream & 2500 & euro    \\
      Crocodile      & 1500  & dollars \\
    \hline
      Phone call     & .25   & dollars \\
      X-Men          & 1.25  & dollars \\
      Dollar         & 1     & dollars \\
    \hline
  \end{tabular}
\end{table}

\subsection{Citations}
We're almost done, just some citations~\cite{ref:apo}
and we will be over~\cite{ref:pul,ref:bra}.


\appendix

\section{}
Let us go then, you and I\ldots

\acknowledgments
This work was produced, supported and perpetrated by M. Bellacosa under
the auspices of the Italian Physical Society.

\begin{thebibliography}{0}
\bibitem{ref:apo} \BY{Boccaccio~G. \atque de~Cam\~oes~L.}
  \IN{Phys. Rev. A}{13}{1999}{12};
  \SAME{69}{999}{1666}.
\bibitem{ref:pul} \BY{Pulci~L.}
  preprint INFN 8181.
\bibitem{ref:bra} \BY{Bragg~B.}
  \TITLE{Tender comrade},
  in \TITLE{Workers Playtime},
                  edited by \NAME{Tizio A. \atque Caio B.}
                  (Unexeditor, Bologna) 1997, pp.~1-10.
\end{thebibliography}

\end{document}

%%
