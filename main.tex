%%%%%%%%%%%%%%%%%%%%%%%%%%%%%%%%%%%%%%%%%%%%%%%%%%
%%%%%%%%%%%%%%%%%%%%%%%%%%%%%%%%%%%%%%%%%%%%%%%%%%
%%%%%%%%%%%%%%%%%%%%%%%%%%%%%%%%%%%%%%%%%%%%%%%%%%
\documentclass{subnucbo}

\usepackage{graphicx}
% got figures? uncomment this

% lingua italiana
\usepackage[italian]{babel}
\usepackage[utf8x]{inputenc}

% feynman diagrams
\usepackage[compat=1.1.0]{tikz-feynman}

% math
\usepackage{amsmath}

\title{Processi Deboli e Mixing dei Quark}
% Insert your title here

\author{Massimiliano Galli}
% Insert your name(s) here.  Quick note: Authors must be separated
% with \from{n} to mark their referring
% institute, e.g.:
% \author{A.U. Thor\from{1}, W.R. Iter\from{2} \atque A. Einstein\from{3}\thanks{Thanks here}}
% TeX automagically determines the institute you are referring to.

\instlist{\inst{} Dipartimento di Fisica Universit\`a di Bologna, Via Irnerio 46 - 40126 Bologna, Italy}
% Insert your institute(s) here. Quick note: as above, e.g.
%\instlist{\inst{1} Dipartimento di Fisica Universit\`a di Bologna, Via Irnerio 46 - 40126 Bologna, Italy
%\inst{2} Dipartimento di Fisica dell'Universit\`{a} and INFN, Sezione di Milano - Milano, Italy
%\inst{3} Societ\`a Italiana di Fisica}

\acyear{2017--2018}
% Insert the academic year


\begin{document}

\maketitle

\begin{abstract}
        This sample paper is intended to briefly expose \LaTeX\ package \texttt{subnucbo}.
\end{abstract}

\section{Introduzione}
\subsection{Caratteristiche generali dell'interazione debole}
L'interazione debole presenta una serie di proprietà che la rendono peculiare rispetto alle altre interazioni fondamentali:
\begin{itemize}
        \item è l'unica interazione capace di cambiare il \textit{flavor} dei quark;
        \item non conserva alcune quantità che sono invece conservate nelle altre interazioni, quali ad esempio parità, coniugazione di carica e stranezza;
        \item è la sola interazione che coinvolge i neutrini;
        \item è responsabile del decadimento beta dei nuclei, processo che rappresenta la trasmutazione di un elemento (\textit{Z}, \textit{N}), ove \textit{Z} è il numero di protoni ed \textit{N} quello di neutroni del nucleo verso un nucleo con $\textit{Z} + 1$ protoni (decadimento beta negativo) oppure con $\textit{Z} - 1$ protoni (decadimento beta positivo).
\end{itemize}
Essa presenta inoltre dei bosoni mediatori ($W^{+}, W^{-}, Z^{0}$) molto massivi
\begin{equation}
        M _ { W } = 82 \pm 2 \text { GeV } / c ^ { 2 } , \quad M _ { Z } = 92 \pm 2 \text { GeV } / c ^ { 2 }
        \label{eq:bosons_masses}
\end{equation}
i quali fanno sì che l'interazione debole abbia un cortissimo raggio d'azione. I bosoni $W^{+}$ e $W^{-}$ sono responsabili dell'interazione debole a corrente carica (CC) mentre $Z^{0}$ dell'interazione debole a corrente neutra (NC). I vertici dell'interazione debole sono rappresentati in Fig.~\ref{fig:weak_vertex}.
\begin{figure}[!h]
        \centering
        \subfloat{
                \feynmandiagram [layered layout, horizontal=a to b] {
                        a [particle=\(l^{-}\)] -- [fermion] b -- [fermion] c [particle=\(\nu_{l}\)],
                        b -- [boson, edge label'=\(W^{-}\)] d,
                };
        }
        \qquad
        \qquad
        \subfloat{
                \feynmandiagram [layered layout, horizontal=a to b] {
                        a [particle=\(l^{+}\)] -- [fermion] b -- [fermion] c [particle=\(\overline{\nu_{e}}\)],
                        b -- [boson, edge label'=\(W^{+}\)] d,
                };
        } \\
        \subfloat{
                \feynmandiagram [layered layout, horizontal=a to b] {
                        a [particle=\(q^{-1/3}\)] -- [fermion] b -- [fermion] c [particle=\(q^{+2/3}\)],
                        b -- [boson, edge label'=\(W^{-}\)] d,
                };
        }
        \qquad
        \qquad
        \subfloat{
                \feynmandiagram [layered layout, horizontal=a to b] {
                        a [particle=\(q^{+1/3}\)] -- [fermion] b -- [fermion] c [particle=\(q^{-2/3}\)],
                        b -- [boson, edge label'=\(W^{+}\)] d,
                };
        } \\
        \subfloat{
                \feynmandiagram [layered layout, horizontal=a to b] {
                        a [particle=\(l\)] -- [fermion] b -- [fermion] c [particle=\(l\)],
                        b -- [boson, edge label'=\(Z^{0}\)] d,
                };
        }
        \qquad
        \qquad
        \subfloat{
                \feynmandiagram [layered layout, horizontal=a to b] {
                        a [particle=\(q\)] -- [fermion] b -- [fermion] c [particle=\(q\)],
                        b -- [boson, edge label'=\(Z^{0}\)] d,
                };
        }
        \caption{Vertici dell'interazione debole.}
        \label{fig:weak_vertex}
\end{figure}

\subsection{Nozioni fondamentali}
In questa parte si introducono nozioni alle quali si farà riferimento in seguito.\\
Attraverso la teoria perturbativa è possibile ricavare la probabilità di transizione da uno stato iniziale \textit{i} ad uno stato finale \textit{f}:
\begin{equation}
        \Gamma = \frac{2 \pi} {\hbar} \cdot | \mathcal{M}|^{2} \rho_{f}
        \label{eq:decay_rate}
\end{equation}
dove $\rho_{f}$ è la densità degli stati per unità di intervallo di energia (spazio delle fasi), mentre $\mathcal{M}$ è l'elemento di matrice per la probabilità di transizione e si calcola valutando i diagrammi di Feynman più rilevanti e applicando le regole opportune. La (\ref{eq:decay_rate}) è anche detta "regola d'oro di Fermi". \\
Inoltre è possibile dimostrare \cite{ref:griff} che per processi di decadimento è possibile scrivere la vita media della particella $\tau$ come:
\begin{equation}
        \tau = \frac{1}{\Gamma}
        \label{eq:tau}
\end{equation}
dove $\Gamma$ corrisponde in realtà alla sommatoria su tutti gli $n$ modi di decadimento della particella:
\begin{equation}
        \Gamma = \sum _ { i = 1 } ^ { n } \Gamma _ { i }
        \label{eq:gamma_sum}
\end{equation}
Si introduce infine la regola di Sargent~\cite{ref:BGS}: se $\tau$ è la vita media di una particella e $\Gamma_{i}/\Gamma$ è il suo \textit{branching ratio} in un particolare decadimento debole a tre corpi nello stato finale. allora la probabilità di transizione corrisponde a:
\begin{equation}
        W = \frac{(\Gamma_{i}/\Gamma)}{\tau} \simeq G_{F}^{2}E_{0}^{5} \simeq G_{F}^{2}\Delta m^{5}
        \label{eq:sargent_rule}
\end{equation}

\section{Verso l'universalità dell'interazione debole: la necessità del \textit{mixing} e l'intuizione di Cabibbo}
La teoria di Fermi, sviluppata a partire dal 1934, è considerata il prototipo dell'interazione debole: essa è puntiforme (in quanto coinvolge l'interazione di quattro fermioni in un punto) e presenta una costante di accoppiamento $G_{F}$, detta \textit{costante di Fermi}. Una rappresentazione grafica della teoria è mostrata in Fig. \ref{fig:fermi_decay}. Applicando la teoria a diversi decadimenti, come ad esempio nei casi di neutrone e muone, risulta evidente che $G_{F}$ calcolata in funzione dei risultati sperimentali dà risultati leggermente diversi. \\
Questo fatto, insieme ad alcune anomalie rilevate nei decadimenti deboli di particelle strane, portò alla necessità di aggiungere un nuovo tassello alla teoria.
\begin{figure}[!h]
        \centering
        \feynmandiagram [layered layout, horizontal= nuc1 to a ] {
                nuc1 [particle =\(^{A}_{Z}X\)]-- [fermion] a,
                a -- [anti fermion] neutrino [particle =\(\overline \nu_{e}\)],
                a -- [fermion] nuc2 [particle =\(^{A}_{Z+1}Y\)],
                a -- [fermion] el [particle=\(e^{-}\)],
        };
        \caption{Vertice a quattro fermioni della teoria di Fermi.}
        \label{fig:fermi_decay}
\end{figure}
\subsection{Teoria V-A delle interazioni deboli}
\label{subsec:v-a}
La teoria $V-A$ dell'interazione debole venne sviluppata a partire dal 1957 da Feynmann e Gell-Mann come estensione della teoria di Fermi: ciò che rese necessario un aggiornamento della teoria fu la scoperta, avvenuta alcuni anni prima, della violazione della parità da parte dell'interazione debole. \\
Formalizzata in analogia con l'interazione elettromagnetica, essa si basa sui seguenti fatti:
\begin{itemize}
        \item per particelle con spin 1/2, le funzioni d'onda appropriate sono spinori a quattro componenti soddisfacenti l'equazione di Dirac (Appendice \ref{app:dirac});
        \item l'ampiezza del processo è proporzionale al quadrivettore densità di corrente attraverso una costante da determinarsi.
\end{itemize}
In generale, una corrente può essere scritta nella seguente forma:
\begin{equation}
        J ^ { \mu } = \overline { \psi } O _ { i } \psi
        \label{eq:current}
\end{equation}
dove $J ^ { \mu } = \left( J ^ { 0 } , \vec { J } \right)$, $\psi$ e $\overline{\psi}$ sono spinori a quattro componenti e $O_{i}$ è l'operatore che definisce il tipo di interazione, combinazione lineare delle matrici di Dirac $\gamma^{\mu}$.
Le (\ref{eq:current}) sono forme bilineari che si trasformano sotto trasformazioni di Lorentz in modo analogo ad una quantità scalare (S), pseudo-scalare (P), vettoriale (V), assiale (A) e tensoriale (T); a seconda dell'operatore scelto, l'indice in $O_{i}$ può assumere i valori $i = S, P, V, T, A$. Le possibili correnti sono riportate in Tab~\ref{tab:bilinear}.
\begin{table}
        \centering
        \begin{tabular}{l  c  c}
                \hline
                & Corrente & Numero di Componenti \\
                \hline
                Scalare & $\overline { \psi } \psi$ & 1 \\
                Vettore & $\overline { \psi } \gamma ^ { \mu } \psi$ & 4 \\
                Tensore & $\overline { \psi } \sigma ^ { \mu \nu } \psi$ & 6 \\
                Vettore assiale & $\overline { \psi } \gamma ^ { \mu } \gamma ^ { 5 } \psi$ & 4 \\
                Pseudo-scalare & \overline { \psi } \gamma ^ { 5 } \psi & 1 \\
                \hline
        \end{tabular}
        \caption{Forme bilineari.}
        \label{tab:bilinear}
\end{table}
L'operatore di un'interazione descritta dallo scambio di una particella di spin unitario può avere solo natura vettoriale o assiale. Tuttavia è possibile mostrare che per avere violazione di parità (come accade nell'interazione debole) è necessaria una teoria che misceli correnti vettoriali e assiali, in quanto queste due, prese singolarmente, conservano la parità. Si noti che "conservare la parità" significa che i bosoni mediatori possono accoppiarsi in ugual modo con particelle sinistrorse o destrorse. \\
Potendo scrivere in maniera del tutto generale l'elemento di matrice per l'interazione debole nella seguente maniera:
\begin{equation}
        \mathcal{M} = G _ { F } \eta _ { \mu \nu } J ^ { \mu } J ^ { \nu }
        \label{eq:weak_matrix_element}
\end{equation}
si può verificare come un operatore composto da una miscela di $V$ e $A$ non sia invariante sotto l'operazione di parità. L'elemento di matrice è:
\begin{equation}
        \begin{aligned}
                M _ { i f } & \propto \eta _ { \mu \nu } \left( V _ { 1 } ^ { \mu } - A _ { 1 } ^ { \mu } \right) \left( V _ { 2 } ^ { \mu } - A _ { 2 } ^ { \mu } \right) \\ & = \left( V _ { 1 } ^ { 0 } - A _ { 1 } ^ { 0 } \right) \left( V _ { 2 } ^ { 0 } - A _ { 2 } ^ { 0 } \right) - \left( \vec { V } _ { 1 } - \vec { A } _ { 1 } \right) \left( \vec { V } _ { 2 } - \vec { A } _ { 2 } \right)
        \end{aligned}
\end{equation}
che sotto parità si trasforma come:
\begin{equation}
        \begin{aligned}
                M _ { i f } & = \left( V _ { 1 } ^ { 0 } - A _ { 1 } ^ { 0 } \right) \left( V _ { 2 } ^ { 0 } - A _ { 2 } ^ { 0 } \right) - \left( \vec { V } _ { 1 } - \vec { A } _ { 1 } \right) \left( \vec { V } _ { 2 } - \vec { A } _ { 2 } \right) \\ \stackrel { P } { \longrightarrow } M _ { i f } & = \left( V _ { 1 } ^ { 0 } + A _ { 1 } ^ { 0 } \right) \left( V _ { 2 } ^ { 0 } + A _ { 2 } ^ { 0 } \right) - \left( - \vec { V } _ { 1 } - \vec { A } _ { 1 } \right) \left( - \vec { V } _ { 2 } - \vec { A } _ { 2 } \right) \\ & = \left( V _ { 1 } ^ { 0 } + A _ { 1 } ^ { 0 } \right) \left( V _ { 2 } ^ { 0 } + A _ { 2 } ^ { 0 } \right) - \left( \vec { V } _ { 1 } + \vec { A } _ { 1 } \right) \left( \vec { V } _ { 2 } + \vec { A } _ { 2 } \right)
        \end{aligned}
\end{equation}

\subsection{Decadimento del muone}
Il muone decade secondo:
\begin{equation}
        \mu^{-} \rightarrow e^{-} \overline{\nu_{e}} \nu_{\mu}
        \label{eq:muon_decay}
\end{equation}
Il diagramma di Feynman del processo è mostrato in Fig. \ref{fig:muon_decay}.
\begin{figure}[!h]
        \centering
        \feynmandiagram [layered layout, horizontal=a to b] {
                a [particle=\(\mu^{-}\)] -- [fermion] b -- [fermion] f1 [particle=\(\nu_{\mu}\)],
                b -- [boson, edge label'=\(W^{-}\)] c,
                c -- [anti fermion] f2 [particle=\(\overline \nu_{e}\)],
                c -- [fermion] f3 [particle=\(e^{-}\)],
        };
        \caption{Diagramma di Feynman per il decadimento del muone.}
        \label{fig:muon_decay}
\end{figure}
Per l'interazione debole, le regole di Feynman~\cite{ref:griff} prevedono il fattore V-A
\begin{equation}
        \mathcal { K } ^ { \mu } \rightarrow - \frac { i g _ { w } } { 2 \sqrt { 2 } } \gamma ^ { \mu } \left( 1 - \gamma ^ { 5 } \right)
\end{equation}
ad entrambi i vertici. A causa della grande massa del bosone $W^{-}$, il propagatore può essere approssimato da
\begin{equation}
        \mathcal { D } _ { \mu \nu } \rightarrow i \frac { g _ { \mu \nu } } { M _ { W } ^ { 2 } }
        \label{eq:propagator}
\end{equation}
Inserendo questi valori nell'espressione per l'ampiezza di decadimento si trova:
\begin{equation}
        \mathcal { M } = \frac { g _ { w } ^ { 2 } } { 8 M _ { W } ^ { 2 } } \left[ \overline { u } \left( \nu _ { \mu } \right) \gamma ^ { \mu } \left( 1 - \gamma ^ { 5 } \right) u ( \mu ) \right] \cdot \left[ \overline { u } \left( e ^ { - } \right) \gamma _ { \mu } \left( 1 - \gamma ^ { 5 } \right) v \left( \nu _ { e } \right) \right]
        \label{eq:amplitude_muon}
\end{equation}
dove $u$ e $v$ sono spinori di Dirac nello spazio dei momenti con $\overline { u } = u ^ { \dagger } \gamma ^ { 0 }$ e $\overline { v } = v ^ { \dagger } \gamma ^ { 0 }$. Sviluppando il calcolo e utilizzando la regola d'oro di Fermi si trova:
\begin{equation}
        \Gamma = \frac { m _ { \mu } ^ { 5 } c ^ { 4 } } { 192 \pi ^ { 3 } \hbar ^ { 7 } } G _ { F } ^ { 2 }
        \label{eq:muon_decay_rate}
\end{equation}
dove la costante di Fermi $G_{F}$ è definita da
\begin{equation}
        G _ { F } \equiv \frac { \sqrt { 2 } } { 8 } \left[ \frac { g _ { w } } { M _ { W } } \right] ^ { 2 } \cdot ( \hbar c ) ^ { 3 }
        \label{eq:fermi_constant}
\end{equation}
Ridefinendo poi
\begin{equation}
        G_{F} \leftrightarrow \frac{G_{F}}{\hbar^{3}c^{3}}
        \label{eq:fermi_constant_ridef}
\end{equation}
e sostituendo i valori sperimentali per la vita media del muone si ottiene:
\begin{equation}
        G^{\mu} _ { F } = 1.16637 \times 10 ^ { - 5 } \mathrm { GeV } ^ { - 2 }
        \label{eq:gf_muon_value}
\end{equation}

\subsection{Decadimento del neutrone}
Il neutrone decade secondo:
\begin{equation}
        n \rightarrow p e^{-} \overline{\nu_{e}}
        \label{eq:neutron_decay}
\end{equation}
Una prima approssimazione del decadimento beta del neutrone si ottiene assumendo neutrone e protone particelle puntiformi che si accoppiano direttamente con il bosone mediatore (Fig.~\ref{fig:neutron_decay_simple}).
\begin{figure}[!h]
        \centering
        \feynmandiagram [layered layout, horizontal=a to b] {
                a [particle=\(n\)] -- [fermion] b -- [fermion] f1 [particle=\(p\)],
                b -- [boson, edge label'=\(W^{-}\)] c,
                c -- [anti fermion] f2 [particle=\(\overline \nu_{e}\)],
                c -- [fermion] f3 [particle=\(e^{-}\)],
        };
        \caption{Diagramma di Feynman per il decadimento del neutrone, assumendo protone e neutrone non costituiti da quark.}
        \label{fig:neutron_decay_simple}
\end{figure}
L'elemento di matrice è determinato facilmente sostituendo $u(\nu_{\mu})$ e $u(\mu)$ in \ref{eq:amplitude_muon} con $u(n)$ e $u(p)$. Omettendo i dettagli del calcolo, che sono tuttavia descritti in \cite{ref:hayes}, si ricava:
\begin{equation}
        \Gamma = \frac { 2 \rho_{f} } { \pi ^ { 3 } \hbar ^ { 7 } } G _ { F } ^ { 2 } m _ { e } ^ { 5 } c ^ { 4 }
        \label{eq:neutron_decay_rate}
\end{equation}
Calcolando il valore di $\rho_{f}$ \cite{ref:BGSex} e sostituendo a $\Gamma$ l'inverso della vita media del neutrone $\tau_{n}=885.7 s$ si trova un valore di $G_{F}$ pari a:
\begin{equation}
        G^{n} _ { F } = 1.140 \times 10 ^ { - 5 } \mathrm { GeV } ^ { - 2 }
        \label{eq:gf_neutron_value}
\end{equation}
Questo valore discorda da quello ottenuto (Eq.~\ref{eq:gf_muon_value}) da un decadimento che coinvolge solo leptoni. Sembrò quindi inizialmente che interazioni coinvolgenti leptoni e quark fossero diverse, facendo così cadere l'ipotesi di universalità dell'interazione debole.

\subsection{Decadimenti deboli di particelle strane}
Poiché le interazioni deboli coinvolgono sia leptoni che adroni, occorre fare alcune classificazioni, basate sul fatto che nei processi dovuti alla WI siano coinvolti o meno dei leptoni. Nel caso di processi semi-leptonici o non-leptonici, si osservano decadimenti in cui non sono coinvolte particelle strane ($\Delta S=0$), oppure con decadimenti che violano la stranezza ($\Delta S=1$). Il decadimento debole delle particelle strane presenta alcune anomalie, indipendentemente che si considerino processi leptonici, semi-leptonici o non-leptonici.
\subsubsection{Decadimenti leptonici}
Si consideri il caso dei decadimenti puramente leptonici riportati in Tab.~\ref{tab:leptonic_decays}.
\begin{table}[!h]
        \begin{tabular}{llccc}
                \hline
                Decadimento & & $\Delta S$ & \tau\: (s)& $BR = \Gamma_{i}/\Gamma$    \\
                \hline
                $\pi^{-} \rightarrow \mu^{-} \overline{\nu_{\mu}}$ & $\overline{u}d \rightarrow W^{-} \rightarrow \mu^{-} \overline{\nu_{\mu}}$ & 0 & $2.6 \times 10^{-8}$ & $100\%$ \\
                $K^{-} \rightarrow \mu^{-} \overline{\nu_{\mu}}$ & $\overline{u}s \rightarrow W^{-} \rightarrow \mu^{-} \overline{\nu_{\mu}}$ & 1 & $1.27 \times 10^{-8}$ & $63.5\%$ \\
                \hline
        \end{tabular}
        \caption{Decadimenti leptonici di pione e kaone.}
        \label{tab:leptonic_decays}
\end{table}

\subsubsection{Decadimenti semi-leptonici}
In questo caso si ha che gli stati finali comprendono sia leptoni che adroni. Si considerino i decadimenti della $\Sigma^{-}$ mostrati in Tab.~\ref{tab:isemileptonic_decays}: la regola di Sargent (\ref{eq:sargent_rule}) fornisce una buona approssimazione per il calcolo della vita media. Tuttavia, anche in questo caso i risultati numerici sono corretti per i decadimenti con $\Delta S = 0$, ed errati di un fattore \sim 20 per i decadimenti con $\Delta S = 1$ . Di nuovo, si può immaginare che nel caso di decadimenti senza variazione di stranezza intervenga la costante $G_{d}$, mentre in quelli con $\Delta S = 1$ intervenga la costante $G_{s}$. Si può stimare il rapporto tra le due costanti usando i decadimenti della $\Sigma^{-}$:
\begin{equation}
        \frac{G_{s}^{2}}{G_{d}^{2}} = \frac{\Gamma(\Sigma^{-} \rightarrow n e^{-} \overline{\nu_{e}})/\Delta m^{5}_{\Delta S = 1}}{\Gamma(\Sigma^{-} \rightarrow \Lambda^{0} e^{-} \overline{\nu_{e}})/\Delta m^{5}_{\Delta S = 0}} = 0.057
        \label{eq:ratio_semileptonic}
\end{equation}

\begin{table}[!h]
        \begin{tabular}{llccc}
                \hline
                Decadimento & \Delta m\: (MeV) & $\Delta S$ & \tau\: (s)& $BR = \Gamma_{i}/\Gamma$    \\
                \hline
                $\Sigma^{-} \rightarrow \Lambda^{0} e^{-} \overline{\nu_{e}}$ & 81.7 & 0 & $1.48 \times 10^{-10}$ & $0.57 \times 10^{-4}$ \\
                $\Sigma^{-} \rightarrow n e^{-} \overline{\nu_{e}}$ & 257.8 & 1 & $1.48 \times 10^{-10}$ & $1.02 \times 10^{-3}$ \\
                \hline
        \end{tabular}
        \caption{Decadimenti semi-leptonici della $\Sigma^{-}$.}
        \label{tab:isemileptonic_decays}
\end{table}


\subsection{L'angolo di Cabibbo}
I fatti sperimentali sopracitati vennero interpretati da Nicola Cabibbo nel 1964. Egli mostrò che sia i leptoni che i quark sono autostati dell'interazione debole, con le seguenti assunzioni:
\begin{itemize}
        \item l'accoppiamento degli elettroni al campo debole è proporzionale ad una carica debole $g_{e\nu}$;
        \item l'accoppiamento dei muoni è proporzionale a $g_{\mu\nu}$, con quest'ultima identica a $g_{e\nu}$;
        \item l'accoppiamento dei quark $(u, d)$ genera le transizioni con $\Delta S = 0$ ed è proporzionale a $g_{ud}$;
        \item l'accoppiamento dei quark $(u, s)$ genera le transizioni con $\Delta S = 1$ ed è proporzionale a $g_{us}$;
\end{itemize}
In ogni vertice di un diagramma di Feynman occorre inserire la costante corrispondente, con il calcolo degli elementi di matrice che viene di conseguenza:
\begin{itemize}
        \item per i processi puramente leptonici si ha
                \begin{equation}
                        \left\langle f \left| H _ { W } \right| i \right\rangle \propto g _ { e v } ^ { 2 } = G _ { F }
                        \label{eq:genu}
                \end{equation}
        \item per i processi semi-leptonici con $\Delta S = 0$ si ha
                \begin{equation}
                \left\langle f \left| H _ { W } \right| i \right) _ { \Delta S = 0 } \propto g _ { e v } g _ { u d } = G _ { d }
                \label{eq:gd}
        \end{equation}
\item per i processi semi-leptonici con $\Delta S = 1$ si ha
        \begin{equation}
                \left\langle f \left| H _ { W } \right| i \right\rangle _ { \Delta S = 1 } \propto g _ { e v } g _ { u s } = G _ { s }
                \label{eq:gs}
        \end{equation}
\end{itemize}
L'ipotesi di Cabibbo è che l'interazione debole dipenda da un solo parametro, la costante di Fermi $G_{F}$. Questa descrive l'accoppiamento del campo debole sia verso i leptoni che verso i quark tramite la relazione:
\begin{equation}
        G _ { F } = g _ { e v } ^ { 2 } = g _ { u d } ^ { 2 } + g _ { u s } ^ { 2 } \longrightarrow g _ { u d } = g _ { e v } \cos \theta _ { C } ; \quad g _ { u s } = g _ { e v } \sin \theta _ { C }
\end{equation}
In questo modello, quanto sopra esposto corrisponde al fatto che i quark che partecipano all'interazione debole non sono gli autostati di sapore che caratterizzano l'interazione forte, ma una loro combinazione lineare, che può considerarsi ruotata di un angolo $\theta_{c}$ rispetto ai quark ordinari. Detti i nuovo autostati dell'interazione debole ($u_{c}, d_{c}, s_{c}$) si hanno così i seguenti doppietti deboli:
\begin{equation}
        \left( \begin{array} { c } { v _ { e } } \\ { e ^ { - } } \end{array} \right) , \left( \begin{array} { c } { v _ { \mu } } \\ { \mu ^ { - } } \end{array} \right) , \left( \begin{array} { c } { u } \\ { d _ { c } } \end{array} \right) = \left( \begin{array} { c } { u } \\ { d \cos \theta _ { c } + s \sin \theta _ { c } } \end{array} \right)
        \label{eq:weak_doublets}
\end{equation}
dove $\theta_{c}$ è l'\textit{angolo di Cabibbo} e si trova sperimentalmente essere $\theta _ { c } = 0.235\: \mathrm { rad }$. La scelta di $u$ come autostato non mescolato è semplicemente una convenzione. Per ogni doppietto di leptoni l'accoppiamento debole è specificato da $G_{F}$. L'apparente differenza tra i valori della costante di accoppiamento è dovuta al processo di miscelamento dei quark. \\
Si possono in questo modo interpretare i fatti sperimentali esposti in precedenza: dato che $\sin\theta_{c}=0.235$ e $\cos\theta_{c}=0.972$, transizioni con $\Delta S = 0$ hanno una costante di accoppiamento effettiva maggiore di transizioni con $\Delta S = 1$. Si spiega inoltre perché $G_{F}^{n}$ (Eq.~\ref{eq:gf_neutron_value}) è più piccola di $G_{F}^{\mu}$ (Eq.~\ref{eq:gf_muon_value}): in realtà, quello che si misura nel decadimento del neutrone è $G_{F}\cos \theta_{c}$.


\subsection{Tipi di transizione nel decadimento $\beta$}
Al fine di introdurre concetti che saranno ripresi nel seguito, si vuole ora trattare dei molteplici tipi di decadimento $\beta$ esistenti. La classificazione viene operata sulla base della variazione di momento angolare tra nucleo iniziale e nucleo finale. Tale variazione è connessa con lo spin dei due leptoni $e ^ { - } , \overline { \nu } _ { e }$ (oppure $e^{+}, \nu_{e}$): entrambi hanno spin 1/2, quindi la variazione dello spin nucleare può essere nulla (spin di neutrino ed elettrone antiparalleli), oppure $\pm1$ (spin paralleli). Si tiene conto del numero di stati permessi dal momento angolare della transizione tramite la grandezza $|\matchal{M}|^{2}$. Questo valore dipende in parte dalla variazione di spin nella transizione del nucleo. Assumendo che l'elettrone e il neutrino siano emessi in uno stato di momento angolare $\matchal{l}=0$, la variazione dello spin del nucleo è pari alla somma degli spin dei due leptoni. Per l'orientazione degli spin di elettrone e neutrino:
\begin{enumerate}
        \item nelle transizioni $0 \rightarrow 0$, gli spin sono antiparalleli (stato di singoletto);
        \item nelle transizioni $0 \rightarrow 1$ gli spin sono paralleli (stato di tripletto);
        \item nelle transizioni $\frac{1}{2} \rightarrow \frac{1}{2}$ gli spin possono essere antiparalleli (lo spin del nucleo non cambia) o paralleli (lo spin del nucleo cambia direzione).
\end{enumerate}
Le transizioni del primo tipo sono dette transizioni di Fermi, quelle del secondo di Gamow-Teller, mentre le terze sono dette di tipo misto.
\paragraph{Transizioni di Fermi} $\Delta J=0$, stato leptonico di singoletto di spin ($\uparrow\downarrow$). Un esempio di transizione di Fermi è:
\begin{equation}
        0 ^ { + } \rightarrow 0 ^ { + } , \Delta J = 0 : ^ { 14 } \mathrm { O } \rightarrow ^ { 14 } \mathrm { N } ^ { * } + e ^ { + } + \nu _ { e }
        \label{eq:fermi_ex}
\end{equation}
Facendo riferimento alle quantità introdotte in Sec. \ref{subsec:v-a}, si ha che queste transizioni possono essere solo di tipo $S$ e $V$. Per la probabilità si transizione si ha:
\begin{equation}
        | \mathcal { M } | ^ { 2 } \equiv \left| \mathcal { M } _ { F } \right| ^ { 2 } = m _ { i , s } g _ { V } ^ { 2 }
        \label{eq:m_fermi}
\end{equation}
dove $g_{V}$ è la costante di accoppiamento vettoriale.
\paragraph{Transizioni di Gamow-Teller} $\Delta J=1$, stato leptonico di tripletto di spin ($\uparrow\uparrow$). Un esempio di transizione di Gamow-Teller è:
\begin{equation}
        1 ^ { + } \rightarrow 0 ^ { + } , \quad \Delta J = 1 : ^{6}H e \rightarrow^ { 6 } L i + e ^ { - } + \overline { \nu }
        \label{eq:gt_ex}
\end{equation}
Facendo riferimento alle quantità introdotte in Sec. \ref{subsec:v-a}, si ha che queste transizioni possono essere solo di tipo $T$ e $A$. Per la probabilità si transizione si ha:
\begin{equation}
        | \mathcal { M } | ^ { 2 } \equiv \left| \mathcal { M } _ { G T } \right| ^ { 2 } = m _ { i , s } g _ { A } ^ { 2 }
        \label{eq:m_gt}
\end{equation}
dove $g_{A}$ è la costante di accoppiamento assiale. \\
Sfruttando i processi introdotti in (\ref{eq:fermi_ex}) e (\ref{eq:gt_ex}) è possibile stimare \cite{ref:BGSex} il rapporto tra le costanti $g_{V}$ e $g_{A}$, che tornerà utile nel seguito:
\begin{equation}
        \lambda = g _ { A } / g _ { v } = - 1.2695 \pm 0.0029
        \label{eq:ga_gv}
\end{equation}

\subsection{Decadimento del neutrone con effetti di \textit{quark mixing}}
Grazie all'introduzione del \textit{mixing} è possibile descrivere il decadimento del neutrone in termini di quark costituenti, elaborando così una teoria più sofisticata. Il diagramma di Feynman del decadimento è rappresentato in Fig. \ref{fig:neutron_decay_quarks}.
\begin{figure}[!h]
        \centering
        \feynmandiagram [layered layout, horizontal=a to b] {
                a [particle=\(d\)] -- [fermion] b -- [fermion] f1 [particle=\(u\)],
                b -- [boson, edge label'=\(W^{-}\)] c,
                c -- [anti fermion] f2 [particle=\(\overline \nu_{e}\)],
                c -- [fermion] f3 [particle=\(e^{-}\)],
        };
        \caption{Diagramma di Feynman per il decadimento del neutrone, assumendo protone e neutrone costituiti da quark.}
        \label{fig:neutron_decay_quarks}
\end{figure}
Il fattore al vertice quark-quark si può scrivere
\begin{equation}
        \mathcal { K } ^ { \mu } \rightarrow - \frac { i g _ { w } } { 2 \sqrt { 2 } } \gamma ^ { \mu } \left( g _ { v } - g _ { A } \gamma ^ { 5 } \right)
\end{equation}
Se la parte vettoriale è conservata, allora $g_{V} = 1$; altrimenti il rapporto tra le due costanti assume il valore (\ref{eq:ga_gv}). Nello scrivere l'elemento di matrice, è inoltre necessario moltiplicare per un fattore $\cos\theta_{c}$, così che si possa scrivere:
\begin{equation}
        \mathcal { M } = \frac { g _ { w } ^ { 2 } } { 8 M _ { W } ^ { 2 } } V _ { u d } \cdot \left[ \overline { u } ( p ) \gamma ^ { \mu } \left( g _ { v } - g _ { A } \gamma ^ { 5 } \right) u ( n ) \right] \cdot \left[ \overline { u } ( e ) \gamma _ { \mu } \left( 1 - \gamma ^ { 5 } \right) v \left( \nu _ { e } \right) \right]
\end{equation}
\section{L'estensione della teoria: correnti neutre, meccanismo GIM e matrice CKM}
\subsection{Interazione debole a corrente neutra}
Come anticipato nell'introduzione, i processi deboli a corrente carica (CC) non sono gli unici esistenti. Ci sono infatti reazioni, come ad esempio
\begin{equation}
        \begin{array} { c } { \nu _ { \mu } e ^ { - } \rightarrow \nu _ { \mu } e ^ { - } } \\ { \overline { \nu } _ { \mu } e ^ { - } \rightarrow \overline { \nu } _ { \mu } e ^ { - } } \end{array}
        \label{eq:neutral_current}
\end{equation}
che possono avvenire solo attraverso lo scambio di un bosone $Z^{0}$. Questi processi vengono chiamati interazioni deboli a corrente neutra (NC). \\
Come il fotone, anche lo $Z^{0}$ può essere scambiato sia nel canale $t$ che nel canale $s$: nel primo caso si ha lo scambio del bosone tra un neutrino e un fermione, con questi due che rimangono gli stessi durante tutto il processo, mentre nel secondo caso ha un processo di annichilazione $f \overline { f } \rightarrow Z ^ { 0 }$ seguito dalla creazione di una coppia $f ^ { \prime } \overline { f ^ { \prime } }$. \\
L'interazione a corrente neutra fu scoperta nel 1977 utilizzando una camera a bolle a liquido pesante (Gargamelle, CERN) esposta ad un fascio di neutrini muonici ad alta energia; furono osservate le reazioni sugli elettroni atomici $\nu _ { \mu } e ^ { - } \rightarrow \nu _ { \mu } e ^ { - }$ e su nucleoni $\nu _ { \mu } N \rightarrow \nu _ { \mu } + \text {adroni}$, in cui sono visibili nello stato finale le sole tracce delle particelle cariche. In entrambi i casi, visto che mancava la segnatura caratteristica della presenza di un muone, si concluse che le reazioni possono procedere solo via NC. \\
Storicamente, le NC sono state introdotte per rimuovere divergenze: nel caso, per esempio, della reazione $\nu \overline { \nu } \rightarrow W ^ { + } W ^ { - }$, il diagramma all'ordine più basso contiene lo scambio di un elettrone e dà luogo ad una divergenza, cancellata dal diagramma contenente lo scambio di una $Z^{0}$. \\
Si noti che le interazioni a corrente neutra non erano previste dalla teoria di Fermi e rappresentano uno dei motivi che ne richiesero l'estensione.


\subsection{Quark \textit{charm} e meccanismo GIM}
Nel 1955 Gell-Mann e Pais osservarono che è possibile formare due combinazioni lineari dei mesoni $K$ neutri che sono autostati della simmetria CP, e quindi dell'interazione debole, e che questi corrispondono alle particelle che decadono nei due diversi stati di CP. Scegliendo le seguenti combinazioni lineari:
\begin{equation}
        | K _ { 1 } ^ { 0 } \rangle = 1 / \sqrt { 2 } ( | K ^ { 0 } \rangle + | \overline { K } ^ { 0 } \rangle ) \quad ; \quad | K _ { 2 } ^ { 0 } \rangle = 1 / \sqrt { 2 } ( | K ^ { 0 } \rangle - | \overline { K } ^ { 0 } \rangle )
\end{equation}
si può vedere come questi siano due stati distinti, combinazioni degli autostati delle interazioni forti, che hanno masse diverse e decadono in modi diversi:
\begin{equation}
        \begin{array} { c c } { | K _ { 1 } ^ { 0 } \rangle } & { \frac { 1 } { \sqrt { 2 } } ( | d \overline { s } \rangle + | s \overline { d } \rangle ) \quad C P = + 1 \longrightarrow \pi \pi } \\ { | K _ { 2 } ^ { 0 } \rangle } & { \frac { 1 } { \sqrt { 2 } } ( | d \overline { s } \rangle - | s \overline { d } \rangle ) \quad C P = - 1 \longrightarrow \pi \pi \pi } \end{array}
\end{equation}
Il valore della differenza di massa tra $K^{0}_{1}$ e $K^{0}_{2}$ si può calcolare nel modello a quark ed è in particolare proporzionale all'elemento di matrice che descrive la probabilità di transizione $K ^ { 0 } \leftrightarrow \overline { K } ^ { 0 }$, che ha $\Delta S = 2$. Il calcolo di questa quantità, considerando il contributo dei soli quark $u, d, s$ fornisce tuttavia un valore molto più grande del risultato sperimentale. Per questo motivo si ipotizzò l'esistenza di qualche fenomeno che impedisce le transizioni in cui cambia il sapore dei quark ma non la carica elettrica. \\
Questo fatto fu messo in luce nel 1970 da Glashow, Iliopulos e Maiani, i quali proposero l'esistenza di un quarto quark (denominato \textit{c, charm}) con le seguenti proprietà:
\begin{enumerate}
        \item carica elettrica +2/3, isospin $I=0$, stranezza $S=0$ e numero barionico $B=1/3$;
        \item un nuovo numero quantico $C$ che, analogamente alla stranezza, si conserva nell'interazione forte ma non in quella debole;
        \item è autostato dell'interazione debole e forma un secondo doppietto di quark con il secondo stato ruotato della teoria di Cabibbo $s ^ { \prime } = s \operatorname { c o s } \theta _ { c } - d \operatorname { s i n } \theta _ { c }$.
\end{enumerate}
Con questa nuova ipotesi, si ha un nuovo doppietto per l'interazione debole:
\begin{equation}
        \left( \begin{array} { l } { c } \\ { s _ { c } } \end{array} \right) = \left( \begin{array} { c } { c } \\ { s \cos \theta _ { c } - d \sin \theta _ { c } } \end{array} \right)
        \label{eq:c_sc_doublet}
\end{equation}
Utilizzando le stessi notazioni usate in (\ref{eq:genu}), (\ref{eq:gd}) e (\ref{eq:gs}) è possibile descrivere i processi semi-leptonici in cui interviene il quark $c$ usando l'ipotesi del meccanismo GIM:
\begin{equation}
        \langle ( \nu _ { e } e ^ { + } ) s | H _ { W } | c \rangle \propto g _ { e \nu } g _ { c s } = G _ { F } \operatorname { c o s } \theta _ { c }
        \label{eq:gcs}
\end{equation}
\begin{equation}
        \left\langle \left( \nu _ { e } e ^ { + } \right) d \left| H _ { W } \right| c \right\rangle \propto g _ { e \nu } g _ { c d } = G _ { F } \sin \theta _ { c }
        \label{eq:gcd}
\end{equation}
da cui segue che nei decadimenti di mesoni charmati con $\Delta C = 1$ in mesoni non charmati, le transizioni $c \rightarrow s$ (che hanno accoppiamento $\cos^{2}\theta_{c}$) dominano sulle transizioni $c \rightarrow d$ (accoppiamento $\sin^{2}\theta_{c}$). \\
Riassumendo, gli stati ruotati autostati dell'interazione debole che compaiono nelle due famiglie di quark
\begin{equation}
        \left( \begin{array} { l } { u } \\ { d_{c} } \end{array} \right)  \quad \left( \begin{array} { l } { c } \\ { s_{c} } \end{array} \right)
\end{equation}
si possono scrivere in forma matriciale
\begin{equation}
        \left( \begin{array} { l } { d_{c} } \\ { s_{c} \end{array} \right) = \left( \begin{array} { c c } { \cos \theta _ {c} } & { \sin \theta _ { c } } \\ { - \sin  \theta _ { c } } & { \cos \theta _ { c } } \end{array} \right) \left( \begin{array} { l } { d } \\ { s } \end{array} \right)
\end{equation}
Grazie a questa scrittura, il termine dell'hamiltoniana $H_{W}$ che dipende dal tipo di quark coinvolto può essere formalmente scritto nel modo seguente (considerando come esempio transizioni da un quark iniziale $d_{c}$ o $s_{c}$ ad uno stato finale $\overline{u}$ o $\overline{c}$):
\begin{equation}
        ( \overline { u } , \overline { c } ) \left( \begin{array} { c } { d _ { c } } \\ { s _ { c } } \end{array} \right) = ( \overline { u } , \overline { c } ) \left( \begin{array} { c } { \operatorname { c o s } \theta _ { c } \operatorname { s i n } \theta _ { c } } \\ { - \operatorname { s i n } \theta _ { c } \operatorname { c o s } \theta _ { c } } \end{array} \right) \left( \begin{array} { l } { d } \\ { s } \end{array} \right) = ( \overline { u } , \overline { c } ) V _ { c } \left( \begin{array} { l } { d } \\ { s } \end{array} \right)
\end{equation}

\begin{figure}[!h]
        \centering
        \feynmandiagram [layered layout, horizontal=a to b] {
                a [particle=\(c\)] -- [fermion] b [label=0:\(g_{cd} (g_{cs})\)]-- [fermion] f1 [particle=\(d (s)\)],
                b -- [boson, edge label'=\(W^{+}\)] c,
                c -- [anti fermion] f2 [particle=\(\nu_{e}\)],
                c -- [fermion] f3 [particle=\(e^{+}\)],
        };
        \caption{Interazione debole con un quark $c$ che si trasforma in $d$ o $s$.}
        \label{fig:gim_decay}
\end{figure}

\subsection{Indizi sul quarto quark dalle correnti neutre}

\subsection{I sei quark e la matrice CKM}
Nel 1973 Kobayashi e Maskawa estesero l'idea di Cabibbo a tre generazioni di quark. Si hanno così i seguenti autostati dell'interazione debole:
\begin{equation}
        \left( \begin{array} { l } { u } \\ { d_{c}  } \end{array} \right)  \quad \left( \begin{array} { l } { c } \\ { s_{c} \end{array} \right) , \quad \left( \begin{array} { l } { t } \\ { b_{c} } \end{array} \right)
\end{equation}
che sono legati agli stati fisici dalla matrice di Cabibbo-Kobayashi-Maskawa:
\begin{equation}
        \left( \begin{array} { l } { d _ { c } } \\ { s _ { c } } \\ { b _ { c } } \end{array} \right) = V _ { C K M } \left( \begin{array} { l } { d } \\ { s } \\ { b } \end{array} \right)
\end{equation}
dove
\begin{equation}
        V_{CKM} = \left( \begin{array} { c } { V _ { u d } V _ { u s } V _ { u b } } \\ { V _ { c d } V _ { c s } V _ { c b } } \\ { V _ { t d } V _ { t s } V _ { t b } } \end{array} \right)
        \label{eq:ckm}
\end{equation}
dove ogni elemento specifica l'accoppiamento tra i due quark scritti a pedice. La parametrizzazione proposta originariamente è la seguente ($c_{i}=\cos\theta_{i}, s_{i}=\sin\theta_{i}$):
\begin{equation}
V _ { C K M } = \left( \begin{array} { c c c } { c _ { 1 } } & { c _ { 3 } s _ { 1 } } & { s _ { 1 } s _ { 3 } } \\ { - c _ { 2 } s _ { 1 } } & { c _ { 1 } c _ { 2 } c _ { 3 } - s _ { 2 } s _ { 3 } e ^ { i \delta } } & { c _ { 1 } c _ { 2 } s _ { 3 } + c _ { 3 } s _ { 2 } e ^ { i \delta } } \\ { s _ { 1 } s _ { 2 } } & { - c _ { 1 } c _ { 3 } s _ { 2 } - c _ { 2 } s _ { 3 } e ^ { i \delta }} & {- c _ { 1 } s _ { 2 } s _ { 3 } + c _ { 2 } c _ { 3 } e ^ { i \delta } } \end{array} \right)
        \label{eq:ckm_param}
\end{equation}
dove $\theta _ { 1 } , \theta _ { 2 } , \theta _ { 3 }$ sono tre angoli di mixing e $\delta$ è un angolo di fase che, se diverso da zero, porta alla violazione di $CP$ nell'interazione debole. Gli elementi della matrice CKM, determinati sperimentalmente, sono i seguenti:
\begin{equation}
        V _ { C K M } = \left( \begin{array} { c c c } { 0.97428 \pm 0.00015 } & { 0.2253 \pm 0.0007 } & { 0.00347 \pm 0.00016 } \\ { 0.2252 \pm 0.0007 } & { 0.97333 \pm 0.00015 } & { 0.041 \pm 0.001 } \\ { 0.0086 \pm 0.0003 } & { 0.040 \pm 0.001 } & { 0.99915 \pm 0.00005 } \end{array} \right)
        \label{eq:ckm_values}
\end{equation}
Il fatto che gli elementi non diagonali siano molto piccoli mentre quelli sulla diagonale siano vicini all'unità ha un significato fisico importante, cioè che il modello predice una specifica sequenza di decadimenti:
\begin{equation}
        t \rightarrow \mathrm { b } \rightarrow \mathrm { c } \rightarrow \mathrm { s } \rightarrow \mathrm { u }
\end{equation}
Con lo schema di mixing, quark e leptoni hanno lo stesso accoppiamento e si può così parlare di universalità quark-leptoni.



\appendix

\section{Equazione e formalismo di Dirac}
\label{app:dirac}
\subsection{Equazione e matrici $\gamma$}
L'equazione di Dirac (1928) descrive il comportamento quanto-meccanico e relativistico di particelle puntiformi massive con spin semintero:
\begin{equation}
        \left( i \gamma ^ { \mu } \partial _ { \mu } - m \right) \psi = 0
        \label{eq:dirac_equation}
\end{equation}
dove $\psi$ è un vettore colonna a quattro componenti (spinore)
\begin{equation}
        \psi = \left( \begin{array} { c } { \psi _ { 1 } } \\ { \psi _ { 2 } } \\ { \psi _ { 3 } } \\ { \psi _ { 4 } } \end{array} \right)
        \label{eq:spinor}
\end{equation}
e $\gamma^{\mu} = \left\{ \gamma ^ { 0 } , \gamma ^ { 1 } , \gamma ^ { 2 } , \gamma ^ { 3 } \right\}$ è un set di quattro matrici $4\times 4$ che, seguendo la rappresentazione di Dirac, si scrivono:
\begin{equation}
        \gamma ^ { 0 } = \left( \begin{array} { c c } { 1 } & { 0 } \\ { 0 } & { - 1 } \end{array} \right) \quad \gamma ^ { i } = \left( \begin{array} { c c } { 0 } & { \sigma ^ { i } } \\ { - \sigma ^ { i } } & { 0 } \end{array} \right)
        \label{eq:gamma_components}
\end{equation}
con le $\sigma^{i}$ matrici di Pauli:
\begin{equation}
        \sigma _ { x } = \left( \begin{array} { c c } { 0 } & { 1 } \\ { 1 } & { 0 } \end{array} \right) \quad \sigma _ { y } = \left( \begin{array} { c c } { o } & { - i } \\ { i } & { 0 } \end{array} \right) \quad \sigma _ { z } = \left( \begin{array} { c c } { 1 } & { 0 } \\ { 0 } & { - 1 } \end{array} \right)
        \label{eq:pauli_matrices}
\end{equation}
Scritte per esteso, le matrici $\gamma$ saranno:
\begin{equation}
        \begin{array} {c} \gamma ^ { 0 } = \left( \begin{array} { c c c c } { 1 } & { 0 } & { 0 } & { 0 } \\ { 0 } & { 1 } & { 0 } & { 0 } \\ { 0 } & { 0 } & { - 1 } & { 0 } \\ { 0 } & { 0 } & { 0 } & { - 1 } \end{array} \right)  \quad \gamma ^ { 1 } = \left( \begin{array} { c c c c } { 0 } & { 0 } & { 0 } & { 1 } \\ { 0 } & { 0 } & { 1 } & { 0 } \\ { 0 } & { - 1 } & { 0 } & { 0 } \\ { - 1 } & { 0 } & { 0 } & { 0 } \end{array} \right) \\ \gamma ^ { 2 } = \left( \begin{array} { c c c c } { 0 } & { 0 } & { 0 } & { - i } \\ { 0 } & { 0 } & { i } & { 0 } \\ { 0 } & { i } & { 0 } & { 0 } \\ { - i } & { 0 } & { 0 } & { 0 } \end{array} \right)  \quad \gamma ^ { 3 } = \left( \begin{array} { c c c c } { 0 } & { 0 } & { 1 } & { 0 } \\ { 0 } & { 0 } & { 0 } & { - 1 } \\ { - 1 } & { 0 } & { 0 } & { 0 } \\ { 0 } & { 1 } & { 0 } & { 0 } \end{array} \right) \end{array}
        \label{eq:gamma_extended}
\end{equation}
con
\begin{equation}
        \begin{array} { c } { \left( \gamma ^ { 0 } \right) ^ { 2 } = 1 , \quad \left( \gamma ^ { 1 } \right) ^ { 2 } = \left( \gamma ^ { 2 } \right) ^ { 2 } = \left( \gamma ^ { 3 } \right) ^ { 2 } = - 1 } \\ { \gamma ^ { \mu } \gamma ^ { \nu } + \gamma ^ { \nu } \gamma ^ { \mu } = 0 , \quad \text { per } \mu \neq \nu } \end{array}
        \label{eq:gamma_properties}
\end{equation}
Si evidenziano inoltre i seguenti fatti e proprietà:
\begin{itemize}
        \item in molti casi è utile definire la matrice:
                \begin{equation}
                        \gamma ^ { 5 } : = i \gamma ^ { 0 } \gamma ^ { 1 } \gamma ^ { 2 } \gamma ^ { 3 } = \left( \begin{array} { l l l l } { 0 } & { 0 } & { 1 } & { 0 } \\ { 0 } & { 0 } & { 0 } & { 1 } \\ { 1 } & { 0 } & { 0 } & { 0 } \\ { 0 } & { 1 } & { 0 } & { 0 } \end{array} \right)
                        \label{eq:gamma5}
                \end{equation}
                con
                \begin{equation}
                        \begin{array} { c } { \left( \gamma ^ { 5 } \right) ^ { 2 } = 1 , \quad { \gamma ^ { 5 } \gamma ^ { \mu } + \gamma ^ { \mu } \gamma ^ { 5 } = 0 \end{array}
                                        \label{eq:gamma5_properties}
                                \end{equation}
                        \item definendo lo spinore aggiunto:
                                \begin{equation}
                                        \overline { \psi } \equiv \psi ^ { \dagger } \gamma ^ { 0 } = \psi ^ { T ^ { * } } \gamma ^ { 0 }
                                        \label{eq:spinor_adj}
                                \end{equation}
                                è possibile costruire le grandezze relativisticamente invarianti elencate in Tab.~\ref{tab:bilinear};
                        \item esiste una funzione $\psi^{\prime}$ rappresentabile nella forma:
                                \begin{equation}
                                        \psi ^ { \prime } = S \psi \quad ; \quad \operatorname { con } S ^ { - 1 } \gamma ^ { \mu } S = \sum _ { \nu } a _ { \mu \nu } \gamma ^ { \nu }
                                        \label{eq:s_operator}
                                \end{equation}
                                per la quale si ha Lorentz-invarianza dell'equazione di Dirac; questa funzione viene utilizzata per dimostrare \cite{ref:BGSex} la Lorentz-invarianza delle grandezze citate nel punto precedente ;
                        \item $\gamma^{0}$ rappresenta l'operatore Parità, per il quale si ha:
                                \begin{equation}
                                        \gamma ^ { 0 } \psi ( - \mathbf { r } , t ) \equiv \psi ( \mathbf { r } , t ) \quad ; \quad \gamma ^ { 0 } \psi ( \mathbf { r } , t ) \equiv \psi ( - \mathbf { r } , t )
                                \end{equation}
                        \item dato $\mathcal$ operatore di coniugazione complessa, si ha che $\gamma ^ { 1 } \gamma ^ { 3 } \mathcal { K }$ rappresenta l'operatore di inversione temporale, per il quale si ha:
                                \begin{equation}
                                        \gamma ^ { 1 } \gamma ^ { 3 } \mathcal { K } \psi ( \mathbf { r } , t ) = \psi ( \mathbf { r } , - t )
                                \end{equation}
                        \item $i \gamma ^ { 2 } \mathcal { K }$ rappresenta l'operatore di coniugazione di carica e parità CP, per il quale si ha:
                                \begin{equation}
                                        \left( i \gamma ^ { 2 } \mathcal { K } \right) \psi _ { + } = \psi _ { - } ( - \mathbf { p } )
                                \end{equation}
                \end{itemize}

                \subsection{Proprietà delle soluzioni}
                L'equazione di Dirac è un insieme di quattro equazioni ognuna delle quali deve avere una soluzione.
                Tali soluzioni si scrivono nella seguente maniera (utilizzando le unità naturali):
                \begin{equation}
                        \psi = u e ^ { i ( \mathbf { p } \cdot \mathbf { r } - E t ) } = u e ^ { - i p _ { \mu } x ^ { \mu } }
                        \label{eq:dirac_sol}
                \end{equation}
                dove $u$ è uno spinore a quattro componenti che dipende dal momento. Sostituendo queste soluzioni nella \ref{eq:dirac_equation} si ottiene:
                \begin{equation}
                        \left( \gamma ^ { \mu } p _ { \mu } - m \right) u = 0
                        \label{eq:dirac_eq_momentum}
                \end{equation}
                detta anche equazione di Dirac nello spazio dei momenti. Sviluppando i conti e utilizzando la rappresentazione
                \begin{equation}
                        \psi = \left( \begin{array} { l } { u_{A} } \\ { u_{B} } \end{array} \right) e ^ { i ( \mathbf { p } \cdot \mathbf { r } - E t ) }
                        \label{eq:dirac_sol_spinors}
                \end{equation}
                dove $u_{A}$ e $u_{B}$ sono spinori a due componenti, si ottiene:
                \begin{equation}
                        u _ { A } = \frac { \vec { \sigma } \cdot \vec { p } } { E - m } u _ { B } \quad u _ { B } = \frac { \vec { \sigma } \cdot \vec { p } } { E + m } u _ { A }
                \end{equation}
                Sostituendo la prima nella seconda (o viceversa), si può vedere come questo sia un sistema omogeneo che ammette soluzione solo se
                \begin{equation}
                        E = \pm \sqrt { p ^ { 2 } + m ^ { 2 } }
                        \label{eq:energy}
                \end{equation}
                Scegliendo per $u_{A}$ e $u_{B}$ gli autostati dell'operatore $\sigma_{3}$, $u _ { A } = \left( \begin{array} { c } { 1 } \\ { 0 } \end{array} \right) , u _ { A } = \left( \begin{array} { l } { 0 } \\ { 1 } \end{array} \right), u _ { B } = \left( \begin{array} { c } { 1 } \\ { 0 } \end{array} \right) , u _ { B } = \left( \begin{array} { l } { 0 } \\ { 1 } \end{array} \right)$, si ottengono:
                \begin{equation}
                        \begin{align} u ^ { 1 } = \left( \begin{array} { c } { 1 } \\ { 0 } \\ { p _ { z } / ( E + m ) } \\ { \left( p _ { x } + i p _ { y } \right) / ( E + m ) } \end{array} \right) \quad u ^ { 2 } = \left( \begin{array} { c } { 0 } \\ { 1 } \\ { \left( p _ { x } - i p _ { y } \right) / ( E + m ) } \\ { - p _ { z } / ( E + m ) } \end{array} \right) \\ u ^ { 3 } = \left( \begin{array} { c } { - p _ { z } / ( - E + m ) } \\ { \left( - p _ { x } - i p _ { y } \right) / ( - E + m ) } \\ { 1 } \\ { 0 } \end{array} \right) \quad u ^ { 4 } = \left( \begin{array} { c } { \left( - p _ { x } + i p _ { y } \right) / ( - E + m ) } \\ { p _ { z } / ( - E + m ) } \\ { 0 } \\ { 1 } \end{array} \right) \end{align}
                        \label{eq:u1u2u3u4}
                \end{equation}
                Le soluzioni $u_{1}$ e $u_{2}$ descrivono particelle di energia (\ref{eq:energy}) positiva e momento $\boldsymbol{p}$. Per le due rimanenti, che descrivono particelle ad energia negativa, si applica:
                \begin{equation}
                        \begin{align} v ^ { 2 } \left( p ^ { \mu } \right) \equiv u ^ { 3 } \left( - p ^ { \mu } \right) = \left( \begin{array} { c } { p _ { z } / ( E + m ) } \\ { \left( p _ { x } + i p _ { y } \right) / ( E + m ) } \\ { 1 } \\ { 0 } \end{array} \right) \\ v ^ { 1 } \left( p ^ { \mu } \right) \equiv u ^ { 4 } \left( - p ^ { \mu } \right) = \left( \begin{array} { c } { \left( p _ { x } - i p _ { y } \right) / ( E + m ) } \\ { - p _ { z } / ( E + m ) } \\ { 0 } \\ { 1 } \end{array} \right) \end{align}
                        \label{eq:v1v2}
                \end{equation}
                così da introdurre l'interpretazione di antiparticelle ad energia positiva. Utilizzando quanto appena riportato, le soluzioni della (\ref{eq:dirac_equation}) si scrivono:
                \begin{equation}
                        \begin{align} \psi = u ^ { 1 } \left( p ^ { \mu } \right) e ^ { - i p_{\mu} x^{\mu} } \quad \psi = u ^ { 2 } \left( p ^ { \mu } \right) e ^ { - i p_{\mu} x^{\mu} } \\ \psi = v ^ { 1 } \left( p ^ { \mu } \right) e ^ { - i p_{\mu}  x^{\mu} } \quad \psi = v ^ { 2 } \left( p ^ { \mu } \right) e ^ { - i p_{\mu} \cdot x^{\mu} } \end{align}
                        \label{eq:u1u2v1v2}
                \end{equation}
                \subsection{Spin, elicità e chiralità}
                Ricordando che
                \begin{equation}
                        \mathbf { S } = \frac { \hbar } { 2 } \mathbf { \Sigma } , \quad \text { con } \mathbf { \Sigma } = \left( \begin{array} { l l } { \sigma } & { 0 } \\ { 0 } & { \sigma } \end{array} \right)
                \end{equation}
                si può facilmente verificare che $u^{1}, u^{2}, v^{1}, v^{2}$ sono autostati di $\hat { S _ { z } }$ solo se il momento è lungo l'asse $z$. \\
                In maniera più generale, è conveniente definire l'operatore elicità, che estrae la componente dello spin lungo la direzione del moto di una particella:
                \begin{equation}
                        \hat { h } = \frac { \vec { S } \cdot \vec { p } } { | \vec { S } | | \vec { p } | } = \frac { 2 \vec { S } \cdot \vec { p } } { | \vec { p } | }
                        \label{eq:helicity}
                \end{equation}
                Per fermioni di spin semintero, gli autovalori di $\hat{h}$ possono essere $h=1$ ("particella destrorsa") e $h=-1$ ("particella sinistrorsa"). Si noti che:
                \begin{itemize}
                        \item fermioni non massivi sono puramente sinistrorsi ($u^{2}$), mentre fermioni massivi necessitano di una sovrapposizione degli stati $u^{1}$ e $u^{2}$;
                        \item antifermioni non massivi sono puramente destrorsi ($v^{1}$), mentre antifermioni non massivi necessitano di una sovrapposizione degli stati $v^{1}$ e $v^{2}$.
                \end{itemize}
                Non essendo l'elicità una quantità Lorentz-invariante, risulta conveniente introdurre una quantità che lo sia: la chiralità. Essa può essere definita attraverso i seguenti operatori di proiezione:
                \begin{equation}
                        P _ { L } = \frac { 1 } { 2 } \left( 1 - \gamma _ { 5 } \right) \quad P _ { R } = \frac { 1 } { 2 } \left( 1 + \gamma _ { 5 } \right)
                        \label{eq:plpr}
                \end{equation}
                $P_{L}$ e $P_{R}$ agiscono rispettivamente estraendo le componenti sinistrorse e destrorse di un fermione
                \begin{equation}
                        u _ { L } = P _ { L } u \quad u _ { R } = P _ { R } u
                \end{equation}
                e le componenti destrorse e sinistrorse di un antifermione
                \begin{equation}
                        v _ { R } = P _ { L } v \quad v _ { L } = P _ { R } v
                \end{equation}
                Si noti infine che nel caso relativistico elicità e chiralità coincidono.

                \begin{thebibliography}{0}
                        \bibitem{ref:BGS}
                                \BY{Braibant S., Giacomelli G., Spurio M.}
                                \TITLE{Particelle e interazioni fondamentali},
                                \PUBLISHER{Springer},
                                \YEAR{2012}
                        \bibitem{ref:griff}
                                \BY{Griffiths D.}
                                \TITLE{Introduction to elementary particles},
                                \YEAR{1987}
                        \bibitem{ref:hayes}
                                \BY{Hayes C.B.}
                                \TITLE{Neutron Beta-Decay},
                                \YEAR{2012}
                        \bibitem{ref:BGSex}
                                \BY{Braibant S., Giacomelli G., Spurio M.}
                                \TITLE{Particles and Fundamental Interactions: Supplements, Problems and Solutions},
                                \PUBLISHER{Springer},
                                \YEAR{2012}
                \end{thebibliography}

                \end{document}

%%
