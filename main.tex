%%%%%%%%%%%%%%%%%%%%%%%%%%%%%%%%%%%%%%%%%%%%%%%%%%
%%%%%%%%%%%%%%%%%%%%%%%%%%%%%%%%%%%%%%%%%%%%%%%%%%
%%%%%%%%%%%%%%%%%%%%%%%%%%%%%%%%%%%%%%%%%%%%%%%%%%
\documentclass{subnucbo}

\usepackage{graphicx}
% got figures? uncomment this

% lingua italiana
\usepackage[italian]{babel}
\usepackage[utf8x]{inputenc}

% feynman diagrams
\usepackage[compat=1.1.0]{tikz-feynman}

\title{Processi Deboli e Mixing dei Quark}
% Insert your title here

\author{Massimiliano Galli}
% Insert your name(s) here.  Quick note: Authors must be separated
% with \from{n} to mark their referring
% institute, e.g.:
% \author{A.U. Thor\from{1}, W.R. Iter\from{2} \atque A. Einstein\from{3}\thanks{Thanks here}}
% TeX automagically determines the institute you are referring to.

\instlist{\inst{} Dipartimento di Fisica Universit\`a di Bologna, Via Irnerio 46 - 40126 Bologna, Italy}
% Insert your institute(s) here. Quick note: as above, e.g.
%\instlist{\inst{1} Dipartimento di Fisica Universit\`a di Bologna, Via Irnerio 46 - 40126 Bologna, Italy
%\inst{2} Dipartimento di Fisica dell'Universit\`{a} and INFN, Sezione di Milano - Milano, Italy
%\inst{3} Societ\`a Italiana di Fisica}

\acyear{2017--2018}
% Insert the academic year


\begin{document}

\maketitle

\begin{abstract}
        This sample paper is intended to briefly expose \LaTeX\ package \texttt{subnucbo}.
\end{abstract}

\section{Introduzione}
In questa sezione si introducono nozioni alle quali si farà riferimento in seguito.\\
Attraverso la teoria perturbativa è possibile ricavare la probabilità di transizione da uno stato iniziale \textit{i} ad uno stato finale \textit{f}:
\begin{equation}
        \Gamma = \frac{2 \pi} {\hbar} \cdot | M_{if}|^{2} \rho_{f}
        \label{eq:decay_rate}
\end{equation}
dove $\rho_{f}$ è la densità degli stati per unità di intervallo di energia, mentre $M_{if}$ è l'elemento di matrice per la probabilità di transizione ed è dato da:
\begin{equation}
        M_{if} = \int \phi^{*}_{i} V \phi_{f} d\tau
        \label{eq:matrix_element}
\end{equation}
con $V$ potenziale responsabile della transizione da uno stato all'altro. \\
Si introduce infine la regola di Sargent~\cite{ref:BGS}: se $\tau$ è la vita media di una particella e $\Gamma_{i}/\Gamma$ è il suo \textit{branching ratio} in un particolare decadimento debole a tre corpi nello stato finale. allora la probabilità di transizione corrisponde a:
\begin{equation}
        W = \frac{(\Gamma_{i}/\Gamma)}{\tau} \simeq G_{F}^{2}E_{0}^{5} \simeq G_{F}^{2}\Delta m^{5}
        \label{eq:sargent_rule}
\end{equation}

\section{La teoria di Fermi del decadimento $\beta$}
Il decadimento beta dei nuclei rappresenta una trasmutazione di un elemento (\textit{z}, \textit{n}), ove \textit{z} è il numero di protoni ed \textit{n} quello di neutroni del nucleo verso un nucleo con $\textit{z} + 1$ protoni (decadimento beta negativo) oppure con $\textit{z} - 1$ protoni (decadimento beta positivo). Per descriverlo, Fermi sviluppò a partire dal 1934 una teoria basata sull'ipotesi di interazione puntiforme tra quattro fermioni. Due interazioni che possono essere descritte da questa teoria sono il decadimento del muone e quello del neutrone.

\subsection{Decadimento del muone}
Il muone decade secondo:
\begin{equation}
        \mu^{-} \rightarrow e^{-} \overline{\nu_{e}} \nu_{\mu}
        \label{eq:muon_decay}
\end{equation}
Il diagramma di Feynman del processo è mostrato in Fig. \ref{fig:muon_decay}.
\begin{figure}[!h]
        \centering
        \feynmandiagram [layered layout, horizontal=a to b] {
                a [particle=\(\mu^{-}\)] -- [fermion] b -- [fermion] f1 [particle=\(\nu_{\mu}\)],
                b -- [boson, edge label'=\(W^{-}\)] c,
                c -- [anti fermion] f2 [particle=\(\overline \nu_{e}\)],
                c -- [fermion] f3 [particle=\(e^{-}\)],
        };
        \caption{Diagramma di Feynman per il decadimento del muone.}
        \label{fig:muon_decay}
\end{figure}



\subsection{Decadimento del neutrone}

\begin{equation}
        n \rightarrow p e^{-} \overline{\nu_{e}}
        \label{eq:neutron_decay}
\end{equation}

L'assunzione di interazione puntiforme fa sì che il propagatore bosonico \ref{eq:matrix_element} sia semplicemente una costante; si pone dunque:
\begin{equation}
        |M_{if}|^{2} = G_{F}^{2}|\mathcal{M}|^{2}
        \label{eq:mat_element_fermi_const}
\end{equation}
dove $G_{F}$ è la costante di Fermi, che occorre determinare, mentre $|\mathcal{M}|^{2}$ è una costante numerica adimensionale, dell'ordine dell'unità, caratteristica del processo considerato; in questo modo la (\ref{eq:decay_rate}) si può riscrivere:
\begin{equation}
        \Gamma = \frac{2 \pi} {\hbar} \cdot G_{F}^{2}|\mathcal{M}|^{2} \rho_{0}
        \label{eq:decay_rate_neutron}
\end{equation}


\section{Decadimenti deboli di particelle strane}
Poiché le interazioni deboli coinvolgono sia leptoni che adroni, occorre fare alcune classificazioni, basate sul fatto che nei processi dovuti alla WI siano coinvolti o meno dei leptoni. Nel caso di processi semi-leptonici o non-leptonici, si osservano decadimenti in cui non sono coinvolte particelle strane ($\Delta S=0$), oppure con decadimenti che violano la stranezza ($\Delta S=1$). Il decadimento debole delle particelle strane presenta alcune anomalie, indipendentemente che si considerino processi leptonici, semi-leptonici o non-leptonici.
\subsection{Decadimenti leptonici}
Si consideri il caso dei decadimenti puramente leptonici riportati in Tab.~\ref{tab:leptonic_decays}.
\begin{table}[!h]
        \begin{tabular}{llccc}
                \hline
                Decadimento & & $\Delta S$ & \tau\: (s)& $BR = \Gamma_{i}/\Gamma$    \\
                \hline
                $\pi^{-} \rightarrow \mu^{-} \overline{\nu_{\mu}}$ & $\overline{u}d \rightarrow W^{-} \rightarrow \mu^{-} \overline{\nu_{\mu}}$ & 0 & $2.6 \times 10^{-8}$ & $100\%$ \\
                $K^{-} \rightarrow \mu^{-} \overline{\nu_{\mu}}$ & $\overline{u}s \rightarrow W^{-} \rightarrow \mu^{-} \overline{\nu_{\mu}}$ & 1 & $1.27 \times 10^{-8}$ & $63.5\%$ \\
                \hline
        \end{tabular}
        \caption{Decadimenti leptonici di pione e kaone.}
        \label{tab:leptonic_decays}
\end{table}

\subsection{Decadimenti semi-leptonici}
In questo caso si ha che gli stati finali comprendono sia leptoni che adroni. Si considerino i decadimenti della $\Sigma^{-}$ mostrati in Tab.~\ref{tab:isemileptonic_decays}: la regola di Sargent (\ref{eq:sargent_rule}) fornisce una buona approssimazione per il calcolo della vita media. Tuttavia, anche in questo caso i risultati numerici sono corretti per i decadimenti con $\Delta S = 0$, ed errati di un fattore \sim 20 per i decadimenti con $\Delta S = 1$ . Di nuovo, si può immaginare che nel caso di decadimenti senza variazione di stranezza intervenga la costante $G_{d}$, mentre in quelli con $\Delta S = 1$ intervenga la costante $G_{s}$. Si può stimare il rapporto tra le due costanti usando i decadimenti della $\Sigma^{-}$:
\begin{equation}
        \frac{G_{s}^{2}}{G_{d}^{2}} = \frac{\Gamma(\Sigma^{-} \rightarrow n e^{-} \overline{\nu_{e}})/\Delta m^{5}_{\Delta S = 1}}{\Gamma(\Sigma^{-} \rightarrow \Lambda^{0} e^{-} \overline{\nu_{e}})/\Delta m^{5}_{\Delta S = 0}} = 0.057
        \label{eq:ratio_semileptonic}
\end{equation}

\begin{table}[!h]
        \begin{tabular}{llccc}
                \hline
                Decadimento & \Delta m\: (MeV) & $\Delta S$ & \tau\: (s)& $BR = \Gamma_{i}/\Gamma$    \\
                \hline
                $\Sigma^{-} \rightarrow \Lambda^{0} e^{-} \overline{\nu_{e}}$ & 81.7 & 0 & $1.48 \times 10^{-10}$ & $0.57 \times 10^{-4}$ \\
                $\Sigma^{-} \rightarrow n e^{-} \overline{\nu_{e}}$ & 257.8 & 1 & $1.48 \times 10^{-10}$ & $1.02 \times 10^{-3}$ \\
                \hline
        \end{tabular}
        \caption{Decadimenti semi-leptonici della $\Sigma^{-}$.}
        \label{tab:isemileptonic_decays}
\end{table}

\appendix

\section{}
Let us go then, you and I\ldots

\acknowledgments
This work was produced, supported and perpetrated by M. Bellacosa under
the auspices of the Italian Physical Society.

\begin{thebibliography}{0}
        \bibitem{ref:BGS}
                \BY{Braibant S., Giacomelli G., Spurio M.}
                \TITLE{Particelle e interazioni fondamentali},
                \PUBLISHER{Springer},
                \YEAR{2012}
\end{thebibliography}

\end{document}

%%
